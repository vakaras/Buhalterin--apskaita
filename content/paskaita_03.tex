\chapter{Uždaviniai: Pajamų, gautų iš autorinių sutarčių, apmokestinimas}

\begin{tasks}

  \begin{task}
    \begin{condition}
      Apskaičiuokite autoriaus ir atlygį išmokančio asmens, UAB „X“,
      visus Jums žinomus mokėtinus mokesčius, jeigu pastaroji su
      darbo santykiais nesusijusiam asmeniui pažadėjo 350 Lt autorinio
      atlyginimo už žurnalo moterims viršelio maketą.
    \end{condition}
    \begin{solution}
      Tuos $350 Lt$ jis gaus į rankas. Be to yra nesusijęs darbo
      santykiais todėl:

      \pythonba{autorine|be|darbo}{350 rankos}

    \end{solution}
  \end{task}

  \begin{task}
    \begin{condition}
      Apskaičiuokite autoriaus ir atlygį išmokančio asmens, UAB „X“, visus
      Jums žinomus mokėtinus mokesčius, jeigu pastaroji savo darbuotojui,
      kuris pagal darbo sutartį yra techninis redaktorius, už
      parašytą straipsnį žurnalui į jo banko sąskaitą pervedė 500 Lt.
    \end{condition}
    \begin{solution}
      Tie $500 Lt$ yra į rankas. Kadangi susijęs darbo santykiais, tai:
      
      \pythonba{autorine|su|darbu}{500 rankos}

    \end{solution}
  \end{task}

  \begin{task}
    \begin{condition}
      Apskaičiuokite autoriaus ir atlygį išmokančio asmens, UAB „X“, su
      kuria yra sudaryta darbo sutartis, pagal kurią darbuotojas rašo
      straipsnius žurnalui ir gauna 1200 Lt mėnesinio atlyginimo,
      visus Jums žinomus mokėtinus mokesčius, jeigu pastaroji 2012-04-01
      žurnalistui, už parašytą straipsnį sumokėjo sutartyje nurodytą
      2000 Lt atlygį.
    \end{condition}
    \begin{solution}
      Šiuo atveju autorinės sutarties taikyti negalima. Būtina
      taikyti darbo sutartį. Kadangi tiek 1200 Lt, tiek 2000 Lt
      yra ant popieriaus, tai iš viso žurnalistas tą mėnesį
      ant popieriaus gavo 3200 Lt.

      \pythonba{darbo}{3200 popierius}

    \end{solution}
  \end{task}

  \begin{task}
    \begin{condition}
      Apskaičiuokite autoriaus ir atlygį išmokančio asmens, UAB
      „X“, visus Jums žinomus mokėtinus mokesčius, jeigu
      pastaroji 2012-01-20 architektui, dirbančiam UAB „Y“, į jo
      banko sąskaitą pervedė 8200 Lt už pateiktą naujojo
      administracinio pastato projektą.
    \end{condition}
    \begin{solution}
      Kadangi jis nedirba įmonėje „X“, tai jis už autorinį turi mokėti
      mažesnius mokesčius.

      \pythonba{autorine|be|darbo}{8200 rankos}

    \end{solution}
  \end{task}

  \begin{task}
    \begin{condition}
      Apskaičiuokite autoriaus ir atlygį išmokančio asmens, UAB „X“,
      visus Jums žinomus mokėtinus mokesčius, jeigu pastaroji dizaineriui,
      kuris į kuriamą interjerą teikia tik UAB „X“ prekes, moka 2\%
      nuo prekių vertės, kuri už 2012-03-15 kurtą interjerą siekė
      350 000 Lt.
    \end{condition}
    \begin{solution}
      Laikome, kad tuos du procentus jis gauna ant popieriaus:
      $350 000 \cdot 2\% = 7000 Lt$. Kadangi dizaineris su UAB „X“
      gali būti sudaręs tiek autorinę, tiek darbo sutartį, tai
      paskaičiuojame abiem variantais.

      Jei autorinė:

      \pythonba{autorine|be|darbo}{7000 popierius}

      Jei darbo:

      \pythonba{darbo}{7000 popierius}

    \end{solution}
  \end{task}

  \begin{task}
    \begin{condition}
      Apskaičiuokite individualios įmonės „X“, PVM mokėtojos, kurioje dirba
      2 darbuotojai, praeitų metų pajamos siekė 320 000 Lt, kuri
      užsiima žurnalų leidyba, visus Jums žinomus mokėtinus mokesčius,
      jeigu per 2011 metų laikotarpį.
    \end{condition}

    \begin{subtask}
      \begin{condition}
        VU darbuotojui, su kuriuo nebuvo pasirašyta autorinė sutartis,
        už parašyta straipsnį į jo banko sąskaitą pervedė 600 Lt;
      \end{condition}
      \begin{solution}
        Galima nesudaryti autorinės sutarties, pakanka žodinio susitarimo,
        bet tada išlieka grėsmė, kad kas nors kitas panaudos tą
        straipsnį. Šiuo atveju reikia mokėti mokesčius pagal
        autorinę sutartį.

        \pythonba{autorine|be|darbo}{600 rankos}

      \end{solution}
    \end{subtask}
    \begin{subtask}
      \begin{condition}
        Redakcijos darbuotojui A, žurnalistui, kas mėnesį pervedė
        1500 Lt dydžio atlyginimą į jo banko sąskaitą.
      \end{condition}
      \begin{solution}
        Šiuo atveju turime normalią darbo sutartį.

        \pythonba{darbo}{1500 rankos}

      \end{solution}
    \end{subtask}
    \begin{subtask}
      \begin{condition}
        Redakcijos darbuotojui A, žurnalistui, pervedė autorinėje
        sutartyje nurodyta 800 Lt atlygį už užsienio leidinyje
        esančio straipsnio, kuris vėliau bus publikuotas IĮ „X“
        leidžiamajame žurnale, vertimą.
      \end{condition}
      \begin{solution}
        Kadangi dirba tą patį darbą, tai už autorinę sutartį vis tiek
        reikia mokėti tokius mokesčius, kokius mokėtų sudarius
        darbo sutartį. Kadangi \emph{pervedė}, tai tie 800 Lt yra
        į rankas.

        \pythonba{darbo}{800 rankos}

      \end{solution}
    \end{subtask}
    \begin{subtask}
      \begin{condition}
        Radakcijos darbuotojui B, žurnalistui, kas mėnesį pervedė
        2300 Lt darbo sutartyje nurodyto darbo užmokesčio už tiesiogines
        atliekamas pareigas ir 200 Lt autorinėje sutartyje
        nurodyto atlygio už žurnalo viršelio maketą;
      \end{condition}
      \begin{solution}
        Kadangi redaktoriui nepriklauso daryti maketus, tai galima
        daryti autorinę sutartį.

        Už tiesioginį darbą:

        \pythonba{darbo}{2300 rankos}
        
        Už maketavimą:

        \pythonba{autorine|su|darbu}{200 rankos}

      \end{solution}
    \end{subtask}
    \begin{subtask}
      \begin{condition}
        Gavo 180000 Lt pajamų iš žurnalo pardavimų, UAB „Y“
        sumokėjo 50000 Lt už žurnalo spaudą.
      \end{condition}
      \begin{solution}
        TODO: Parašyti programą.

        PVM = $(180000 - 50000)\cdot 0,21 = 27300$ Lt
      \end{solution}
    \end{subtask}
  \end{task}

\end{tasks}
