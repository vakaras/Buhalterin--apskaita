\chapter{Uždaviniai}

\begin{exmp}
  Apskaičiuokite autoriaus ir atlygį išmokančio asmens, UAB „X“,
  visus Jums žinomus mokėtinus mokesčius, jeigu pastaroji su
  darbo santykiais nesusijusiam asmeniui pažadėjo 350 Lt autorinio
  atlyginimo už žurnalo moterims viršelio maketą.

  Į rankas jis gauna 350 Lt į rankas.

  \begin{align*}
    \t{Suma ant popieriaus:} &\frac{350}{100\%-15\%-\frac{9\%}{2}}
    &= 434,78 Lt \\
    \t{Darbuotojas sumoka:} &438,78 - 350 &= 88,78 Lt \\
    \t{Iš jų gyventojų pajamų (GPM):} &438,78 \cdot 15\% &= 65,82 Lt \\
    \t{Iš jų … (PSDF):} &\frac{438,78}{2} \cdot 9\% &= 19,75 Lt \\
    \t{Darbdavys sumoka VSDF:} & \frac{438,78}{2} \cdot 29,7\%
      &= 65,16 Lt
    \t{Darbo vietos kaina:} & 434,78 + 65,16 = 499,94 Lt \\
  \end{align*}

\end{exmp}

\begin{exmp}
  Apskaičiuokite autoriaus ir atlygį išmokančio asmens, UAB „X“, visus
  Jums žinomus mokėtinus mokesčius, jeigu pastaroji savo darbuotojui,
  kuris pagal darbo sutartį yra techninis redaktorius, už
  parašytą straipsnį žurnalui į jo banko sąskaitą pervedė 500 Lt.

  \begin{align*}
    \t{Į rankas:} & 500Lt \\
    \t{Ant popieriaus:} & \frac{500}{0,76} &= 657,89 Lt \\
    \t{GPM:} & 98,68 Lt \\
    \t{PSDF:} & 59,21 Lt \\
    \t{VSDF:} & 203,81 Lt \\
    \t{Įmonės sąnaudos:} & 657,89 + 203,81 = 851,7 Lt \\
  \end{align*}
\end{exmp}

\begin{exmp}
  Apskaičiuokite autoriaus ir atlygį išmokančio asmens, UAB „X“, su
  kuria yra sudaryta darbo sutartis, pagal kurią darbuotojas rašo
  straipsnius žurnalui ir gauna 1200 Lt mėnesinio atlyginimo,
  visus Jums žinomus mokėtinus mokesčius, jeigu pastaroji 2012-04-01
  žurnalistui, už parašytą straipsnį sumokėjo sutartyje nurodytą
  2000 Lt atlygį.

  Šiuo atveju sudaryti autorinės sutarties negalima.
  GPM = 15\%
  PSDF = 9\%
  VSDF = 30,98\%
  NPD netaikom, nes gauta suma yra didesnė už 3100 Lt.

  Įmonė dar moka įmokas į garantinį fondą, kurios yra 0,2\% nuo
  visos sumos.

  \begin{align*}
    \t{Ant popieriaus:} & 3200 Lt \\
    \t{GPM:} & 480 Lt \\
    \t{PSD:} & 288 Lt \\
    \t{Į rankas:} & 2432 Lt \\
    \t{Įmonės išlaidos:} & 4191 Lt \\
    \t{VSDF:} & 991,36 Lt \\
    \t{Garantinis fondas (GF):} & 6,4 Lt \\
    \t{Įmonės sąnaudos:} & 4197,76 Lt \\
  \end{align*}
\end{exmp}

\begin{exmp}
  Apskaičiuokite autoriaus ir atlygį išmokančio asmens, UAB „X“,
  visus Jums žinomus mokėtinus mokesčius, jeigu pastaroji 2012-01-20
  architektui, dirbančiam UAB „Y“, į jo banko sąskaitą pervedė
  8200 Lt už pateiktą naujojo administracinio pastato projektą.

  Kadangi jis nedirba įmonėje „Y“, tai jis už autorinį turi mokėti
  mažesnius mokesčius.

  \begin{align*}
    \t{Į rankas:} & 8200 Lt \\
    \t{Ant popieriaus:} & \frac{8200}{80,5\%} &= 10186,34 Lt \\
    \t{GPM:} & 1527,95 Lt \\
    \t{PSDF:} & 458,39 Lt \\
    \t{VSDF:} & 1512,67 Lt \\
    \t{Įmonė sumoka:} & 11699,00 Lt \\
  \end{align*}

\end{exmp}

\begin{exmp}
  Apskaičiuokite autoriaus ir atlygį išmokančio asmens, UAB „X“,
  visus Jums žinomus mokėtinus mokesčius, jeigu pastaroji dizaineriui,
  kuris į kuriamą interjerą teikia tik UAB „X“ prekes, moka 2\%
  nuo prekių vertės, kuri 2012-03-15 kurtą interjerą siekė
  350 000 Lt.

  Pagal autorinį:
  \begin{align*}
    \t{Ant popieriaus:} & 7000 Lt \\
    \t{Į rankas:} & 5635 Lt \\
    \t{GPM:} & 1050 Lt \\
    \t{PSD:} & 315 Lt \\
    \t{VSDF:} & 1039,50 Lt \\
    \t{Darbo vietos kaina:} & 8039,50 Lt \\
  \end{align*}

  Pagal darbo sutartį:
  \begin{align*}
    \t{Ant popieriaus:} & 7000 Lt \\
    \t{Į rankas:} & 5320 Lt \\
    \t{GPM:} & 1050 Lt \\
    \t{PSD:} & 630 Lt \\
    \t{VSDF:} & 2168,60 Lt \\
    \t{Gf:} & 14 Lt \\
    \t{Darbo vietos kaina:} & 9182,60 Lt \\
  \end{align*}
\end{exmp}

\begin{exmp}
  Apskaičiuokit individualios įmonės „X“, PVM mokėtoja, kurioja dirba
  2 darbuotojai, praeitų metų pajamos siekė 320 000 Lt, kuri
  užsiima žurnalų leidyba, visus Jums žinomus mokėtinus mokesčius,
  jeigu per 2011 metų laikotarpį:
  \begin{itemize}
    \item VU darbuotojui, su kuriuo nebuvo pasirašyta autorinė sutartis,
      už parašyta straipsnį į jo banko sąskaitą pervedė 600 Lt;

      \begin{note}
        Galima nesudaryti autorinės sutarties, pakanka žodinio susitarimo,
        bet tada išlieka grėsmė, kad kas nors kitas panaudos tą
        straipsnį. Šiuo atveju reikia mokėti mokesčius pagal
        autorinę sutartį.
      \end{note}
      \begin{align*}
        \t{Į rankas:} & 600 Lt \\
        \t{GPM:} & 111,80 Lt \\
        \t{PSD:} & 33,54 Lt \\
        \t{Ant popieriaus:} & 745,34 Lt \\
        \t{VSD:} & 110,68 Lt \\
        \t{Įmonės sąnaudos:} & 856,02 \\
      \end{align*}

    \item redakcijos darbuotojui A, žurnalistui, kas mėnesį pervedė
      1500 Lt dydžio atlyginimą į jo banko sąskaitą;o

      Šiuo atveju turime darbo sutartį.
      \begin{align*}
        \t{Ant popieriaus:} & \frac{1500 - 94,5}{0,73} &= 1925,34 LT \\
        \t{NPD:} & 244,93 Lt \\
        \t{GPM:} & 252,06 Lt \\
        \t{PSD:} & 173,28 Lt \\
        \t{VSD:} & 596,47 Lt \\
        \t{GF:} & 3,85 Lt \\
        \t{Įmonės sąnaudos:} & 2525,66 Lt \\
      \end{align*}
    \item redakcijos darbuotojui A, žurnalistui, pervedė autorinėje
      sutartyje nurodyta 800 Lt atlygį už užsienio leidinyje
      esančio straipsnio, kuris vėliau bus publikuotas IĮ „X“
      leidžiamajame žurnale, vertimą;

      \begin{note}
        Kadangi dirba tą patį darbą, tai už autorinę sutartį vis tiek
        reikia mokėti tokius mokesčius, kokius mokėtų sudarius
        darbo sutartį.

        Kadangi \emph{autorinėje sutartyje} nurodytą (beje, paminėta,
        kad \emph{pervedė}), tai 800 Lt yra „į rankas“.
      \end{note}

      \begin{align*}
        \t{Į rankas:} & 800 Lt \\
        \t{Ant popieriaus:} & 966,44 Lt \\
        \t{NPD} &= 470-0,2\cdot(966,44 - 800) &= 436,71 Lt \\
        \t{GPM:} & 79,46 Lt \\
        \t{PSD:} & 86,98 Lt \\
        \t{VSDF:} & 299,40 Lt \\
        \t{GF:} & 1,93 Lt \\
        \t{Įmonės sąnaudos:} & 1267,77 Lt \\
      \end{align*}
    \item radakcijos darbuotojui B, žurnalistui, kas mėnesį pervedė
      2300 Lt darbo sutartyje nurodyto darbo užmokesčio už tiesiogines
      atliekamas pareigas ir 200 Lt autorinėje sutartyje
      nurodyto atlygio už žurnalo viršelio maketą;

      \begin{note}
        Kadangi redaktoriui nepriklauso daryti maketus, tai galima
        daryti autorinę sutartį.
      \end{note}

      \begin{align*}
        \t{Į rankas:} & 2300 Lt \\
        \t{Ant popieriaus:} & 3021,23 Lt \\
        \t{NPD:} & 25,75 Lt \\
        \t{GPM:} & 449,32 Lt \\
        \t{PSD:} & 271,91 Lt \\
        \t{VSD:} & 935,98 Lt \\
        \t{GF:} & 6,04 Lt \\
        \t{Darbo vietos kaina:} & 3963,25 Lt \\
        \t{Į rankas:} & 200 Lt \\
        \t{Ant popieriaus:} & 263,16 Lt \\
        \t{GPM:} & 39,47 Lt \\
        \t{PSD:} & 23,68 Lt \\
        \t{VSD:} & 81,53 Lt \\
        \t{Įmonės sąnaudos:} & 344,68 Lt \\
      \end{align*}

    \item gavo 180000 Lt pajamų iš žurnalo pardavimų, UAB „Y“ sumokėjo
      50000 Lt už žurnalo spaudą.

      PVM = $(180000 - 50000)\cdot 0,21 = 27300$ Lt

  \end{itemize}
\end{exmp}
