\chapter{Lentelės}

\section{Autorinės su darbo sutartimi}

Tarkime, kad autorius iš savo darbdavio ant popieriaus gavo $P$ (arba
į rankas $G$). Tada teisingi tokie sąryšiai:

\begin{tabularx}{15cm}{p{7.5cm}|p{7.5cm}}
  Ant popieriaus: & $P = \frac{G}{0,76}$ \\
  Į rankas: & $G = 0,76 \cdot P$ \\
  GPM: & $P \cdot 15\%$ \\
  PSDF: & $P \cdot 9\%$ \\
  Darbuotojas sumoka: & $P \cdot 24\%$ \\
  VSDF: & $P \cdot 30,98\%$ \\
  Darbo vietos kaina: & $P \cdot 130,98\%$ \\
\end{tabularx}

\section{Autorinės be darbo sutarties}

Tarkime, kad autorius iš su juo darbo santykiais nesusijusio
asmens ant popieriaus gavo $P$ (arba į rankas $G$). Tada teisingi tokie
sąryšiai:

\begin{tabularx}{15cm}{p{7.5cm}|p{7.5cm}}
  Ant popieriaus: & $P = \frac{G}{0,805}$ \\
  Į rankas: & $G = 0,805 \cdot P$ \\
  GPM: & $P \cdot 15\%$ \\
  PSDF: & $\frac{P}{2} \cdot 9\%$ \\
  Darbuotojas sumoka: & $P \cdot 19,5\%$ \\
  VSDF: & $\frac{P}{2} \cdot 29,7\%$ \\
  Darbo vietos kaina: & $P \cdot 129,7\%$ \\
\end{tabularx}

\section{Darbo sutartis}

Tarkime, kad darbuotojas iš darbdavio ant popieriaus gavo $P$. Tada
teisingi tokie sąryšiai:

\begin{tabularx}{15cm}{p{7.5cm}|p{7.5cm}}
  NPD: & $470 Lt$, jei $P \leq 800$ \\
  NPD: & $0 Lt$, jei $P \geq 3150$ \\
  NPD: & $470 - 0,2 \cdot (P - 800)$, kitu atveju \\
  GPM: & $(P - NPD) \cdot 15\%$ \\
  PSDF: & $P \cdot 9\%$ \\
  VSDF: & $P \cdot 30,98\%$ \\
  GF: & $P \cdot 0,2\%$ \\
\end{tabularx}

\section{Verslo liudijimas}

Jei $P$ yra gautos pajamos, o $S$ yra sąnaudos, o $L$ leidžiami
atskaitymai, tai teisingi tokie sąryšiai:

\begin{tabularx}{15cm}{p{7.5cm}|p{7.5cm}}
  GPM: & $120 Lt$ (kiekvieną savivaldybė gali nustatyti pati, bet ne
    mažesnį nei $120 Lt$) \\
  PVM: & $(P - 100 000 Lt) \cdot 21\%$, jei $P \geq 100 000 Lt$ \\
  PSDF(mėnesio): & $MMA \cdot 9\% = 72 Lt$ \\
  PSDF(metų): & $(P - S) \cdot 9\%$ \\
  VSDF: & $\frac{BPD}{2} = 180 Lt$ \\
\end{tabularx}

\begin{itemize}
  \item BPD – bazinės pensijos dydis ($360 Lt$).
  \item MMA – minimalus mėnesinis atlyginimas ($800Lt$).
\end{itemize}

\section{Individuali veikla pagal pažymą}

Jei $P$ yra gautos pajamos, o $S$ yra sąnaudos, o $L$ leidžiami
atskaitymai, o $AP$ yra leidžiami atskaitymai tai teisingi tokie
sąryšiai.

GPM galime skaičiuoti dviem būdais:
\begin{enumerate}
  \item pasinaudodami GPM „baze“:
    \begin{tabularx}{15cm}{p{7.5cm}|p{7.5cm}}
      AP: & $P - S$ \\
      VSDF: & $\frac{AP}{2} \cdot 28.5\%$ \\
      PSDF(mėnesio): & $MMA \cdot 9\% = 72 Lt$ \\
      PSDF(metų): & $AP \cdot 9\%$ \\
      PVM: & $AP \cdot 21\%$ \\
      GPM „bazė“: & $AP - VSDF - PSDF$ \\
      MNPD: $5630 - 0,2 \cdot (GPM \t{„bazė“} - 9600)$ \\
      GPM: $(GPM \t{„bazė“} - MNPD) \cdot 5\%$ \\
    \end{tabularx}
  \item galime taip pat tarti, kad sąnaudos buvo 30\%, tada:
    \begin{tabularx}{15cm}{p{7.5cm}|p{7.5cm}}
      AP: & $P \cdot 70\%$ \\
      GPM: & $AP \cdot 5\%$ \\
      VSDF: & $\frac{AP}{2} \cdot 28.5\%$ \\
      PSDF(mėnesio): & $MMA \cdot 9\% = 72 Lt$ \\
      PSDF(metų): & $AP \cdot 9\%$ \\
      PVM: & $AP \cdot 21\%$ \\
    \end{tabularx}
\end{enumerate}

\section{Susijusių atlikėjų ir sportininkų pajamos}

TODO: Ką tiksliai reiškia „susijusių“?

$A$ – apmokestinamosios pajamos, $P$ – atlygis.

\begin{tabularx}{15cm}{p{7.5cm}|p{7.5cm}}
  GPM: & $A \cdot 5\%$ \\
  PSDF: & $P \cdot 9\%$ \\
  VSDF: & $P \cdot 30,98\%$ \\
\end{tabularx}

\section{Nesusijusių atlikėjų ir sportininkų pajamos}

TODO: Ką tiksliai reiškia „nesusijusių“?

$A$ – apmokestinamosios pajamos, $P$ – atlygis.

\begin{tabularx}{15cm}{p{7.5cm}|p{7.5cm}}
  GPM: & $A \cdot 5\%$ \\
  PSDF: & $\frac{P}{2} \cdot 9\%$ \\
  VSDF: & $\frac{P}{2} \cdot 28,5\%$ \\
\end{tabularx}

\section{Advokatų, notarų ir antstolių pajamos}

$A$ – apmokestinamosios pajamos, $P$ – atlygis.

\begin{tabularx}{15cm}{p{7.5cm}|p{7.5cm}}
  GPM: & $A \cdot 15\%$ \\
  PSDF(metinis): & $\frac{A}{2} \cdot 9\%$ \\
  PSDF(mėnesinis): & $MMA \cdot 9\% = 72 Lt$ \\
  VSDF(metinis): & $\frac{A}{2} \cdot 28,5\%$ \\
  VSDF(mėnesinis): & avansu nuo pasirinktos sumos \\
\end{tabularx}
