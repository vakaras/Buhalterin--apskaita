\chapter{Lentelės}

\section{Autorinės su darbo sutartimi}

Tarkime, kad autorius iš savo darbdavio ant popieriaus gavo $P$ (arba
į rankas $G$). Tada teisingi tokie sąryšiai:

\begin{tabularx}{15cm}{p{7.5cm}|p{7.5cm}}
  Ant popieriaus: & $P = \frac{G}{0,76}$ \\
  Į rankas: & $G = 0,76 \cdot G$ \\
  GPM: & $P \cdot 15\%$ \\
  PSDF: & $P \cdot 9\%$ \\
  Darbuotojas sumoka: & $P \cdot 24\%$ \\
  VSDF: & $P \cdot 30,98\%$ \\
  Darbo vietos kaina: & $P \cdot 130,98\%$ \\
\end{tabularx}

\section{Autorinės be darbo sutarties}

Tarkime, kad autorius iš su juo darbo santykiais nesusijusio
asmens ant popieriaus gavo $P$ (arba į rankas $G$). Tada teisingi tokie
sąryšiai:

\begin{tabularx}{15cm}{p{7.5cm}|p{7.5cm}}
  Ant popieriaus: & $P = \frac{G}{0,805}$ \\
  Į rankas: & $G = 0,805 \cdot G$ \\
  GPM: & $P \cdot 15\%$ \\
  PSDF: & $\frac{P}{2} \cdot 9\%$ \\
  Darbuotojas sumoka: & $P \cdot 19,5\%$ \\
  VSDF: & $\frac{P}{2} \cdot 29,7\%$ \\
  Darbo vietos kaina: & $P \cdot 129,7\%$ \\
\end{tabularx}

\section{Darbo sutartis}

Tarkime, kad darbuotojas iš darbdavio ant popieriaus gavo $P$. Tada
teisingi tokie sąryšiai:

\begin{tabularx}{15cm}{p{7.5cm}|p{7.5cm}}
  NPD: & $470 Lt$, jei $P \leq 800$ \\
  NPD: & $0 Lt$, jei $P \geq 3150$ \\
  NPD: & $470 - 0,2 \cdot (P - 800)$, kitu atveju \\
  GPM: & $(P - NPD) \cdot 15\%$ \\
  PSDF: & $P \cdot 9\%$ \\
  VSDF: & $P \cdot 30,98\%$ \\
  GF: & $P \cdot 0,2\%$ \\
\end{tabularx}

\section{Verslo liudijimas}

\begin{tabularx}{15cm}{p{7.5cm}|p{7.5cm}}
  GPM: & $120 Lt$ (kiekvieną savivaldybė gali nustatyti pati, bet ne
    mažesnį nei $120 Lt$) \\
  PVM: & $(P - 100 000 Lt) \cdot 21\%$, jei $P \geq 100 000 Lt$ \\
  PSDF(mėnesio): & $MMA \cdot 9\% = 72 Lt$ \\
  PSDF(metų): & $(P - S) \cdot 9\%$ \\
  VSDF: & $\frac{BPD}{2} = 360 Lt$ \\
\end{tabularx}

\begin{itemize}
  \item BPD – bazinės pensijos dydis ($360 Lt$).
  \item MMA – minimalus mėnesinis atlyginimas ($800Lt$).
\end{itemize}

\section{Individuali veikla pagal pažymą}

Jei $P$ yra gautos pajamos, o $S$ yra sąnaudos, o $L$ leidžiami
atskaitymai, tai teisingi tokie sąryšiai:

\begin{tabularx}{15cm}{p{7.5cm}|p{7.5cm}}
%  GPM: & $(P - L) \cdot 15\%$, jei $P \in$ laisvųjų profesijų veikla arba
%    vertybinių popierių \\
%  GPM: & $(P - L) \cdot 5\%$, kitu atveju \\
%  PVM: & $(P - 100 000 Lt) \cdot 21\%$, jei $P \geq 100 000 Lt$ \\
 
  AP: P - S
  VSDF: 28.5\% $\cdot$ AP
  PVM: & (P - S) $\cdot$ 21\% \\
  
  \begin{itemize}
  	\item Jei Skaičiuojam: Pajamos - Sąnaudos
	\begin{itemize}
		\item GPM „bazė“: AP - VSDF - PSDF
  		\item MNPD: 5640 - 0.2 $\cdot$ („bazė“ - 9600)
  		\item GPM: („bazė“ - MNPD) $\cdot$ 5\%
	\end{itemize}	  	
  	
  	\item Jei Skaičiuojam: Kad sąnaudos buvo 30\%
  	\begin{itemize}
  		\item AP: P $\cdot$ 70\%
  		\item GPM: AP $\cdot$ 5\%
  	\end{itemize}
  \end{itemize}

  
  
%  PSDF(mėnesio): & $MMA \cdot 9\% = 72 Lt$ \\
%  PSDF(metų): & $(P - S) \cdot 9\%$ \\
%  VSDF: & $\frac{P-S}{2} \cdot 28,5$ \\
\end{tabularx}
