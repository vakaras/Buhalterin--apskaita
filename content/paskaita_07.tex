\chapter{Apskaitinė informacija}

\setslideprefix{10 - apskaita imonese/sumazintos/sumazintos\_}

\section{Apskaitinė informacija}

\slide{01}

\section{Įmonės vaidmuo apskaitoje}

\slide{02}

\section{Apskaitos samprata}

\slide{03}

\section{Apskaitos rūšys}

\slide{04}
\slide{05}

\begin{itemize}
  \item Finansinė.
  \item Valdymo – skirta įmonės vadovybei, siekiančiai priimti teisingus
    valdymo sprendimus. (Įmonė naudoja savo reikmėms, jos nereikia
    pateikti išorei, jos nereglamentuoja įstatymai.)
  \item Mokestinė.
  \item Statistinė. Renka statistikos departamentas.
\end{itemize}

\subsection{Finansinė apskaita}

\slide{06}

\subsection{Valdymo apskaita}

\slide{06}

\subsection{Statistinė apskaita}

\slide{07}

\slide{08}

\section{Apskaitos informacijos vartotojai}

\slide{09}

Apskaita yra reikalinga:
\begin{itemize}
  \item vidiniams „vartotojams“ – vadovybei, akcininkams;
  \item išoriniams „vartotojams“ – esantys ir potencialūs kreditoriai,
    tiekėjai, pirkėjai;
  \item valstybinės institucijos;
  \item kiti: konkurentai, auditoriai.
\end{itemize}

\section{Socialinė apskaita}

\slide{10}

\subsection{Socialinės apskaitos nauda įmonei}

\slide{11}

\subsection{Socialinės apskaitos požymiai įmonėje}

\slide{12}

\section{Apskaitos sistemos tikslai}

\slide{13}

\section{Finansinės apskaitos reglamentavimas}

\slide{14}

\subsection{Buhalterinės apskaitos įstatymas}

\slide{15}

\slide{16}

\slide{17}

\slide{18}

\slide{19}

\slide{20}

\slide{21}

\slide{22}

\subsection{Tarptautiniai verslo standartai}

\slide{23}

\slide{24}

\section{Fundamentinė apskaitos lygybė}

\slide{25}

Apskaitos lygybė nusako, kad visas įmonei priklausantis turtas turi
būti lygus bendrai finansinių interesų sumai.

\begin{align*}
  \t{turtas} &= \t{nuosavybė}
  \t{nuosavybė} &= \t{akcijos} + \t{skolos} + \cdots
\end{align*}

\section{Bendrieji apskaitos principai (BAP)}

\slide{26}

Nepriklausomai nuo verslo organizavimo formų įmonės veiklos apskaita
iš esmės yra tvarkoma vienodai.

\slide{27}

\subsection{BAP vieta įmonių apskaitos politikoje}

\slide{28}

Piramidė:
\begin{itemize}
  \item įmonių apskaitos politika;
  \item apskaitos tvarkymą reglamentuojantys įstatymai ir kiti norminiai
    aktai;
  \item bendrieji apskaitos principai.
\end{itemize}

\subsection{Apskaitos politikos samprata}

\slide{29}

\subsection{Įmonės principas}

\slide{30}

Įmonės principas taikomas visoms verslo organizavimo formoms ir
pasireiškia tuo, kad kiekvienos įmonės turto ir nuosavybės bei
įsipareigojimų apskaita atskirta nuo savininkų turto ir skolinimų
apskaitos.

\subsection{Veiklos tęstinumo principas}

\slide{31}

Šis principas teigia, kad įmonė, kaip apskaitos objektas, artimiausiu
metu nebus likviduojama ir jos egzistavimo laikotarpis nėra kaip
nors apribotas. Toks teiginys atitinka visų įmonės veikla suinteresuotų
asmenų interesus.

\subsection{Periodiškumo principas}

\slide{32}

Periodiškumo principas numato, kad nepaisant įmonės būklės, jos
ūkinė veikla gali būti dirbtinai suskirstyta į tam tikrus laikotarpius.
Taikant šį principą, galima reguliariai gauti apskaitinę informaciją,
susijusią su konkrečiais ataskaitiniais laikotarpiais.

\slide{33}

Kiekviena įmonė, atsižvelgdama į veiklos intensyvumą, gali pati
pasirinkti finansinių metų pradžią ir pabaigą. Dažniausiai metų
laikotarpis yra dalijamas ketvirčiais arba mėnesiais, norint
gauti kuo daugiau informacijos apie situacijos kitimą įmonėje.
Be to, tai susiję ir su mokesčių, mokamų valstybei, tiksliu
apskaičiavimu.

\subsection{Pastovumo principas}

\subsection{Piniginio mato principas}

\slide{34}

Taikant šį principą apskaitoje, visi atsargų, turto bei veiklos pokyčiai
yra matuojami pinigine išraiška.

Tai, kad finansinės apskaitos objektai turi būti išmatuoti piniginiu
matu, dar nereiškia, kad metinėje atskaitomybėje apskritai negalima
vartoti natūrinių matų. Galima ir reikia, jeigu tai padeda išreikšti
įmonės ataskaitinius rodiklius.

\slide{35}

Trūkumai:
\begin{itemize}
  \item matuojant ir pateikiant visą įmonės ūkinę veiklą finansinėse
    ataskaitose tik pinigine išraiška, galima prarasti pakankamai
    svarbią, pinigais neišreiškiamą informaciją;
  \item pasaulyje nėra nei vienos stabilios valiutos, kadangi keičiasi
    jų tarpusavio santykis, taip pat kinta pinigų perkamoji galia
    šalies viduje.
\end{itemize}

\subsection{Atsargumo principas}

\slide{36}

Šis principas reikalauja, kad apskaitininkai labai atsargiai vertintų
visus įmonės veiklos rezultatus. Tai itin svarbu akcinėse bendrovėse,
kai yra ne vienas savininkas, o administracija, siekdama palankių
bendrovės vadovų perrinkimo rezultatų, dažnai yra linkusi „pagrąžinti“
tikrąją padėti.

\subsection{Kaupimo principas}

\slide{37}

Šis principas yra vienas iš svarbiausių laisvosios rinkos apskaitos
nuostatų. Jis reikalauja, kad ūkiniai faktai apskaitoje būtų fiksuojami
jau įvykę: uždirbamos pajamos turi būti registruojamos tada, kai 
jos uždirbamos, o jas uždirbant patirtos sąnaudos – kai jos patiriamos,
nepriklausomai nuo pinigų gavimo ir išmokėjimo.

\subsection{Palyginimo principas}

\slide{38}

Principas numato laikotarpį, per kurį turi būti pripažįstamos sąnaudos.
Principo esmė ta, kad apskaitoje yra pripažįstamos ir palyginamos
tik to laikotarpio pajamos ir sąnaudos. Išlaidos, kurios susijusios
su kitų, būsimų laikotarpių pajamomis, laikomos turto kūrimo išlaidomis.

\subsection{Neutralumo principas}

\slide{39}

Reiškia tai, kad finansinėje atskaitomybėje pateikta informacija turi
būti objektyvi ir nešališka. Jos pateikimas neturi priklausyti
nuo siekio priversti apskaitos informacijos vartotojus priimti
įmonei palankius sprendimus.

\section{Apskaitos informacijos vartotojai}

\slide{40}

\slide{41}

\slide{42}

\slide{43}

\slide{44}

\slide{45}

\slide{46}

\subsection{Turinio svarbos principas}

\section{Apskaitos ciklas}

\slide{47}

Prasideda sausio 1 dieną, baigiasi metų pabaigoje. Kartais gali būti
ir kitoks.

Prasideda finansinių metų sąskaitų atidarymu, o baigiasi
finansinių metų sąskaitų uždarymu.

\section{Ūkinis faktas}

\slide{48}

Kiekvienos įmonės veiklą sudaro daugybė \strong{ūkinių faktų}, darančių
tam tikrą poveikį įmonės turtui, nuosavam kapitalui, įsipareigojimams
ar veiklos rezultatams.

Kiekvienas \strong{ūkinis faktas}, susijęs su turto ar įsipareigojimų
pasikeitimu turi būti tiksliai įkainotas, nustatyta, ar tai
– \strong{ūkinė operacija} ar \strong{ūkinis įvykis} ir įrašytas
į tinkamai parinktą sąskaitą.

\subsection{Ūkinė operacija}

\slide{49}

Tikslinga veikla, siekiant ūkinio rezultato.

Ji fiksuojama tada, kai faktiškai pakeičia turto ir nuosavybės struktūrą
arba apimtį.

\section{Apskaitos lygybė}

\slide{50}

\begin{equation*}
  \t{Turtas} = \t{Įsipareigojimai} + \t{Nuosavas kapitalas},
\end{equation*}
čia:
\begin{description}
  \item[turtas] – ištekliais, kuriuos naudodama įmonė tikisi uždirbti
    naudos ateityje;
  \item[įsipareigojimai] – turto dalis, į kurią turi teisių kreditoriai;
  \item[nuosavas kapitalas] – turto dalis, likusi įvykdžius visus
    įsipareigojimus ir į kurią turi teises savininkai (kitaip dar
    vadinama \emph{grynasis turtas}).
\end{description}

\section{Dvejybinio įrašo taisyklė}

\slide{51}

Ūkinio fakto atspindėjimas vienos sąskaitos debete, o kitos – kredite
vadinamas dvejybiniu įrašu.

Sąskaitų derinys, kai viena sąskaita debetuojama, o kita kredituojama,
vadinamas sąskaitų korespondencija.

\section{Ūkinių operacijų registravimas apskaitoje}

\slide{52}

\begin{itemize}
  \item Turto sąskaitos yra 1 ir 2 klasės.
  \item Nuosavo kapitalo ir įsipareigojimų sąskaitos yra 3 ir 4 klasės.
  \item Pajamų sąskaitos yra 5 klasė.
  \item Sąnaudų sąskaitos yra 6 klasė.
\end{itemize}

Pelno (nuostolių) ataskaita yra sudaroma iš 5 ir 6 klasių. Joje yra
nurodomas finansinis rezultatas (pelnas arba nuostoliai).

Balansas yra sudaromas iš 1-4 klasių bei finansinio rezultato.
