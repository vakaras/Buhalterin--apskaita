\chapter{Apskaitinė informacija}

\begin{note}
  Balandžio 23 dieną nebus paskaitos.
  Birželio 4 dieną 101 auditorijoje bus egzaminas.
\end{note}

Buhalterinė apskaita buvo kuriame tikslu sukontroliuoti?

\section{Apskaitos samprata}

\section{Apskaitos …}

\section{Apskaitos rūšys}

\begin{itemize}
  \item Finansinė.
  \item Valdymo – skirta įmonės vadovybei, siekiančiai priimti teisingus
    valdymo sprendimus. (Įmonė naudoja savo reikmėms, jos nereikia
    pateikti išorei, jos nereglamentuoja įstatymai.)
  \item Mokestinė.
  \item Statistinė. Renka statistikos departamentas.
\end{itemize}

\subsection{Statistinė apskaita}

\section{Apskaitos informacijos vartotojai}

Apskaita yra reikalinga:
\begin{itemize}
  \item vidiniams „vartotojams“ – vadovybei, akcininkams;
  \item išoriniams „vartotojams“ – esantys ir potencialūs kreditoriai,
    tiekėjai, pirkėjai;
  \item valstybinės institucijos;
  \item kiti: konkurentai, auditoriai.
\end{itemize}

\section{Socialinė apskaita}

\subsection{Socialinės apskaitos nauda įmonei}

\subsection{Socialinės apskaitos požymiai įmonėje}

\section{Apskaitos sistemos tikslai}

\section{Finansinės apskaitos reglamentavimas}

\subsection{Buhalterinės apskaitos įstatymas}

\subsection{Tarptautiniai verslo standartai}

\section{Fundamentinė apskaitos lygybė}

\begin{align*}
  \t{turtas} &= \t{nuosavybė}
  \t{nuosavybė} &= \t{akcijos} + \t{skolos} + \cdots
\end{align*}

\section{Bendrieji apskaitos principai (BAP)}

\subsection{BAP vieta įmonių apskaitos politikoje}

Piramidė:
\begin{itemize}
  \item įmonių apskaitos politika;
  \item apskaitos tvarkymą reglamentuojantys įstatymai ir kiti norminiai
    aktai;
  \item bendrieji apskaitos principai.
\end{itemize}

\subsection{Įmonės principas}

\subsection{Veiklos tęstinumo principas}

\subsection{Periodiškumo principas}

\subsection{Pastovumo principas}

\subsection{Piniginio mato principas}

\subsection{Atsargumo principas}

\subsection{Kaupimo principas}

\subsection{Turinio svarbos principas}

\section{Apskaitos ciklas}

Prasideda sausio 1 dieną, baigiasi metų pabaigoje. Kartais gali būti
ir kitoks.

\section{Apskaitos lygybė}

Turtas = Įsipareigojimai + …

\section{Dvejybinio įrašo taisyklė}

\section{Ūkinių operacijų registravimas apskaitoje}
