\section{Finansinė atskaitomybė AB ir UAB}

Pagal Akcinių bendrovių įstatymą, metinė finansinė atskaitomybė AB
ir UAB turi patvirtinta…

\section{Pilna finansinė atskaitomybė}

Finansinės atskaitomybės rūšys: Pilna ir trumpa.

\subsection{Balansas}

\subsection{Pelno (nuostolių) ataskaita}

\subsection{Pinigų srautų ataskaita}

\subsection{Nuosavo…}

\subsection{Aiškinamasis raštas}

\section{Trumpa finansinė atskaitomybė}

Nėra pinigų srautų ataskaitos.

\subsection{Rengiama tada}

\begin{itemize}
  \item grynosios pardavimo pajamos nesiekia 10 mln. Lt per metus.
\end{itemize}

\section{Finansinė atskaitomybė}

\section{Įmonių atskaitomybės klasifikavimas}

\subsection{Mokestinė atskaitomybė} 

\subsection{Statistinė atskaitomybė}

\subsection{Atskaitomybė pagal …}

\section{Atskaitomybės principai}

\subsection{Suprantamumo principas}
\subsection{Tinkamumo principas}
\subsection{Patikimumo principas}

\section{Balanso samprata}
\section{Įprastinė įmonės veikla}

\section{Pagautė (ypatingas pelnas)}
Rodo netikėtą pelną dėl įmonės valdytojų…

\section{Netekimai…}

\section{Grynojo pelno apskaičiavimas}

\chapter{Uždaviniai}

1 a

2 a

3 d

4. Ekonominiai ištekliai yra:
c

5. Kuri iš pateiktų lygybių yra teisinga:
c

6. Įmonėje esančios perparduoti skirtos prekės priskirtinos: b
įmonės turtui.

7. Metinę finansinę atskaitomybę sudaro: d finansinės ataskaitos
ir paaiškinamasis raštas.

8. Trumpalaikis turtas, tai turtas, kuris: d įmonėje naudojamas
ne ilgiau kaip vienerius metus arba kurio vertė mažesnė už
įmonės nustatytą ir paaiškinamajame rašte atskleistą minimalią
ilgalaikio turto vertę.

9. Įmonė apyvartinį kapitalą apskaičiuoja: a iš pastoviojo kapitalo
atimdama įmonės ilgalaikio turto sumą.

10. Išrinkite teisingą teiginį: b suma, kuria uždirbtos…

\chapter{Uždaviniai 2}

1. Pagal įmonės principą:
b Kiekviena įmonė, kuri sudaro  finansinę atskaitomybę…

2. Pagal veiklos tęstinumo principą:
d Tvarkant apskaitą daroma prielaida, kad įmonės veiklos laikotarpis
neribotas ir įmonės nenumatoma likviduoti. …

3. Pagal periodiškumo principą:
a įmonės veikla tvarkant apskaitą suskirstoma į finansinius metus
arba kitos trukmės ataskaitinius laikotarpius, kuriems pasibaigus
sudaroma finansinė atskaitomybė.

4. Pagal pastovumo principą:
a Įmonės pasirinktą apskaitos metodą turi taikyti…

5. Pagal piniginio mato principą:
c Visas įmonės turtas, nuosavas kapitalas ir įsipareigojimai
finansinėje atskaitomybėje išreiškiami pinigais.

6. Pagal kaupimo principą:
b Ūkinės operacijos … neatsižvelgiant …

7. Pagal palyginimo principą:
b Pajamos, uždirbtos per ataskaitinį laikotarpį, siejamos su
sąnaudomis, patirtomis uždirbant tas pajamas.

8. Pagal atsargumo principą:
d Įmonė pasirenka tokius apskaitos metodus…

9. Pagal neutralumo principą:
a Apskaitos informacija pateikiama nešališkai. …

10. Pagal turinio svarbos principą:
c Ūkinės operacijos ir ūkiniai įvykiai į apskaitą traukiami pagal jų
turinį ir ekonominę prasmę, o ne tik pagal jų juridinę formą.

\chapter{Uždaviniai 3}

\section{1 užduotis}

\begin{enumerate}
  \item skolos tiekėjamas (nuosavybė);
  \item skolos rangovams-tiekėjamas (nuosavybė);
  \item gautinos sumos (turtas);
  \item nepaskirstytasis ataskaitinių metų pelnas (nuosavybė);
  \item iš anksto apmokėta nuoma (turtas);
  \item pinigai (turtas);
  \item įrenginiai (turtas);
  \item akcinis kapitalas (nuosavybė);
  \item medžiagos (turtas).
\end{enumerate}

\section{2 užduotis}

1.8 = (12000 + 2000 + 500 + 300 + 1000) - (2500 + 1500) = 11800
2.3 = 1300
…
2.9 = 2.10 = 1000

\section{3 užduotis}

1.5 = (40000 + 20100 + 5000 + 2300 - 1600) - (10100 + 20500 + 21000 + 11300) = 
3.9 = 3.10 = (17000 + 31440 + 23560 + 7540 + 35460) - (… 12320  - 7680) = 2 \cdot 32680

\section{4 užduotis}

Turtas:
Atsargos
Pinigai
Įrengimai
Patentai
Ateinančių laikotarpių sąnaudos
Perparduoti skirtos prekės
Pirkėjų skola
Medžiagos

Nuosavybė:
Trumpalaikė banko paskola
Nepaskirstytasis pelnas
Ataskaitinių metų pelnas
Akcinis kapitalas
Ilgalaikė paskola pastatui statyti
