\chapter{Atsakomybės}

\setslideprefix{11 - Atsakomybes + Pratybos/fotkinta/P4301}

\section{Finansinės atskaitomyubės teisinė bazė}

\slide{887}

\section{Finansinių ataskaitų rinkinys}

\slide{888}

\section{Finansinė atskaitomybė AB ir UAB}

\slide{889}

Pagal Akcinių bendrovių įstatymą, metinė finansinė atskaitomybė AB
ir UAB turi patvirtinta…

\section{Pilna finansinė atskaitomybė}

\slide{890}

Finansinės atskaitomybės rūšys: Pilna ir trumpa.

Pilną finansinę atskaitomybę sudaro:
\begin{itemize}
  \item pilnas balansas;
  \item pilna pelno (nuostolių) ataskaita;
  \item nuosavojo kapitalo pokyčių ataskaita;
  \item pinigų srautų ataskaita;
  \item pilnas aiškinamasis raštas.
\end{itemize}

\subsection{Balansas}

\slide{891}

Balansas yra finansinė ataskaita, kurioje parodomas visas įmonės turtas,
nuosavas kapitalas ir įsipareigojimai paskutinę ataskaitinio laikotarpio
dieną.

\subsection{Pelno (nuostolių) ataskaita}

\slide{892}

Pelno (nuostolių) ataskaita yra finansinė ataskaita, kurioje parodomos
visos įmonės per ataskaitinį laikotarpį uždirbtos pajamos, patirtos
sąnaudos ir gauti rezultatai.

\subsection{Pinigų srautų ataskaita}

\slide{893}

Pinigų srautų ataskaita yra finansinė ataskaita, kurioje pateikiama
informacija apie įmonės pinigų ir pinigų ekvivalentų pasikeitimus
per ataskaitinį laikotarpį.

\subsection{Nuosavo kapitalo pokyčių ataskaita}

\slide{894}

Nuosavo kapitalo pokyčių ataskaita yra finansinė ataskaita, kurioje
parodomas įmonės nuosavo kapitalo pokytis per ataskaitinius ir
praėjusius finansinius metus.

\subsection{Aiškinamasis raštas}

\slide{895}

Aiškinamasis raštas yra sudedamoji finansinių ataskaitų rinkinio dalis,
kurioje pateikiama papildoma balanse, pelno (nuostolių), pinigų srautų
ir nuosavo kapitalo pokyčių ataskaitose neatskleista ar nepakankamai
detaliai atskleista reikšminga informacija.

\section{Trumpa finansinė atskaitomybė}

\slide{896}

Iš principo trumpa nuo pilnos skiriasi tuo, kad trumpoje nėra pinigų
srautų ataskaitos.

Trumpą finansinę atskaitomybę sudaro:
\begin{itemize}
  \item trumpas balansas;
  \item trumpa pelno (nuostolių) ataskaita;
  \item nuosavojo kapitalo pokyčių ataskaita;
  \item trumpas aiškinamasis raštas.
\end{itemize}

\subsection{Rengiama tada}

\slide{897}

…
\begin{itemize}
  \item grynosios pardavimo pajamos nesiekia 10 mln. Lt per metus.
  \item balanse nurodyto turto vertė nesiekia 6 mln. Lt;
  \item vidutinis metinis darbuotojų skaičius pagal sąrašą
    neviršija 15.
\end{itemize}

\section{Finansinė atskaitomybė}

\slide{899}

\section{Įmonių atskaitomybės klasifikavimas}

\slide{900}

Buhalterinis balansas priklauso esamos padėties ataskaitų kategorijai.

Pelno (nuostolio), pinigų srautų ir nuosavojo kapitalo pokyčių ataskaitos
priklauso pasikeitimų ataskaitų kategorijai.

\slide{901}

\slide{902}

Pagal paskirtį:
\begin{itemize}
  \item finansinė;
  \item mokestinė;
  \item statistinė;
  \item specialiosios paskirties.
\end{itemize}

\slide{907}

Pagal turinį ir periodiškumą:
\begin{itemize}
  \item metinė;
  \item ketvirtinė;
  \item mėnesinė;
  \item savaitinė;
  \item rengiama pagal pareikalavimą.
\end{itemize}

\slide{909}

Pagal skelbimo pobūdį:
\begin{itemize}
  \item privaloma viešai skelbti;
  \item neprivaloma viešai skelbti.
\end{itemize}

\slide{910}

Pagal apimtį:
\begin{itemize}
  \item pilna;
  \item trumpa.
\end{itemize}

\subsection{Mokestinė atskaitomybė} 

\slide{903}

\subsection{Statistinė atskaitomybė}

\slide{904}

\subsection{Specialiosios paskirties atskaitomybė}

\slide{905}

\subsection{Atskaitomybė pagal turinį ir periodiškumą}

\slide{908}

\section{Atskaitomybės principai}

\slide{911}

\subsection{Suprantamumo principas}

\slide{912}

Suprantamumas reiškia, kad finansinių ataskaitų informacija turi būti
suprantama visiems vartotojams. Todėl labai svarbu, kad finansinių
ataskaitų turinys atspindėtų esmę, nebūtų dviprasmybių.

\subsection{Tinkamumo principas}

\slide{913}

Finansinių ataskaitų tinkamumas rodo, kad ataskaitos padeda vartotojams
priimti tinkamus sprendimus, daryti objektyvias išvadas. Finansinių
ataskaitų tinkamumą apibūdina jose esamos informacijos reikšmingumas.
Reikšminga informacija laikoma tokia, kurią išbraukus iš finansinės
ataskaitos, tai turės įtakos vartotojų sprendimams.

\subsection{Patikimumo principas}

\slide{914}

Patikimumas rodo, kad visos ūkinės operacijos finansinėse ataskaitose
atspindėtos objektyviai ir jose nėra klaidų. Finansinių ataskaitų
patikimumą lemia šie veiksniai:
\begin{itemize}
  \item lojalus atvaizdavimas;
  \item turinio vyravimas prieš formą;
  \item neutralumas;
  \item apdairumas, atsargumas ir tikslumas;
  \item kompleksiškumas.
\end{itemize}

\section{Balanso samprata}

\slide{915}

Balansas yra pagrindinė įmonės finansinės ataskaitos forma. Jis turi
labai didelę reikšmę atliekant įmonės finansinės būklės analizę,
todėl visose šalyse laikomas svarbiausiu finansinės analizės šaltiniu.

Norint parengti balansą pirmiausia būtina suskaičiuoti visą turtą, kuriuo
disponuoja įmonė ataskaitinio laikotarpio pabaigoje, nustatyti kokia
turto dalis buvo sunaudota per ataskaitinį laikotarpį.

\section{Pelno (nuostolio) ataskaita}

\slide{916}

Informacija apie įmonės ataskaitinio laikotarpio veiklos rezultatus
pateikiama pelno (nuostolio) ataskaitoje. Joje yra parodoma finansinio
rezultato susidarymo schema arba pelno ar nuostolio apskaičiavimo
tvarka.

Pelno (nuostolio) ataskaitoje fiksuojamos per ataskaitinį laikotarpį
uždirbtos pajamos ir sąnaudos, patirtos uždirbant tas pajamas. Jas
palyginus nustatoma ar įmonė per ataskaitinį laikotarpį uždirbo pelną
ar patyrė nuostolį.

\section{Įprastinė įmonės veikla}

\slide{917}

Įprastine įmonės veikla laikomos pasikartojančios ūkinė operacijos,
susijusios su visa įmonės veikla. Įprastinė veiklos duomenys dar grupuojami
į tipinės (veikla, kuriai vykdyti buvo įkurta įmonė) ir netipinės
veiklos straipsnius.

Duomenys, apie netipinę įmonės veiklą yra kita, finansinė ir investicinė
veikla.

\section{Pardavimo pajamos}

\slide{918}

Pardavimo pajamų straipsnyje turi būti pateikiamos grynosios pardavimo
pajamos. Jos apima įmonės pajamų sumą, uždirbtą pardavus prekes,
taip pat ir atlikus paslaugas.

\section{Pardavimo savikaina}

\slide{919}

Pardavimo savikainos straipsnyje turi būti pateikiama grynoji pardavimo
savikaina. Šiame straipsnyje pateikiama suma sąnaudų, tiesiogiai susijusių
su pardavimų ir paslaugų pajamų uždirbimu. Tai sunaudotų žaliavų,
komplektavimo gaminių bei tiesioginio darbo užmokesčio, apskaičiuoto
parduotų prekių gamintojams, suma.

\section{Veiklos sąnaudos}

\slide{920}

Veiklos sąnaudos apima pardavimo, bendrąsias ir administracines sąnaudas.
Vykdant savo veiklą, prekiaujant ar teikiant paslaugas susidaro tiesiogiai
parduotų prekių ir atliktų darbų savikainai nepriskirtinų sąnaudų, kurios
ir pateikiamos atskirai šiame straipsnyje.

\section{Tipinės ir netipinės veiklos pelnas}

\slide{921}

\section{Finansinės ir investicinės veiklos pajamos}

\slide{922}

\section{Įprastinės veiklos pelnas}

\slide{923}

Šis rodiklis parodo įmonės įprastinė veiklos rezultatą per ataskaitinį
laikotarpį. Šis rodiklis apskaičiuojamas prie tipinės veiklos pelno
(nuostolio) pridedant kitos veiklos bei finansinės ir investicinės
veiklos rezultatus.

\section{Pagautė (ypatingas pelnas)}

\slide{924}

Rodo netikėtą pelną dėl įmonės valdytojų nevaldomų ir nuo jų valios
nepriklausančių įvykių. Pavyzdžiui, stichinę nelaimę patyrusi įmonė
iš draudimo kompanijos gali gauti kompensaciją, viršijančią tos stichinės
nelaimės padarinių likvidavimo sumą. Šis perviršis ir yra nepriklausomai
nuo įmonės valdytojų valios uždirbtas pelnas – pagautė\footnote{
Pajamos iš šalies, prisidūrimas. DLKŽ.}.

\section{Netekimai (ypatingi praradimai)}

\slide{925}

Atspindi netikėtą nuostolį dėl įmonės valdytojų nevaldomų ir nuo jų
valios nepriklausomų įvykių.

\section{Grynojo pelno apskaičiavimas}

\slide{926}

Grynojo pelno (nuostolio) rodiklis parodo galutinį įmonės veiklos
rezultatą, tai yra įmonėje liekantį pelną, kuris gali būti
paskirstytas, arba nuostolius. Grynojo pelno rodiklis apskaičiuojamas
iš pelno (nuostolio) prieš apmokestinimą atimant pelno mokesčio
sąnaudų rodiklį.

\subsection{Grynojo pelno apskaičiavimo problemos}

\slide{927}

Pelno (nuostolio) ataskaitoje nėra tokių svarbių rodiklių, kurie
tarnauja apskaičiuojant pelno mokestį:
\begin{itemize}
  \item apmokestinamojo pelno;
  \item neapmokestinamojo pelno;
  \item išlaidų, didinančių apmokestinamąjį pelną;
  \item išlaidų, mažinančių apmokestinamąjį pelną.
\end{itemize}

Dėl to skaičiuodamos pelno mokestį įmonės privalo pildyti specialią
ataskaitą, kuri yra priskiriama mokestinėms ataskaitoms – metinę pelno
mokesčio deklaraciją.

\section{Pinigų srautų ataskaitos tikslai}

\slide{928}

\slide{931}

\begin{enumerate}
  \item Pateikti informaciją apie įmonės per ataskaitinį laikotarpį
    gautus ir išleistus pinigus.
  \item Pateikti duomenis apie pinigų panaudojimą įmonės įprastinėje
    gamybinėje ar prekybinėje, taip pat investicinėje ir finansinėje
    veikloje.
\end{enumerate}

\section{Informacijos pateikimas pinigų srautų ataskaitoje}

\slide{929}

\slide{930}

\chapter{Uždaviniai: Finansinė apskaita}

\slide{935}

\begin{tasks}
  
  \emph{Atsakymą žymėkite taip: apibraukite teisingo atsakymo numerį.}

  \begin{task}
    \begin{condition}
      Sąnaudos yra:
      \begin{enumerate}
        \titem{a} sunaudotų uždirbant pajamas išteklių dalis;
        \titem{b} pajamų ir išlaidų santykis, išreikštas tam tikro
          laikotarpio procentais;
        \titem{c} išleisti pinigai ar kitas turtas;
        \titem{d} pinigai, sumokėti už atsargas, skirtas prekėms
          gaminti.
      \end{enumerate}
    \end{condition}
    \begin{solution}
      Teisingas atsakymas yra \tref{a}.
    \end{solution}
  \end{task}

  \begin{task}
    \begin{condition}
      Apskaita – socialinis, tai yra visuomeninis mokslas, nes:
      \begin{enumerate}
        \titem{a} parodo sudėtingus visuomenėje vykstančius procesus;
        \titem{b} parodo nesudėtingus visuomenėje vykstančius procesus;
        \titem{c} parodo sudėtingus visuomenėje nevykstančius procesus;
        \titem{d} parodo sudėtingas visuomenėje esančias galimybes.
      \end{enumerate}
    \end{condition}
    \begin{solution}
      Teisingas atsakymas yra \tref{a}.
    \end{solution}
  \end{task}

  \begin{task}
    \begin{condition}
      Apskaitos sistema – tai metodų, taikomų apskaitoje, visuma,
      o bet kuris atskiras šios visumos metodas – tai:
      \begin{enumerate}
        \titem{a} veikos, susijusios su apskaitos reglamentavimu,
          tyrimo būdas;
        \titem{b} veiklos, susijusios su apskaitos tvarkymu,
          tyrimo galimybė;
        \titem{c} veiklos, nesusijusios su apskaitos tvarkymu,
          tyrimo būdas;
        \titem{d} veiklos, susijusios su apskaitos tvarkymu,
          tyrimo būdas.
      \end{enumerate}
    \end{condition}
    \begin{solution}
      Teisingas atsakymas yra \tref{d}.
    \end{solution}
  \end{task}

  \begin{task}
    \begin{condition}
      Ekonominiai ištekliai yra:
      \begin{enumerate}
        \titem{a} visos įmonės veiklai reikalingos materialios
          vertybės;
        \titem{b} vertybės, kurias įmonė parduota, o jos buvo nupirktos
          arba pagamintos, turint tikslą jas perparduoti tretiesiems
          asmenims;
        \titem{c} visos verslui reikalingos materialios ir nematerialios
          vertybės;
        \titem{d} visos verslui reikalingos nematerialios vertybės,
          turinčios savininkus.
      \end{enumerate}
    \end{condition}
    \begin{solution}
      Teisingas atsakymas yra \tref{c}.
    \end{solution}
  \end{task}

  \begin{task}
    \begin{condition}
      Kuri iš pateiktų lygybių yra teisinga:
      \begin{enumerate}
        \titem{a} $\t{turtas} + \t{skolintoji nuosavybė} =%
          \t{savininkų nuosavybė}$;
        \titem{b} $\t{turtas} + \t{savininkų nuosavybė} =%
          \t{skolintoji nuosavybė}$;
        \titem{c} $\t{turtas} - \t{skolintoji nuosavybė} =%
          \t{savininkų nuosavybė}$;
        \titem{d} $\t{turtas} - \t{skolintoji nuosavybė} +%
          \t{savininkų nuosavybė} = 0$.
      \end{enumerate}
    \end{condition}
    \begin{solution}
      Teisingas atsakymas yra \tref{c}. Čia reikia remtis
      lygybe $\t{Turtas} = \t{nuosavybė}$.
    \end{solution}
  \end{task}

  \begin{task}
    \begin{condition}
      Įmonėje esančios perparduoti skirtos prekės priskirtinos:
      \begin{enumerate}
        \titem{a} įmonės savininkų nuosavybei;
        \titem{b} įmonė turtui;
        \titem{c} banko nuosavybei, jeigu įmonė iš pastarojo paėmusi
          ilgalaikę paskolą;
        \titem{d} per ataskaitinį laikotarpį uždirbtoms pajamoms.
      \end{enumerate}
    \end{condition}
    \begin{solution}
      Teisingas atsakymas yra \tref{b}.
    \end{solution}
  \end{task}

  \begin{task}
    \begin{condition}
      Metinę finansinę atskaitomybę sudaro:
      \begin{enumerate}
        \titem{a} bendrosios pastabos, pažymos su pastabomis ir
          pastabos;
        \titem{b} pelno (nuostolių) ataskaita, pelno (nuostolio)
          paskirstymo ataskaita, pinigų srautų ataskaita;
        \titem{c} balansas ir bendrosios pastabos;
        \titem{d} finansinės ataskaitos ir paaiškinamasis raštas.
      \end{enumerate}
    \end{condition}
    \begin{solution}
      Teisingas atsakymas yra \tref{d}.
    \end{solution}
  \end{task}

  \begin{task}
    \begin{condition}
      Trumpalaikis turtas – tai turtas, kuris:
      \begin{enumerate}
        \titem{a} nesunaudojamas uždirbant vieno ataskaitinio laikotarpio
          pajamas;
        \titem{b} įmonėje naudojamas nuo 1 iki 3 metų;
        \titem{c} yra patvirtintas sutartimis ar kitais jo priklausymą
          įmonei patvirtinančiais dokumentais;
        \titem{d} įmonėje naudojamas ne ilgiau kaip vienerius metus
          arba kurio vertė mažesnė už įmonės nustatyta ir 
          paaiškinamajame rašte atskleistą minimalią ilgalaikio
          turto vertę.
      \end{enumerate}
    \end{condition}
    \begin{solution}
      Teisingas atsakymas yra \tref{d}.
    \end{solution}
  \end{task}

  \begin{task}
    \begin{condition}
      Įmonė apyvartinį kapitalą apskaičiuoja:
      \begin{enumerate}
        \titem{a} iš pastoviojo kapitalo atimdama įmonės ilgalaikio turto
          sumą;
        \titem{b} iš ilgalaikio turto, kuriuo disponuoja, atimdama
          trumpalaikio turto sumą;
        \titem{c} iš ilgalaikio ir trumpalaikio turto sumos atimdama
          ilgalaikį finansinį turtą;
        \titem{d} susumavusi visą trumpalaikį ir ilgalaikį įmonės turtą.
      \end{enumerate}
    \end{condition}
    \begin{solution}
      Teisingas atsakymas yra \tref{a}.
    \end{solution}
  \end{task}

  \begin{task}
    \begin{condition}
      Išrinkite teisingą teiginį:
      \begin{enumerate}
        \titem{a} sąnaudoms viršijus pajamas įmonė uždirba pelno, kuris
          ne tik didina bendrą turto sumą, bet ir savininkų nuosavybę;
        \titem{b} suma, kuria uždirbtos pajamos viršija jas uždirbant
          patirtas sąnaudas, vadinama uždirbtu pelnu;
        \titem{c} pelnas yra banko nuosavybė, jei įmonė pastarajam
          skolinga didesnę nei uždirbtas pelnas sumą;
        \titem{d} pelnu laikomas turto arba įsipareigojimų padidėjimas,
          kuris įvyko per ataskaitinį laikotarpį.
      \end{enumerate}
    \end{condition}
    \begin{solution}
      Teisingas atsakymas yra \tref{b}.
    \end{solution}
  \end{task}

\end{tasks}

\chapter{Uždaviniai: Finansinė apskaita}

\begin{tasks}

  \emph{Atsakymą žymėkite taip: apibraukite teisingo atsakymo numerį.}

  \begin{task}
    \begin{condition}
      Pagal įmonės principą
      \begin{enumerate}
        \titem{a} kiekviena įmonė, kuri sudaro finansinę atskaitomybę,
          nelaikoma atskiru apskaitos vienetu – į apskaitą įtraukiamas
          ir kitos įmonės turtas, nuosavas kapitalas ir įsipareigojimai;
        \titem{b} kiekviena įmonė, kuri sudaro finansinę atskaitomybę,
          laikoma atskiru apskaitos vienetu – į apskaitą įtraukiamas tik
          tos įmonės turtas, nuosavas kapitalas ir įsipareigojimai;
        \titem{c} kiekviena įmonė, tik kuri nesudaro finansinės
          atskaitomybės, laikoma atskiru apskaitos vienetu – į apskaitą
          įtraukiamas tik tos įmonės turtas, nuosavas kapitalas ir
          įsipareigojimai;
        \titem{d} kiekviena įmonė, kuri sudaro finansinę atskaitomybę,
          laikoma atskiru apskaitos vienetu – į apskaitą gali būti
          įtrauktas ir kitos įmonės turtas, nuosavas kapitalas ir
          įsipareigojimai.
      \end{enumerate}
    \end{condition}
    \begin{solution}
      Teisingas atsakymas yra \tref{b}.
    \end{solution}
  \end{task}

  \begin{task}
    \begin{condition}
      Pagal veiklos tęstinumo principą:
      \begin{enumerate}
        \titem{a} tvarkant apskaitą daroma prielaida, kad įmonės
          veiklos laikotarpis ribotas ir įmonę numatoma likviduoti –
          šis principas taikomas, kai priimamas sprendimas likviduoti
          įmonę, taip pat toms įmonėms, kurios įsteigiamos ribotam
          veiklos laikotarpiui;
        \titem{b} tvarkant apskaitą daroma prielaida, kad įmonės veiklos
          laikotarpis neribotas ir įmonės nenumatoma likviduoti –
          šis principas taikomas ir tuomet, kai priimamas sprendimas
          likviduoti įmonę, taip pat toms įmonėms, kurios įsteigiamos
          ribotam veiklos laikotarpiui;
        \titem{c} tvarkant apskaitą daroma prielaida, kad įmonės
          veiklos laikotarpis neribotas ir įmonės nenumatoma
          likviduoti tiek, kiek užtenka kapitalo – šis principas
          netaikomas, kai priimamas sprendimas likviduoti įmonę,
          taip pat toms įmonėms, kurios įsteigiamos ribotam veiklos
          laikotarpiui;
        \titem{d} tvarkant apskaitą daroma prielaida, kad įmonės
          veiklos laikotarpis neribotas ir įmonės nenumatoma
          likviduoti – šis principas netaikomas, kai priimamas
          sprendimas likviduoti įmonę, taip pat toms įmonėms, kurios
          įsteigiamos ribotam veiklos laikotarpiui.
      \end{enumerate}
    \end{condition}
    \begin{solution}
      Teisingas atsakymas yra \tref{d}.
    \end{solution}
  \end{task}

  \begin{task}
    \begin{condition}
      Pagal periodiškumo principą:
      \begin{enumerate}
        \titem{a} įmonės veikla tvarkant apskaitą suskirstoma į
          finansinius metus arba kitos trukmės ataskaitinius
          laikotarpiu, kuriems pasibaigus sudaroma finansinė
          atskaitomybė;
        \titem{b} <++>
        \titem{c} <++>
        \titem{d} <++>
      \end{enumerate}
    \end{condition}
    \begin{solution}
      Teisingas atsakymas yra \tref{a}.
    \end{solution}
  \end{task}

  \begin{task}
    \begin{condition}
      Pagal pastovumo principą:
      \begin{enumerate}
        \titem{a} įmonės pasirinktą apskaitos metodą turi taikyti
          kiekvienas finansiniais metais, apskaitos metodą galima
          keisti tik tuo atveju, jeigu tuo siekiama teisingai
          parodyti įmonės finansinių metų turtą, nuosavą kapitalą
          ir įsipareigojimus;
        \titem{b} <++>
        \titem{c} <++>
        \titem{d} <++>
      \end{enumerate}
    \end{condition}
    \begin{solution}
      Teisingas atsakymas yra \tref{a}.
    \end{solution}
  \end{task}

  \begin{task}
    \begin{condition}
      Pagal piniginio mato principą:
      \begin{enumerate}
        \titem{a} visas įmonės turtas, nuosavas kapitalas, be ne
          įsipareigojimai finansinėje atskaitomybėje išreiškiami
          pinigais;
        \titem{b} visas įmonės turtas, nuosavas kapitalas ir
          įsipareigojimai finansinėje atskaitomybėje išreiškiami
          pinigais ir kitais matavimo vienetais;
        \titem{c} visas įmonės turtas, nuosavas kapitalas ir
          įsipareigojimai finansinėje atskaitomybėje išreiškiami
          pinigais;
        \titem{d} tik įmonės turtas, bet ne nuosavas kapitalas ir
          įsipareigojimai finansinėje atskaitomybėje išreiškiami
          pinigais.
      \end{enumerate}
    \end{condition}
    \begin{solution}
      Teisingas atsakymas yra \tref{c}.
    \end{solution}
  \end{task}

  \begin{task}
    \begin{condition}
      Pagal kaupimo principą:
      \begin{enumerate}
        \titem{a} <++>
        \titem{b} ūkinės operacijos ir ūkiniai įvykiai į apskaitą
          įtraukiami po to, kai jie įvyksta, ir pateikiami tų
          ataskaitinių laikotarpių finansinėje atskaitomybėje,
          neatsižvelgiant į pinigų gavimą arba išmokėjimą – pagal
          pinigų kaupimo principą pajamos registruojamos tada, kai
          jos uždirbamos.
        \titem{c} <++>
        \titem{d} <++>
      \end{enumerate}
    \end{condition}
    \begin{solution}
      Teisingas atsakymas yra \tref{b}.
    \end{solution}
  \end{task}

  \begin{task}
    \begin{condition}
      Pagal palyginimo principą:
      \begin{enumerate}
        \titem{a} <++>
        \titem{b} pajamos, uždirbtos per ataskaitinį laikotarpį,
          siejamos su sąnaudomis, patirtomis uždirbant tas pajamas –
          finansinė ataskaitos turi būti parengtos taip, kad finansinės
          informacijos vartotojai galėtų palyginti jose pateikiamą
          informaciją su kitų ataskaitinių laikotarpių bei kitų
          įmonių pateikiama informacija ir teisingai įvertinti
          įmonės finansinės būklės pokyčius;
        \titem{c} <++>
        \titem{d} <++>
      \end{enumerate}
    \end{condition}
    \begin{solution}
      Teisingas atsakymas yra \tref{b}.
    \end{solution}
  \end{task}

  \begin{task}
    \begin{condition}
      Pagal atsargumo principą:
      \begin{enumerate}
        \titem{a} <++>
        \titem{b} <++>
        \titem{c} <++>
        \titem{d} įmonė pasirenka tokius apskaitos metodus, kuriais
          įmonės turto, nuosavo kapitalo ir įsipareigojimų bei
          pajamų ir sąnaudų vertė negali būti nepagrįstai padidinta
          arba nepagrįstai sumažinta.
      \end{enumerate}
    \end{condition}
    \begin{solution}
      Teisingas atsakymas yra \tref{d}.
    \end{solution}
  \end{task}

  \begin{task}
    \begin{condition}
      Pagal neutralumo principą:
      \begin{enumerate}
        \titem{a} apskaitos informacija pateikiama nešališkai – jos
          pateikimas neturėtų daryti įtakos apskaitos informacijos
          vartotojų priimamiems sprendimams ir juo neturėtų būti
          siekiama iš anksto numatyto rezultato;
        \titem{b} <++>
        \titem{c} <++>
        \titem{d} <++>
      \end{enumerate}
    \end{condition}
    \begin{solution}
      Teisingas atsakymas yra \tref{a}.
    \end{solution}
  \end{task}

  \begin{task}
    \begin{condition}
      Pagal turinio svarbos principą:
      \begin{enumerate}
        \titem{a} ūkinės operacijos ir ūkiniai įvykiai į apskaitą
          traukiami tik pagal jų turinį ir ekonominę prasmę, o ne
          pagal jų juridinę formą;
        \titem{b} ūkinės operacijos ir ūkiniai įvykiai į apskaitą
          traukiami pagal jų turinį ir ekonominę prasmę, ir pagal
          jų juridinę formą;
        \titem{c} ūkinės operacijos ir ūkiniai įvykiai į apskaitą
          traukiami pagal jų turinį ir ekonominę prasmę, o ne tik
          pagal jų juridinę formą;
        \titem{d} ūkinės operacijos ir ūkiniai įvykiai į apskaitą
          netraukiami pagal jų turinį ir ekonominę prasmę, o tik
          pagal jų juridinę formą.
      \end{enumerate}
    \end{condition}
    \begin{solution}
      Teisingas atsakymas yra \tref{c}.
    \end{solution}
  \end{task}
  
\end{tasks}

\chapter{Uždaviniai}

\strong{11 - Atsakomybes + Pratybos/1 pratybos.pdf}

\begin{tasks}
  
  \begin{task}
    \begin{condition}
      Nustatykite, kuriose iš „X“ įmonės sąskaitų atspindimas turtas,
      o kuriose – nuosavybė:
      \begin{enumerate}
        \item skolos tiekėjams;
        \item skolos rangovams – tiekėjams;
        \item gautinos sumos;
        \item nepaskirstytasis ataskaitinių metų pelnas;
        \item iš anksto apmokėta nuoma;
        \item pinigai;
        \item įrengimai;
        \item akcinis kapitalas;
        \item medžiagos.
      \end{enumerate}
    \end{condition}
    \begin{solution}
      \begin{tabularx}{\tablewidth}[]{X | X}
        Turtas & Nuosavybė \\
        \hline
        gautinos sumos
          & skolos tiekėjams \\
        iš anksto apmokėta nuoma
          & skolos rangovams – tiekėjams \\
        pinigai
          & nepaskirstytasis ataskaitinių metų pelnas \\
        įrengimai
          & akcinis kapitalas \\
        medžiagos & \\
      \end{tabularx}
    \end{solution}
  \end{task}

  \begin{task}
    \begin{condition}
      Fundamentinėse apskaitos lygybėse apskaičiuokite trūkstamas
      sumas.
    \end{condition}
    \begin{solution}
      \begin{enumerate}
        \item
          \begin{align*}
            \t{Akcinis kapitalas}
            &= \underbrace{(12000 + 2000 + 500 + 300 + 1000)}_{\t{Turtas}}
              - \underbrace{(2500 + 1500)}_{\t{Nuosavybė}} \\
            &= 15800 - 4000 \\
            &= 11800 Lt \\
          \end{align*}
        \item
          \begin{align*}
            \t{Įrengimai}
            &= \underbrace{(2000 + 1750 + 1250)}_{\t{Nuosavybė}}
              - \underbrace{(1550 + 750 + 200 + 1200)}_{\t{Turtas}} \\
            &= 5000 - 3700 \\
            &= 1300 Lt \\
          \end{align*}
        \item
          \begin{align*}
            \t{Medžiagos}
            &= TODO
          \end{align*}
        \item
          \begin{align*}
            \t{Skolos bankui}
            &= TODO
          \end{align*}
        \item
          \begin{align*}
            \t{Pinigai}
            &= TODO
          \end{align*}
        \item
          \begin{align*}
            \t{Skolos tiekėjams}
            &= TODO
          \end{align*}
        \item
          \begin{align*}
            \t{Iš anksto sumokėta nuoma}
            &= TODO
          \end{align*}
        \item
          \begin{align*}
            \t{Skolos bankui} &= \\
            \t{Skolos tiekėjams}
            &= \frac{%
              \underbrace{(950 + 1100 + 680 + 250 + 1020)}_{\t{Turtas}}
              - \underbrace{(2000)}_{\t{Nuosavybė}}
              }{2} \\
            &= \frac{4000 - 2000}{2} \\
            &= 1000 Lt
          \end{align*}
      \end{enumerate}
    \end{solution}
  \end{task}

  \begin{task}
    \begin{condition}
      Fundamentinėse apskaitos lygybėse apskaičiuokite trūkstamas
      sumas.
    \end{condition}
    \begin{solution}
      \begin{enumerate}
        \item
          \begin{align*}
            \t{Atsargos}
            &= \underbrace{
              (40000 + 20100 + 5000 + 2300 - 1600)}_{\t{Nuosavybė}}
              - \underbrace{
              (10100 + 20500 + 21000 + 11300)}_{\t{Turtas}} \\
            &= 65800 - 62900 \\
            &= 2900 Lt \\
          \end{align*}
        \item
          \begin{align*}
            \t{Medžiagos}
            &= TODO \\
          \end{align*}
        \item
          \begin{align*}
            \t{Skolos bankui} &= \\
            \t{Skolos tiekėjams}
            &= \frac{%
              \underbrace{
              (17000 + 31440 + 23560 + 7540 + 35460)}_{\t{Turtas}}
              - \underbrace{(45000 + 12320 - 7680)}_{\t{Nuosavybė}}
              }{2} \\
            &= \frac{115000 - 49640}{2} \\
            &= \frac{65360}{2} \\
            &= 32680 Lt \\
          \end{align*}
        \item
          \begin{align*}
            \t{Sąnaudos}
            &= TODO \\
          \end{align*}
        \item
          \begin{align*}
            \t{Pinigai}
            &= TODO \\
          \end{align*}
        \item
          \begin{align*}
            \t{Akcinis kapitalas}
            &= TODO \\
          \end{align*}
      \end{enumerate}
    \end{solution}
  \end{task}

  \begin{task}
    \begin{condition}
      Įmonės „X“ 2010 m. gruodžio 31 d. balanso kai kurie straipsniai
      įrašyti netinkamose vietose. Teisingai sugrupuokite straipsnius,
      atsižvelgdami į jų atsispindėjimo balanse tvarką.
    \end{condition}
    \begin{solution}
      \begin{tabularx}{\tablewidth}[]{l | X}
        \multicolumn{2}{c}{Turtas} \\
        \hline
        Atsargos & 28476 \\
        Pinigai & 57508 \\
        Įrengimai & 240000 \\
        Patentai & 193010 \\
        Ateinančių laikotarpių sąnaudos & 30096 \\
        Perparduoti skirtos prekės & 84704 \\
        Pirkėjų skola & 107296 \\
        Medžiagos & 8000 \\
        \hline
        Turtas iš viso: & 749090 \\
        \multicolumn{2}{c}{Nuosavybė} \\
        \hline
        Trumpalaikė banko paskola & 182800 \\
        Nepaskirstytasis pelnas & 40306 \\
        Ataskaitinių metų pelnas & 200000 \\
        Akcinis kapitalas & 240000 \\
        Ilgalaikė paskola pastatui statyti & 85984 \\
        \hline
        Nuosavybė iš viso: & 749090 \\
      \end{tabularx}
    \end{solution}
  \end{task}

  \begin{task}
    \begin{condition}
      Įmonės „X“ 2010 m. gruodžio 31 d. balanso kai kurie straipsniai
      įrašyti netinkamose vietose. Teisingai sugrupuokite straipsnius,
      atsižvelgdami į jų atsispindėjimo balanse tvarką.
    \end{condition}
    \begin{solution}
      \begin{tabularx}{\tablewidth}[]{l | X}
        \multicolumn{2}{c}{Turtas} \\
        \hline
        Ateinančių laikotarpių nuoma & 11100 \\
        Transporto priemonės & 8100 \\
        Įrengimai & 30100 \\
        Pinigai & 2800 \\
        Skolos įmonei & 1400 \\
        Žemė & 15000 \\
        Įrengimų nusidėvėjimo suma & (1100) \\
        \hline
        Turtas iš viso: & 67400 \\
        \multicolumn{2}{c}{Nuosavybė} \\
        \hline
        Mokėtinas atlyginimas & 500 \\
        Ilgalaikės skolos bankui & 17600 \\
        Akcinis kapitalas & 20200 \\
        Nepaskirstytasis pelnas & 29100 \\
        \hline
        Nuosavybė iš viso: & 67400 \\
      \end{tabularx}
    \end{solution}
  \end{task}

  \begin{task}
    \begin{condition}
      Žemiau pateikti „Skonis“ įmonės sąskaitų likučiai 2010 m.
      gruodžio 31d. Sudarykite AB „Skonis“ 2010 m. gruodžio 31 d.
      balansą.
    \end{condition}
    \begin{solution}
      \begin{tabularx}{\tablewidth}[]{l | X}
        \multicolumn{2}{c}{Turtas} \\
        \hline
        Pinigai & 14300 \\
        Iš anksto apmokėta nuoma & 400 \\
        Įrenginių įsigijimo savikaina & 46500 \\
        Žemė & 150000 \\
        Pastatai & 137120 \\
        Patentai & 37100 \\
        Pastatų ir įrengimų nusidėvėjimas & (2000) \\
        Pirkėjų skolos & 12700 \\
        \hline
        Turtas iš viso: & 396120 \\
        \multicolumn{2}{c}{Nuosavybė} \\
        \hline
        Akcinis kapitalas & 165500 \\
        Skolos tiekėjams & 39500 \\
        Ilgalaikės skolos bankui & 104000 \\
        Nepaskirstytasis pelnas & 87120 \\
        \hline
        Nuosavybė iš viso: & 396120 \\
      \end{tabularx}
    \end{solution}
  \end{task}

  \begin{task}
    \begin{condition}
      Šios operacijos buvo atliktos AB „X“ per 2010 m. gruodį -
      pirmąjį jos veiklos mėnesį. Sudarykite ūkinių operacijų
      įtakos apskaitinei lygybei lentelę. Parenkite 2010 m. gruodžio
      1 dienos ir 31 dienos balansą.
    \end{condition}
    \begin{solution}
      \begin{tabularx}{\ltablewidth}[]{p{0.5cm}| X|X|X|X | X|X|X|X }
        & \multicolumn{4}{c|}{Turtas} & \multicolumn{4}{c}{Nuosavybė} \\
        {\small Eil. Nr.} & {\small Transporto priemonės} &
        {\small Atsargos }& {\small Pinigai} &
        {\small Iš anksto apmokėta nuoma} & {\small Akcinis kapitalas} &
        {\small Skola Tiekėjams} & {\small Skola bankui} &
        {\small Pajamos (+) Sąnaudos (-)} \\
        \hline
        1 & &&100 000& & 100 000 &&& \\
        2 & &&40 000& & &&40 000& \\
        3 & &&(750)&500 & &&&(250) \\
        4 & 7000&&(7000)& & &&& \\
        5 & &&(350)& & &&&(350) \\
        6 & &500&(500)& & &&& \\
        7 & &&1200& & &&&1200 \\
        8 & &&750& & &&&750 \\
        9 & &&(3400)& & &&&(3400) \\
        10& &50&(50)& & &&& \\
        11& &600&& & &600&& \\
        12& &&4000& & &&&4000 \\
        13& &&1700& & &&&1700 \\
        14& &&(600)& & &(600)&& \\
        16& &&(520)& & &&&(520) \\
        \hline
        Suma & 7000 & 1150 & 134480 & 500 & 100000 & 0 & 40000 & 3130 \\
        \hline
        &
        \multicolumn{3}{r}{Turtas iš viso:} & 143130 &
        \multicolumn{3}{r}{Nuosavybė iš viso:} & 143130 \\
      \end{tabularx}

      \begin{tabularx}{\tablewidth}[]{c X c}
        \multicolumn{3}{c}{Gruodžio 31 dienos balansas} \\
        Eil. Nr. & \multicolumn{2}{c}{Turtas} \\
        1. & Transporto priemonės & 700 \\
        2. & Atsargos & 1150 \\
        3. & Pinigai & 134480 \\
        4. & Iš anksto apmokėta nuoma & 500 \\
        \multicolumn{2}{r}{Turtas iš viso:} & 143130 \\
        Eil. Nr. & \multicolumn{2}{c}{Nuosavybė} \\
        1. & Akcinis kapitalas & 100000 \\
        2. & Skola tiekėjams & 0 \\
        3. & Skola bankui & 40000 \\
        4. & Pelnas & 3130 \\
        \multicolumn{2}{r}{Nuosavybė iš viso:} & 143130 \\
      \end{tabularx}

      \begin{tabularx}{\tablewidth}[]{c X c}
        \multicolumn{3}{c}{Gruodžio 1 dienos balansas} \\
        Eil. Nr. & \multicolumn{2}{c}{Turtas} \\
        1. & Pinigai & 100000 \\
        \multicolumn{2}{r}{Turtas iš viso:} & 100000 \\
        Eil. Nr. & \multicolumn{2}{c}{Nuosavybė} \\
        1. & Akcinis kapitalas & 100000 \\
        \multicolumn{2}{r}{Nuosavybė iš viso:} & 100000 \\
      \end{tabularx}
      
    \end{solution}
  \end{task}
\end{tasks}
