\chapter{Atsakomybės}

\setslideprefix{11 - Atsakomybes + Pratybos/fotkinta/P4301}

\section{Finansinės atskaitomyubės teisinė bazė}

\slide{887}

\section{Finansinių ataskaitų rinkinys}

\slide{888}

\section{Finansinė atskaitomybė AB ir UAB}

\slide{889}

Pagal Akcinių bendrovių įstatymą, metinė finansinė atskaitomybė AB
ir UAB turi patvirtinta…

\section{Pilna finansinė atskaitomybė}

\slide{890}

Finansinės atskaitomybės rūšys: Pilna ir trumpa.

Pilną finansinę atskaitomybę sudaro:
\begin{itemize}
  \item pilnas balansas;
  \item pilna pelno (nuostolių) ataskaita;
  \item nuosavojo kapitalo pokyčių ataskaita;
  \item pinigų srautų ataskaita;
  \item pilnas aiškinamasis raštas.
\end{itemize}

\subsection{Balansas}

\slide{891}

Balansas yra finansinė ataskaita, kurioje parodomas visas įmonės turtas,
nuosavas kapitalas ir įsipareigojimai paskutinę ataskaitinio laikotarpio
dieną.

\subsection{Pelno (nuostolių) ataskaita}

\slide{892}

Pelno (nuostolių) ataskaita yra finansinė ataskaita, kurioje parodomos
visos įmonės per ataskaitinį laikotarpį uždirbtos pajamos, patirtos
sąnaudos ir gauti rezultatai.

\subsection{Pinigų srautų ataskaita}

\slide{893}

Pinigų srautų ataskaita yra finansinė ataskaita, kurioje pateikiama
informacija apie įmonės pinigų ir pinigų ekvivalentų pasikeitimus
per ataskaitinį laikotarpį.

\subsection{Nuosavo kapitalo pokyčių ataskaita}

\slide{894}

Nuosavo kapitalo pokyčių ataskaita yra finansinė ataskaita, kurioje
parodomas įmonės nuosavo kapitalo pokytis per ataskaitinius ir
praėjusius finansinius metus.

\subsection{Aiškinamasis raštas}

\slide{895}

Aiškinamasis raštas yra sudedamoji finansinių ataskaitų rinkinio dalis,
kurioje pateikiama papildoma balanse, pelno (nuostolių), pinigų srautų
ir nuosavo kapitalo pokyčių ataskaitose neatskleista ar nepakankamai
detaliai atskleista reikšminga informacija.

\section{Trumpa finansinė atskaitomybė}

\slide{896}

Iš principo trumpa nuo pilnos skiriasi tuo, kad trumpoje nėra pinigų
srautų ataskaitos.

Trumpą finansinę atskaitomybę sudaro:
\begin{itemize}
  \item trumpas balansas;
  \item trumpa pelno (nuostolių) ataskaita;
  \item nuosavojo kapitalo pokyčių ataskaita;
  \item trumpas aiškinamasis raštas.
\end{itemize}

\subsection{Rengiama tada}

\slide{897}

…
\begin{itemize}
  \item grynosios pardavimo pajamos nesiekia 10 mln. Lt per metus.
  \item balanse nurodyto turto vertė nesiekia 6 mln. Lt;
  \item vidutinis metinis darbuotojų skaičius pagal sąrašą
    neviršija 15.
\end{itemize}

\section{Finansinė atskaitomybė}

\slide{899}

\section{Įmonių atskaitomybės klasifikavimas}

\slide{900}

Buhalterinis balansas priklauso esamos padėties ataskaitų kategorijai.

Pelno (nuostolio), pinigų srautų ir nuosavojo kapitalo pokyčių ataskaitos
priklauso pasikeitimų ataskaitų kategorijai.

\slide{901}

\slide{902}

Pagal paskirtį:
\begin{itemize}
  \item finansinė;
  \item mokestinė;
  \item statistinė;
  \item specialiosios paskirties.
\end{itemize}

\slide{907}

Pagal turinį ir periodiškumą:
\begin{itemize}
  \item metinė;
  \item ketvirtinė;
  \item mėnesinė;
  \item savaitinė;
  \item rengiama pagal pareikalavimą.
\end{itemize}

\slide{909}

Pagal skelbimo pobūdį:
\begin{itemize}
  \item privaloma viešai skelbti;
  \item neprivaloma viešai skelbti.
\end{itemize}

\slide{910}

Pagal apimtį:
\begin{itemize}
  \item pilna;
  \item trumpa.
\end{itemize}

\subsection{Mokestinė atskaitomybė} 

\slide{903}

\subsection{Statistinė atskaitomybė}

\slide{904}

\subsection{Specialiosios paskirties atskaitomybė}

\slide{905}

\subsection{Atskaitomybė pagal turinį ir periodiškumą}

\slide{908}

\section{Atskaitomybės principai}

\slide{911}

\subsection{Suprantamumo principas}

\slide{912}

Suprantamumas reiškia, kad finansinių ataskaitų informacija turi būti
suprantama visiems vartotojams. Todėl labai svarbu, kad finansinių
ataskaitų turinys atspindėtų esmę, nebūtų dviprasmybių.

\subsection{Tinkamumo principas}

\slide{913}

Finansinių ataskaitų tinkamumas rodo, kad ataskaitos padeda vartotojams
priimti tinkamus sprendimus, daryti objektyvias išvadas. Finansinių
ataskaitų tinkamumą apibūdina jose esamos informacijos reikšmingumas.
Reikšminga informacija laikoma tokia, kurią išbraukus iš finansinės
ataskaitos, tai turės įtakos vartotojų sprendimams.

\subsection{Patikimumo principas}

\slide{914}

Patikimumas rodo, kad visos ūkinės operacijos finansinėse ataskaitose
atspindėtos objektyviai ir jose nėra klaidų. Finansinių ataskaitų
patikimumą lemia šie veiksniai:
\begin{itemize}
  \item lojalus atvaizdavimas;
  \item turinio vyravimas prieš formą;
  \item neutralumas;
  \item apdairumas, atsargumas ir tikslumas;
  \item kompleksiškumas.
\end{itemize}

\section{Balanso samprata}

\slide{915}

Balansas yra pagrindinė įmonės finansinės ataskaitos forma. Jis turi
labai didelę reikšmę atliekant įmonės finansinės būklės analizę,
todėl visose šalyse laikomas svarbiausiu finansinės analizės šaltiniu.

Norint parengti balansą pirmiausia būtina suskaičiuoti visą turtą, kuriuo
disponuoja įmonė ataskaitinio laikotarpio pabaigoje, nustatyti kokia
turto dalis buvo sunaudota per ataskaitinį laikotarpį.

\section{Pelno (nuostolio) ataskaita}

\slide{916}

Informacija apie įmonės ataskaitinio laikotarpio veiklos rezultatus
pateikiama pelno (nuostolio) ataskaitoje. Joje yra parodoma finansinio
rezultato susidarymo schema arba pelno ar nuostolio apskaičiavimo
tvarka.

Pelno (nuostolio) ataskaitoje fiksuojamos per ataskaitinį laikotarpį
uždirbtos pajamos ir sąnaudos, patirtos uždirbant tas pajamas. Jas
palyginus nustatoma ar įmonė per ataskaitinį laikotarpį uždirbo pelną
ar patyrė nuostolį.

\section{Įprastinė įmonės veikla}

\slide{917}

Įprastine įmonės veikla laikomos pasikartojančios ūkinė operacijos,
susijusios su visa įmonės veikla. Įprastinė veiklos duomenys dar grupuojami
į tipinės (veikla, kuriai vykdyti buvo įkurta įmonė) ir netipinės
veiklos straipsnius.

Duomenys, apie netipinę įmonės veiklą yra kita, finansinė ir investicinė
veikla.

\section{Pardavimo pajamos}

\slide{918}

Pardavimo pajamų straipsnyje turi būti pateikiamos grynosios pardavimo
pajamos. Jos apima įmonės pajamų sumą, uždirbtą pardavus prekes,
taip pat ir atlikus paslaugas.

\section{Pardavimo savikaina}

\slide{919}

Pardavimo savikainos straipsnyje turi būti pateikiama grynoji pardavimo
savikaina. Šiame straipsnyje pateikiama suma sąnaudų, tiesiogiai susijusių
su pardavimų ir paslaugų pajamų uždirbimu. Tai sunaudotų žaliavų,
komplektavimo gaminių bei tiesioginio darbo užmokesčio, apskaičiuoto
parduotų prekių gamintojams, suma.

\section{Veiklos sąnaudos}

\slide{920}

Veiklos sąnaudos apima pardavimo, bendrąsias ir administracines sąnaudas.
Vykdant savo veiklą, prekiaujant ar teikiant paslaugas susidaro tiesiogiai
parduotų prekių ir atliktų darbų savikainai nepriskirtinų sąnaudų, kurios
ir pateikiamos atskirai šiame straipsnyje.

\section{Tipinės ir netipinės veiklos pelnas}

\slide{921}

\section{Finansinės ir investicinės veiklos pajamos}

\slide{922}

\section{Įprastinės veiklos pelnas}

\slide{923}

Šis rodiklis parodo įmonės įprastinė veiklos rezultatą per ataskaitinį
laikotarpį. Šis rodiklis apskaičiuojamas prie tipinės veiklos pelno
(nuostolio) pridedant kitos veiklos bei finansinės ir investicinės
veiklos rezultatus.

\section{Pagautė (ypatingas pelnas)}

\slide{924}

Rodo netikėtą pelną dėl įmonės valdytojų nevaldomų ir nuo jų valios
nepriklausančių įvykių. Pavyzdžiui, stichinę nelaimę patyrusi įmonė
iš draudimo kompanijos gali gauti kompensaciją, viršijančią tos stichinės
nelaimės padarinių likvidavimo sumą. Šis perviršis ir yra nepriklausomai
nuo įmonės valdytojų valios uždirbtas pelnas – pagautė\footnote{
Pajamos iš šalies, prisidūrimas. DLKŽ.}.

\section{Netekimai (ypatingi praradimai)}

\slide{925}

Atspindi netikėtą nuostolį dėl įmonės valdytojų nevaldomų ir nuo jų
valios nepriklausomų įvykių.

\section{Grynojo pelno apskaičiavimas}

\slide{926}

Grynojo pelno (nuostolio) rodiklis parodo galutinį įmonės veiklos
rezultatą, tai yra įmonėje liekantį pelną, kuris gali būti
paskirstytas, arba nuostolius. Grynojo pelno rodiklis apskaičiuojamas
iš pelno (nuostolio) prieš apmokestinimą atimant pelno mokesčio
sąnaudų rodiklį.

\subsection{Grynojo pelno apskaičiavimo problemos}

\slide{927}

Pelno (nuostolio) ataskaitoje nėra tokių svarbių rodiklių, kurie
tarnauja apskaičiuojant pelno mokestį:
\begin{itemize}
  \item apmokestinamojo pelno;
  \item neapmokestinamojo pelno;
  \item išlaidų, didinančių apmokestinamąjį pelną;
  \item išlaidų, mažinančių apmokestinamąjį pelną.
\end{itemize}

Dėl to skaičiuodamos pelno mokestį įmonės privalo pildyti specialią
ataskaitą, kuri yra priskiriama mokestinėms ataskaitoms – metinę pelno
mokesčio deklaraciją.

\section{Pinigų srautų ataskaitos tikslai}

\slide{928}

\slide{931}

\begin{enumerate}
  \item Pateikti informaciją apie įmonės per ataskaitinį laikotarpį
    gautus ir išleistus pinigus.
  \item Pateikti duomenis apie pinigų panaudojimą įmonės įprastinėje
    gamybinėje ar prekybinėje, taip pat investicinėje ir finansinėje
    veikloje.
\end{enumerate}

\section{Informacijos pateikimas pinigų srautų ataskaitoje}

\slide{929}

\slide{930}

\chapter{Uždaviniai: Finansinė apskaita}


7. Metinę finansinę atskaitomybę sudaro: d finansinės ataskaitos
ir paaiškinamasis raštas.

8. Trumpalaikis turtas, tai turtas, kuris: d įmonėje naudojamas
ne ilgiau kaip vienerius metus arba kurio vertė mažesnė už
įmonės nustatytą ir paaiškinamajame rašte atskleistą minimalią
ilgalaikio turto vertę.

9. Įmonė apyvartinį kapitalą apskaičiuoja: a iš pastoviojo kapitalo
atimdama įmonės ilgalaikio turto sumą.

10. Išrinkite teisingą teiginį: b suma, kuria uždirbtos…

\chapter{Uždaviniai 2}

1. Pagal įmonės principą:
b Kiekviena įmonė, kuri sudaro  finansinę atskaitomybę…

2. Pagal veiklos tęstinumo principą:
d Tvarkant apskaitą daroma prielaida, kad įmonės veiklos laikotarpis
neribotas ir įmonės nenumatoma likviduoti. …

3. Pagal periodiškumo principą:
a įmonės veikla tvarkant apskaitą suskirstoma į finansinius metus
arba kitos trukmės ataskaitinius laikotarpius, kuriems pasibaigus
sudaroma finansinė atskaitomybė.

4. Pagal pastovumo principą:
a Įmonės pasirinktą apskaitos metodą turi taikyti…

5. Pagal piniginio mato principą:
c Visas įmonės turtas, nuosavas kapitalas ir įsipareigojimai
finansinėje atskaitomybėje išreiškiami pinigais.

6. Pagal kaupimo principą:
b Ūkinės operacijos … neatsižvelgiant …

7. Pagal palyginimo principą:
b Pajamos, uždirbtos per ataskaitinį laikotarpį, siejamos su
sąnaudomis, patirtomis uždirbant tas pajamas.

8. Pagal atsargumo principą:
d Įmonė pasirenka tokius apskaitos metodus…

9. Pagal neutralumo principą:
a Apskaitos informacija pateikiama nešališkai. …

10. Pagal turinio svarbos principą:
c Ūkinės operacijos ir ūkiniai įvykiai į apskaitą traukiami pagal jų
turinį ir ekonominę prasmę, o ne tik pagal jų juridinę formą.

\chapter{Uždaviniai 3}

\section{1 užduotis}

\begin{enumerate}
  \item skolos tiekėjamas (nuosavybė);
  \item skolos rangovams-tiekėjamas (nuosavybė);
  \item gautinos sumos (turtas);
  \item nepaskirstytasis ataskaitinių metų pelnas (nuosavybė);
  \item iš anksto apmokėta nuoma (turtas);
  \item pinigai (turtas);
  \item įrenginiai (turtas);
  \item akcinis kapitalas (nuosavybė);
  \item medžiagos (turtas).
\end{enumerate}

\section{2 užduotis}

1.8 = (12000 + 2000 + 500 + 300 + 1000) - (2500 + 1500) = 11800
2.3 = 1300
…
2.9 = 2.10 = 1000

\section{3 užduotis}

$1.5 = (40000 + 20100 + 5000 + 2300 - 1600) - (10100 + 20500 + 21000 + 11300) = $
$3.9 = 3.10 = (17000 + 31440 + 23560 + 7540 + 35460) - (… 12320  - 7680) = 2 \cdot 32680$

\section{4 užduotis}

Turtas:
Atsargos
Pinigai
Įrengimai
Patentai
Ateinančių laikotarpių sąnaudos
Perparduoti skirtos prekės
Pirkėjų skola
Medžiagos

Nuosavybė:
Trumpalaikė banko paskola
Nepaskirstytasis pelnas
Ataskaitinių metų pelnas
Akcinis kapitalas
Ilgalaikė paskola pastatui statyti

\section{5 užduotis}

Nuosavybė:
Akcinis kapitalas
Nepaskirstytas pelnas

Turtas:
Žemė
Įrengimų nusidėvėjimo suma (skliausteliai reiškia, kad su minusu)

\section{6 užduotis}

turtas:
pinigai,
nuoma
įrengimų įsigijimo savikaina
žemė
patentai
…

\section{7 užduotis}

Nuosavybė:
1. Akcinis kapitalas 100 000 Lt
2. Skola bankui 40 000 Lt
3. (500) iš anksto apmokėta nuoma, (250) sąnaudos.
4.
5. Sąnaudos (350) Lt
6.
7. Pajamos 1200 Lt
8. Pajamos 750 Lt
9. Sąnaudos (3400) Lt
10.
11. Skola tiekėjams
12 Pajamos
13. Pajamos
14. - Skola tiekėjams
15. Sąnaudos (520)

Turtas:
1. Pinigai 100 000 Lt
2. Pinigai 40 000 Lt
3. Pinigai (750) Lt
4. Pinigai (7000) Lt Transporto priemonės 7000 Lt
5. Pinigai (350) Lt
6. Pinigai (500) Lt, Atsargos 500 Lt
7. Pinigai 1200 Lt,
8. Pinigai 750 Lt
9. Pinigai (3400) Lt
10. Pinigai, atsargos
11. Atsargos 600
12. Pinigai
13. Pinigai
14. -Pinigai
15. Pinigai (520)

Gruodžio 31 dieną
Turtas:
1. Transporto priemonė 7000
2. Atsargos 1150
4. Pinigai 134480
5. Iš anksto apmokėta nuoma 500
Iš viso: 143130
Nuosavybė:
1. Akcinis kapitalas 100000
2. Skola tiekėjams 0
3. Skola bankui 40000
4. Pelnas 3130
Iš viso 143130
