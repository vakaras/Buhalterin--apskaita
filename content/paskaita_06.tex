\chapter{Uždaviniai: Individualios veiklos ir verslo liudijimo turėtojų
apmokestinimas}

\begin{tasks}
  
  \begin{task}
    \begin{condition}
      Nurodykite pagrindinius apmokestinimo skirtumus tarp individualios
      veiklos vykdytojo ir verslo liudijimo turėtojo kartu
      pagrindžiant ir atitinkamos apskaitos politikos vykdymo
      scenarijų.
    \end{condition}
    \begin{solution}
      
    \end{solution}<++>
  \end{task}<++>
\end{tasks}<++>


\begin{enumerate}
  \item Nurodykite pagrindinius apmokestinimo skirtumus tarp
    individualios veiklos vykdytojo ir verslo liudijimo turėtojo
    kartu pagrindžiant ir atitinkamos apskaitos politikos
    vykdymo scenarijų.

    apmokestinamos pajamos := pajamos - sąnaudos.

    Verslo liudijimui:
    \begin{itemize}
      \item GPM = 120 Lt (kiekviena savivaldybė gali nusistatyti pati,
        bet ne mažesni, nei 120 Lt)
      \item VSDF = 180 Lt.
      \item PSDF = 72 Lt ir gale metų nuo pusės pajamų 9\%.
    \end{itemize}

    Individuali veikla pagal pažymą:
    \begin{itemize}
      \item GPM = 5\%
      \item VSDF = 28,5 \% nuo 50 \% apmokestinamų pajamų (pajamos -
        sąnaudos)
      \item PSDF = 9\% nuo 50\% apmokestinamų pajamų.
    \end{itemize}

    TODO: Atlikėjų: P3121796.JPG ir P3121803.JPG
    TODO: Advokatų: P3121806.JPG

  \item Nurodykite pagrindinius nuolatinio ir nenuolatinio Lietuvos
    gyventojo apmokestinimo gyventojų p

    Nenuolatinis turi daugiau neleidžiamų atskaitymų = nuolatinio
    gyventojo neleidžiami atskaitymai + palūkanos ir honorarai,
    jei jie mokami nenuolatiniam Lietuvos gyventojui besiverčiant
    individualia veikla. PVM mokėtojais registruojami iš karto,
    nežiūrint į 100 000 Lt ribą.

  \item Apskaičiuokite visus mokėtinus mokesčius ir nurodykite jų
    mokėjimo terminus, jeigu nuolatinis Lietuvos gyventojas, ne
    PVM mokėtojas, neturintis jokių mokestinių lengvatų ir
    turintis 2011-02-02 – 2011-05-20 laikotarpiu galiojantį verslo
    liudijimą patyrė 100 000 Lt …

    Kadangi verslo liudijimas, tai nevertinam nei pajamų nei
    sąnaudų.
    
    \begin{align*}
      GPM
      &= 120\cdot \frac{27}{28} + 2\cdot 120 + 120\cdot\frac{20}{31} \\
      &= 433,13 Lt \\
      VSDF
      &= 180\cdot \frac{27}{28} + 2\cdot 180 + 180\cdot\frac{20}{31} \\
      &= 649,70 Lt \\
      PSDF
      &= (72 + 72 + 72) = 216 \\
      \intertext{ir iki 2011-06-15 dienos sumokėti:} 
      (250000 - 100000) \cdot 0,09 &= 13500 - 216 = 13284 \\
      \intertext{Kadangi 250000 > 100000, reikia mokėti PVM:}
      (250 000 - 100 000) \cdot PVM = …
    \end{align*}

  \item Apskaičiuokite visus mokėtinus…

    Apmokestinamos pajamos: $ 250 000 - 100 000 = 150 000$.
    \begin{align*}
      PSD &= \frac{150 000}{2} \cdot 0,09 = 6750 Lt \\
      VSD &= \frac{150 000}{2} \cdot 0,285 = 21375 Lt \\
      GPM \t{„bazė“} &= 150 000 - 21375 - 6750 = 121875 Lt \\
      MNPD (\t{Metinis neapmokestinamų pajamų dydis}) &=
      5640 - 0,2\cdot(121875 - 9600) = -16815 \t{kadangi <0, tai
      prisitaikyti negalime.} \\
      GPM &= 121875 \cdot 0,05 = 6093,75 Lt \\
      PVM &= 250 000 \cdot 0,21 - 100000 \cdot 0,21 = 31500 Lt \\
    \end{align*}

    Apmokestinamosios pajamos taikant $30\%$ fiksuota tarifą:
    $250 000 \cdot 0,7 = 175 000$.
    Tada:
    \begin{align*}
      PSD = 7875 Lt \\
      VSD = 24937,5 Lt \\
      GPM = 8750 Lt \t{Kadangi naudojam 30\%, tai nei PSD nei VSD
        neįtraukiam į leidžiamus atskaitymus.}\\
    \end{align*}

    Dar reikėjo apskaičiuoti ir PVM.
  \item Apskaičiuokite visus mokėtinus mokesčius ir nurodykite jų mokėjimo
    terminus, jeigu nuolatinis Lietuvos gyventojas, neturintis…

    Kadangi 130 000 Lt > 100 000 Lt, tai privalu registruotis PVM
    mokėtoju, bet kadangi jis nėra PVM mokėtojas, tai moka PVM
    tik už viršytą sumą, tai yra už 30 000 Lt.

    Apmokestinamosios pajamos = 30 000 Lt.
    \begin{align*}
      PSD = 1350 Lt \\
      VSD = 4275 Lt \\
      GPM \t{„bazė“} = 24375 Lt \\
      MNPD = 2685 Lt \\
      GPM = (24375 - 2685) \cdot 0,05 = 1084,5 Lt \\
    \end{align*}

    Taikant 30\%, apmokestinamos pajamos = $91000$ Lt.
    \begin{align*}
      GPM = 4550 Lt \\
      VSD = 12967,50 Lt \\
      PSD = 4095 Lt \\
    \end{align*}

    Abiem atvejais PVM = …

  \item Apskaičiuokite visus mokėtinus 2012 m. mokesčius, kai nenuolatinis
    Lietuvos gyventojas, dirbantis UAB „X“ pagal darbo sutartį

    Bruto, tai ant popieriaus.

    Turint prekybinę individualią veiklą, mokesčius skaičiuojame taip
    pat, kaip ir paprastos individualios veiklos.

    \begin{itemize}
      \item tada, kai keičiama žemės paskirtis, tai yra traktuojama,
        kad siekia ekonominės naudos ir tie 900 000 Lt yra jo pajamos.
      \item kadangi nebuvo tęstinumo požymių ir tokių sandorių per
        metus buvo tik vienas, tai todėl tai nėra individuali veikla;
      \item senai pirkto automobilio pardavimas nėra siejamas su
        individualia veikla.
    \end{itemize}

    Taigi mokesčius reikėtų apskaičiuoti nuo 900 000 Lt.
    
    Taip pat turi mokėti PVM nuo 800 000 Lt.

    \begin{align*}
      PSD = 40500 Lt \\
      VSD = 128250 Lt \\
      GMP \t{„bazė“} = 731250 Lt \\
      GPM = 36562,5 Lt \\
    \end{align*}

    30\% atvejis:
    \begin{align*}
      GPM = 31500 Lt \\
      VSD = 89775 Lt \\
      PSD = 28350 Lt \\
    \end{align*}

    PVM = $(900 000 - 100 000)\cdot 0,21 = 168000 Lt$.

    Darbo užmokestis 2000 Lt „ant popieriaus“ per mėnesį, tai
    per metus 24000 Lt.
    \begin{align*}
      GPM (15\%) \t{prieš tai atėmus MNPD} = \\
      VSDF (30,98\%) = \\
      PSD (9\%) = \\
      GF (0,02\%) = \\
    \end{align*}

  \item Apskaičiuokite visus mokėtinus mokesčius, kia nuolatinis Lietuvos
    gyventojas, turintis II invalidumo grupę, …

    Jei būtų verslo liudijimas, tai nebūtų privaloma mokėti PSD, bet
    šiuo atveju jam yra privaloma.

    \begin{itemize}
      \item Atsiranda PVM prievolė, bet ji yra lygi 0.
      \item …
      \item baudų negalime traukti į sąnaudas;
      \item jei būtų sąnaudos, tai irgi negalėtume traukti į sąnaudas;
      \item …
    \end{itemize}

    Apmokestinamosios pajamos: 100 000 - 20 000 + 20 000 = 100000 Lt.
    \begin{align*}
      VSDF =14250 Lt \\
      PSDF = 4500 Lt \\
      GPM = 4062,5 Lt \\
      GPM \t{„bazė“} = 81250 Lt \\
    \end{align*}

    Jei taikom 30\%, apmokestinamos pajamos yra 120 000 Lt.
    \begin{align*}
      GPM = 4200 Lt \\
      PSD = 3780 Lt \\
      VSD = 11 970 Lt \\
    \end{align*}

  \item Apskaičiuokite visus 2011 m. mokėtinus mokesčius ir nurodykite
    mokėjimo terminus…

    \begin{itemize}
      \item sąnaudos;
      \item kadangi sugedo (tas pats, kaip žala) ir į sąnaudas netraukiame;
      \item …
      \item sąnaudos 22 000 Lt, pajamos: 21 000 Lt.
      \item Nuo lapkričio iki sausio mėnesio, dėl ligos nesivertė jokia
        prekybine veikla. Nesiskaičiuoja TODO: kas?
    \end{itemize}

    Pajamos: 111 000 Lt, sąnaudos 72 000 Lt. Apmokestinamosios
    pajamos: 39000 Lt.

    \begin{align*}
      PSD = \frac{39000}{2} \cdot 9\% = 1755 \\
      VSD (\t{fiksuoti už 10 mėnesį, sausio neskaičiuojame}) = 1800 \\
      GPM (\t{fiksuoti už 10 mėnesį, sausio neskaičiuojame}) = 1200 \\
    \end{align*}

    PVM:
    \begin{align*}
      \t{mokėtinas}: 111000 \cdot 0,21 = 23310 \\
      \t{gautinas}: 72 000 \cdot 0,21 = 15120 \\
      PVM = 8190
    \end{align*}

  \item Notarų veikla (antstoliai ir t.t.) yra PVM neapmokestinama
    veikla.

    Ant popieriaus 72 000 Lt.

    Apmokestinamosios pajamos 81 000 Lt.

    Susimokam VSD ir PSD (patikrinti ar yra lubos). Tada randam GPM
    15\%.

    Kitas variantas, taikant 30\% atvejį.

\end{enumerate}

Trys uždaviniai ir keli laisvi klausimai. Turėti skaičiuotuvą, reikia
žinoti mokesčių tarifus.
