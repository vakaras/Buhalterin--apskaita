\chapter{7 paskaita}

\setslideprefix{12 - ukines operacijos ir klases/skaidres.pdf:}

\section{Ūkinės operacijos fiksavimas}

\slide{2}

Kiekviena ūkinė operacija, susijusi su turto ar nuosavybės pasikeitimu,
turi būti užregistruota nustatyta tvarka ir atsispindėti metinėje
finansinėje atskaitomybėje, kuri pateikiama įmonės valdytojams bei
išoriniams naudotojams, norintiems žinoti, kokiu turtu disponuoja
įmonė ir kam šis turtas priklauso.

\section{Duomenys sąskaitose}

\slide{3}

Duomenys į sąskaitas atkeliauja iš pirminių dokumentų, todėl
būtų pravartu į jas įrašyti ne tik besikeisiančias turto ar nuosavybės
sumas, bet ir pažymėti, iš kokių pirminių dokumentų jos atliekamos.
Tai padėtų greičiau patikrinti iš kur yra paimti duomenys.

\section{„Lėktuvėlis“}

\slide{4}

\begin{PlaneTable}{Sąskaitos pavadinimas}
  xxx & yyy \\
\end{PlaneTable}

\section{Pokyčiai pinigų sąskaitoje}

\slide{5}

\begin{tabularx}{\tablewidth}[]{X | l | X}
  Data & Pinigų įplaukos arba išlaidos & Suma \\
  \hline
  2011-01-01 & Gauta & 800 \\
  2011-01-03 & Išmokėta & 500 \\
  2011-01-04 & Gauta & 1200 \\
  2011-01-05 & Išmokėta & 300 \\
  2011-01-10 & Išmokėta & 450 \\
  2011-01-11 & Gauta & 100 \\
  2011-01-15 & Gauta & 250 \\
  2011-01-20 & Išmokėta & 750 \\
\end{tabularx}

\section{Sąskaita „Pinigai“}

\slide{6}

\begin{tabularx}{\tablewidth}[]{X | X | X | X}
  Data & Debetas & Kreditas & Likutis \\
  \hline
  2011-01-01 & 800  &       & 800  \\
  2011-01-03 &      & 500   & 300  \\
  2011-01-04 & 1200 &       & 1500 \\
  2011-01-05 &      & 300   & 1200 \\
  2011-01-10 &      & 450   & 750  \\
  2011-01-11 & 100  &       & 850  \\
  2011-01-15 & 250  &       & 1100 \\
  2011-01-20 &      & 750   & 350  \\
\end{tabularx}

\slide{7}

\begin{tabularx}{\tablewidth}[]{X | X | X}
  Data & Gauta & Išleista  \\
  \hline
  2011-01-01 & 800  &     \\
  2011-01-03 &      & 500 \\
  2011-01-04 & 1200 &     \\
  2011-01-05 &      & 300 \\
  2011-01-10 &      & 450 \\
  2011-01-11 & 100  &     \\
  2011-01-15 & 250  &     \\
  2011-01-20 &      & 750 \\
  \hline
  Iš viso:   & 2350 & 2000 \\
\end{tabularx}

\section{Debeto sąvoka}

\slide{8}

Tai kairioji buhalterinių sąskaitų pusė, kurioje yra apskaitomi skolinimų
įsipareigojimai. Turto sąskaitose debeto įrašai reiškia apskaitomos
sumos padidėjimą, o įsipareigojimų – sumažėjimą.

DLKŽ + TŽŽ:
\begin{description}
  \item[dèbetas] – fin. sąskaitos įplaukos ir skolos bei išlaidos,
    įrašomos buhalterijos knygų kairiojoje pusėje;
  \item[dèbetas] (lot. debet – jis skolingas) – fin. buhalterinių sąskaitų
    kairioji pusė, rodanti skolininkų įsiskolinimo apskaitą; aktyvinėse
    sąskaitose debeto įrašai reiškia apskaitomų sumų padidėjimą,
    pasyvinėse – sumažėjimą.
\end{description}

\section{Kredito sąvoka}

\slide{9}

Tai dešinioji buhalterinių sąskaitų pusė, kurioje yra apskaitomi įmonės
įsipareigojimai.

DLKŽ + TŽŽ:
\begin{description}
  \item[krèditas] – sąskaitos įplaukos ir skolos bei išlaidos, įrašomos
    buhalterijos knygų dešiniojoje pusėje;
  \item[kredìtas] – fin. 1. prekių ar pinigų skolinimas, ppr. už palūkanas,
    paskola. 2. asignuojamos lėšos.
\end{description}

\section{„T“ formos sąskaitos (lėktuvėliai)}

\slide{10}

Apskaitininkai patys pasirenka sąskaitos formą ir joje atspindimos
informacijos kiekį, tačiau visose buhalterinėse sąskaitose būtinai
skiriamos debeto ir kredito skiltys.

\slide{11}

\begin{PlaneTable}{Turto sąskaitos (1,2 klasė)}
  Objekto padidėjimai & Objekto sumažėjimai \\
\end{PlaneTable}

1 klasė yra ilgalaikis turtas, 2 – trumpalaikis.

\begin{PlaneTable}{Nuosavybės sąskaitos (3,4 klasė)}
  Objekto sumažėjimai & Objekto padidėjimai \\
\end{PlaneTable}

\slide{12}

\begin{PlaneTable}{Pajamų sąskaitos (5 klasė)}
  Objekto sumažėjimai & Objekto padidėjimai \\
\end{PlaneTable}

\begin{PlaneTable}{Sąnaudų sąskaitos (6 klasė)}
  Objekto padidėjimai & Objekto sumažėjimai \\
\end{PlaneTable}

\section{Debeto ir kredito taisyklė}

\slide{13}

\begin{PlaneTable}{Taisyklė}
  Turto 1, 2 kl. (+) & Turto 1, 2 kl. (-) \\
  Sąnaudų 6 kl. (+) & Sąnaudų 6 kl. (-) \\
  Pajamų 5 kl. (-) & Pajamų 5 kl. (+) \\
  Nuosavybės 3, 4 kl. (-) & Nuosavybės 3, 4 kl. (+) \\
\end{PlaneTable}

\section{Sąskaitos}

\slide{14}

\begin{PlaneTable}{Turto sąskaita}
  Likutis pradžiai & \\
  Padidėjimas (+) & Sumažėjimas (-) \\
  Likutis pabaigai & \\
\end{PlaneTable}

\begin{PlaneTable}{Nuosavybės sąskaita}
  & Likutis pradžiai \\
  Sumažėjimas (-) & Padidėjimas (+) \\
  & Likutis pabaigai \\
\end{PlaneTable}

\begin{PlaneTable}{Sąnaudų sąskaita}
  Likutis pradžiai & \\
  Padidėjimas (+) & Sumažėjimas (-) \\
  Likutis pabaigai & \\
\end{PlaneTable}

\begin{PlaneTable}{Pajamų sąskaita}
  & Likutis pradžiai \\
  Sumažėjimas (-) & Padidėjimas (+) \\
  & Likutis pabaigai \\
\end{PlaneTable}

\section{Sąskaitų planas}

\slide{15}

Tai visų įmonėje naudojamų sąskaitų sąrašas, kuriame pateikiami visų
sąskaitų pavadinimai bei numeriai.

Pavyzdinis sąskaitų planas yra parodomasis, kuriuo nebūtina vadovautis,
bet jis gali būti taikomas kaip paaiškinamoji ūkinių faktų atspindėjimo
apskaitoje priemonė ir kuriuo remiantis įmonės pagal savo veiklos pobūdį
parengia savo individualų sąskaitų planą.

\section{Sąskaitų plano klasės}

\slide{16}

\slide{17}

\begin{description}
  \item[Pirmoji klasė.] Joje fiksuojamas ilgalaikis turtas.
  \item[Antroji klasė.] Joje fiksuojamas trumpalaikis turtas.
  \item[Trečioji klasė.] Joje fiksuojama savininkų nuosavybė arba įmonės
    įsipareigojimai savininkams bei gautos subsidijos ir dotacijos.
    Savininkų nuosavybę sudaro ne tik kapitalas, rezervai, mokesčių
    atidėjimai, bet ir nepaskirstytasis pelnas.
  \item[…] Ups…
  \item[Septintoji, aštuntoji, devintoji klasės] Šios klasės yra paliktos
    laisvos, nes jos yra skirtos įmonės vidinei apskaitai, o šios
    apskaitos įstatymai nereglamentuoja, jos vedimo tvarką kiekviena
    įmonė pasirenka savo nuožiūra.
  \item[Nulinė klasė.] Ši klasė palikta atspindėti įmonėje esantį turtą
    ar įsipareigojimus, kurie neturi tiesioginio ryšio su įmonės turtu
    bei nuosavybe.
\end{description}

\section{Buhalterinės apskaitos organizavimas}

\slide{18}

\slide{19}

\section{Pirminiai apskaitos dokumentai}

\slide{20}

\slide{21}

\chapter{Uždaviniai: Finansinė apskaita}

\slide{22}

\begin{tasks}

  \emph{Atsakymą žymėkite taip: apibraukite teisingo atsakymo numerį.}

  \begin{task}
    \begin{condition}
      Apskaitos požiūriu individualios (personalinės) įmonės nuosavybė:
      \begin{enumerate}
        \titem{a} negali būti didesnė nei 100 000 litų;
        \titem{b} negali būti mažesnė nei 100 000 litų;
        \titem{c} turi būti neatskirta nuo savininko nuosavybės;
        \titem{d} turi būti atskirta nuo savininko nuosavybės;
      \end{enumerate}
    \end{condition}
    \begin{solution}
      Teisingas atsakymas yra \tref{d}.
    \end{solution}
  \end{task}

  \begin{task}
    \begin{condition}
      Pavyzdinio sąskaitų plano ketvirtoji klasė skirta:
      \begin{enumerate}
        \titem{a} įmonės ilgalaikėms ir trumpalaikėms skoloms bei
          įsipareigojimams registruoti;
        \titem{b} <++>
        \titem{c} <++>
        \titem{d} <++>
      \end{enumerate}
    \end{condition}
    \begin{solution}
      Teisingas atsakymas yra \tref{a}.
    \end{solution}
  \end{task}

  \begin{task}
    \begin{condition}
      Chronologine tvarka registruose ūkinės operacijos fiksuojamos:
      \begin{enumerate}
        \titem{a} pagal datą;
        \titem{b} <++>
        \titem{c} <++>
        \titem{d} <++>
      \end{enumerate}
    \end{condition}
    \begin{solution}
      Teisingas atsakymas yra \tref{a}.
    \end{solution}
  \end{task}

  \begin{task}
    \begin{condition}
      Sistemine tvarka registruose ūkinės operacijos fiksuojamos:
      \begin{enumerate}
        \titem{a} pagal jų informacinį turinį;
        \titem{b} pagal jų reikšmingumo turinį;
        \titem{c} pagal jų ekonominį turinį;
        \titem{d} pagal jų naudingumo turinį;
        \titem{e} pagal jų gavimo datą.
      \end{enumerate}
    \end{condition}
    \begin{solution}
      Teisingas atsakymas yra \tref{c}.
    \end{solution}
  \end{task}

  \begin{task}
    \begin{condition}
      Ūkinių operacijų registravimas registruose, žurnaluose, t.y.
      jų registravimas suvestinėse lentelėse:
      \begin{enumerate}
        \titem{a} priskiriama suminei apskaitai;
        \titem{b} <++>
        \titem{c} <++>
        \titem{d} <++>
      \end{enumerate}
    \end{condition}
    \begin{solution}
      Teisingas atsakymas yra \tref{a}.
    \end{solution}
  \end{task}

  \begin{task}
    \begin{condition}
      Dvejybinis įrašas kaip apskaitos tvarkymo metodas – tai:
      \begin{enumerate}
        \titem{a} susijusių ūkinių operacijų fiksavimas;
        \titem{b} <++>
        \titem{c} <++>
        \titem{d} <++>
      \end{enumerate}
    \end{condition}
    \begin{solution}
      Teisingas atsakymas yra \tref{a}.
    \end{solution}
  \end{task}

  \begin{task}
    \begin{condition}
      Atskaitomybė – tai:
      \begin{enumerate}
        \titem{a} įmonės konkretaus laikotarpio ūkinės finansinės veiklos
          apibendrinančių rodiklių suvestinė;
        \titem{b} <++>
        \titem{c} <++>
        \titem{d} <++>
      \end{enumerate}
    \end{condition}
    \begin{solution}
      Teisingas atsakymas yra \tref{a}.
    \end{solution}
  \end{task}

  \begin{task}
    \begin{condition}
      Ūkinis įvykis – tai:
      \begin{enumerate}
        \titem{a} <++>
        \titem{b} nuo ūkinių subjektų nepriklausantys ūkiniai faktai,
          kurie turi įtakos turtui;
        \titem{c} <++>
        \titem{d} <++>
      \end{enumerate}
    \end{condition}
    \begin{solution}
      Teisingas atsakymas yra \tref{b}. Ūkinis subjektas ~ įmonė.
      Ūkinių įvykių pavyzdžiai: uraganas, potvynis.
    \end{solution}
  \end{task}

  \begin{task}
    \begin{condition}
      Ūkinė operacija:
      \begin{enumerate}
        \titem{a} <++>
        \titem{b} tikslinga veikla siekiant ūkinio rezultato;
        \titem{c} <++>
        \titem{d} <++>
      \end{enumerate}
    \end{condition}
    \begin{solution}
      Teisingas atsakymas yra \tref{b}.
    \end{solution}
  \end{task}

  \begin{task}
    \begin{condition}
      Kiekvieną faktą prieš fiksuojant apskaitoje reikia:
      \begin{enumerate}
        \titem{a} <++>
        \titem{b} identifikuoti, tai yra nustatyti, kada faktas įvyko,
          ar tai ūkinis įvykis, ar operacija;
        \titem{c} <++>
        \titem{d} <++>
      \end{enumerate}
    \end{condition}
    \begin{solution}
      Teisingas atsakymas yra \tref{b}.
    \end{solution}
  \end{task}
  
\end{tasks}

\begin{tasks}

  \emph{Atsakymą žymėkite taip: apibraukite teisingo atsakymo numerį.}

  \begin{task}
    \begin{condition}
      Turto ir sąnaudų sąskaitų apskaitos objektų pakitimai registruojami:
      \begin{enumerate}
        \titem{a} padidėjimas – debete, sumažėjimas – kredite;
        \titem{b} <++>
        \titem{c} <++>
        \titem{d} <++>
      \end{enumerate}
    \end{condition}
    \begin{solution}
      Teisingas atsakymas yra \tref{a}.
    \end{solution}
  \end{task}

  \begin{task}
    \begin{condition}
      Nuosavybės, įmonės įsipareigojimų ir pajamų sąskaitų
      apskaitos objektų pakitimai registruojami:
      \begin{enumerate}
        \titem{a} <++>
        \titem{b} sumažėjimas – debete, padidėjimas – kredite;
        \titem{c} <++>
        \titem{d} <++>
      \end{enumerate}
    \end{condition}
    \begin{solution}
      Teisingas atsakymas yra \tref{b}.
    \end{solution}
  \end{task}

  \begin{task}
    \begin{condition}
      Buhalterinės apskaitos informacijos tiekėjai yra:
      \begin{enumerate}
        \titem{a} dokumentai – juose fiksuojamos ūkinės operacijos
          ir ūkinei įvykiai;
        \titem{b} <++>
        \titem{c} <++>
        \titem{d} <++>
      \end{enumerate}
    \end{condition}
    \begin{solution}
      Teisingas atsakymas yra \tref{a}.
    \end{solution}
  \end{task}

  \begin{task}
    \begin{condition}
      Pagal LR Buhalterinės apskaitos įstatymo nuostatas visos
      ūkinės operacijos ir ūkiniai įvykiai turi būti:
      \begin{enumerate}
        \titem{a} <++>
        \titem{b} <++>
        \titem{c} pagrįsti apskaitos dokumentais, atskirais atvejais
          pagrindžiami su jais susijusių ūkinių operacijų ir ūkinių
          įvykių apskaitos dokumentais;
        \titem{d} <++>
      \end{enumerate}
    \end{condition}
    \begin{solution}
      Teisingas atsakymas yra \tref{c}.
    \end{solution}
  \end{task}

  \begin{task}
    \begin{condition}
      Pinigų apskaitos dokumentus:
      \begin{enumerate}
        \titem{a} taisyti draudžiama – jei padaryta klaida, pinigų
          apskaitos dokumentas anuliuojamas ir surašomas naujas;
        \titem{b} <++>
        \titem{c} <++>
        \titem{d} <++>
      \end{enumerate}
    \end{condition}
    \begin{solution}
      Teisingas atsakymas yra \tref{a}.
    \end{solution}
  \end{task}

  \begin{task}
    \begin{condition}
      Įplaukos, atsižvelgus į kaupimo principą, – tai:
      \begin{enumerate}
        \titem{a} <++>
        \titem{b} <++>
        \titem{c} <++>
        \titem{d} gautas arba gaunamas turtas (dažniausiai pinigai).
      \end{enumerate}
    \end{condition}
    \begin{solution}
      Teisingas atsakymas yra \tref{d}.
    \end{solution}
  \end{task}

  \begin{task}
    \begin{condition}
      Pajamos, atsižvelgus į kaupimo principą, – tai:
      \begin{enumerate}
        \titem{a} <++>
        \titem{b} <++>
        \titem{c} uždirbtas per ataskaitinį laikotarpį turtas
          (dažniausiai pinigai) nepriklausomai nuo to, ar jis jau gautas,
          ar dar ne;
        \titem{d} <++>
      \end{enumerate}
    \end{condition}
    \begin{solution}
      Teisingas atsakymas yra \tref{c}.
    \end{solution}
  \end{task}

  \begin{task}
    \begin{condition}
      Išlaidos, atsižvelgus į kaupimo principą, – tai:
      \begin{enumerate}
        \titem{a} <++>
        \titem{b} <++>
        \titem{c} išleisti arba išleidžiami pinigai (ar kitas
          turtas), kurie iškeičiami į kokį nors kitą turtą;
        \titem{d} <++>
      \end{enumerate}
    \end{condition}
    \begin{solution}
      Teisingas atsakymas yra \tref{c}.
    \end{solution}
  \end{task}

  \begin{task}
    \begin{condition}
      Sąnaudos, atsižvelgus į kaupimo principą, – tai:
      \begin{enumerate}
        \titem{a} sunaudotas turtas, darbo jėga ir kt., siekiant gauti
          tam tikrą naudą, t.y. kuri patirta uždirbant pajamas;
        \titem{b} <++>
        \titem{c} <++>
        \titem{d} <++>
      \end{enumerate}
    \end{condition}
    \begin{solution}
      Teisingas atsakymas yra \tref{a}.
    \end{solution}
  \end{task}

  \begin{task}
    \begin{condition}
      Galima lyginti:
      \begin{enumerate}
        \titem{a} <++>
        \titem{b} <++>
        \titem{c} <++>
        \titem{d} tik įplaukas su išlaidomis arba tik pajamas su
          sąnaudomis.
      \end{enumerate}
    \end{condition}
    \begin{solution}
      Teisingas atsakymas yra \tref{d}.
    \end{solution}
  \end{task}
  
\end{tasks}

\begin{tasks}

  \emph{Atsakymą žymėkite taip: apibraukite teisingo atsakymo numerį.}

  \begin{task}
    \begin{condition}
      Per vienerius metus gautinos sumos priskirtinos:
      \begin{enumerate}
        \titem{a} trumpalaikiam turtui;
        \titem{b} <++>
        \titem{c} <++>
        \titem{d} <++>
      \end{enumerate}
    \end{condition}
    \begin{solution}
      Teisingas atsakymas yra \tref{a}.
    \end{solution}
  \end{task}

  \begin{task}
    \begin{condition}
      Piniginiu turtu laikoma:
      \begin{enumerate}
        \titem{a} <++>
        \titem{b} <++>
        \titem{c} <++>
        \titem{d} visi atsakymai teisingi.
      \end{enumerate}
    \end{condition}
    \begin{solution}
      Teisingas atsakymas yra \tref{d}.
    \end{solution}
  \end{task}

  \begin{task}
    \begin{condition}
      Perdirbimo įmonėje vedamos šios atsargų apskaitos sąskaitos:
      \begin{enumerate}
        \titem{a} <++>
        \titem{b} <++>
        \titem{c} <++>
        \titem{d} visi atsakymai teisingi;
      \end{enumerate}
    \end{condition}
    \begin{solution}
      Teisingas atsakymas yra \tref{d}.
    \end{solution}
  \end{task}

  \begin{task}
    \begin{condition}
      Įrašai kasos knygoje:
      \begin{enumerate}
        \titem{a} <++>
        \titem{b} <++>
        \titem{c} <++>
        \titem{d} <++>
        \titem{e} teisingi atsakymai \tref{b} ir \tref{d};
        \titem{f} nėra teisingų atsakymų.
      \end{enumerate}
    \end{condition}
    \begin{solution}
      Teisingas atsakymas yra \tref{e}.
    \end{solution}
  \end{task}

  \begin{task}
    \begin{condition}
      Banko išrašuose pinigų pasikeitimai įmonės sąskaitose atspindi:
      \begin{enumerate}
        \titem{a} pinigų atsiskaitomoje sąskaitoje padidėjimai
          kredito skiltyje, o sumažėjimai – debeto skiltyje;
        \titem{b} <++>
        \titem{c} <++>
        \titem{d} <++>
      \end{enumerate}
    \end{condition}
    \begin{solution}
      Teisingas atsakymas yra \tref{a}.
    \end{solution}
  \end{task}

  \begin{task}
    \begin{condition}
      Visas ilgalaikis materialus turtas balanse turi būti fiksuojamas:
      \begin{enumerate}
        \titem{a} likvidumo didėjimo tvarka;
        \titem{b} <++>
        \titem{c} <++>
        \titem{d} <++>
      \end{enumerate}
    \end{condition}
    \begin{solution}
      Teisingas atsakymas yra \tref{a}.
    \end{solution}
  \end{task}

  \begin{task}
    \begin{condition}
      Apskaitoje visiškai nudėvėto, tačiau dar naudojamo įmonėje
      turto likutinė vertė:
      \begin{enumerate}
        \titem{a} turi būti lygi nuliui;
        \titem{b} <++>
        \titem{c} <++>
        \titem{d} <++>
      \end{enumerate}
    \end{condition}
    \begin{solution}
      Teisingas atsakymas yra \tref{a}.
    \end{solution}
  \end{task}

  \begin{task}
    \begin{condition}
      Ilgalaikiam turtui nepriklauso:
      \begin{enumerate}
        \titem{a} <++>
        \titem{b} <++>
        \titem{c} <++>
        \titem{d} nėra teisingo atsakymo.
      \end{enumerate}
    \end{condition}
    \begin{solution}
      Teisingas atsakymas yra \tref{d}.
    \end{solution}
  \end{task}

  \begin{task}
    \begin{condition}
      Ilgalaikis nematerialusis turtas:
      \begin{enumerate}
        \titem{a} <++>
        \titem{b} amortizuojamas;
        \titem{c} <++>
        \titem{d} <++>
        \titem{e} visi atsakymai neteisingi.
      \end{enumerate}
    \end{condition}
    \begin{solution}
      Teisingas atsakymas yra \tref{b}.
    \end{solution}
  \end{task}

  \begin{task}
    \begin{condition}
      Pirkėjų skolos, kurios bus grąžintos vėliau nei po vienerių
      metų, priklauso:
      \begin{enumerate}
        \titem{a} <++>
        \titem{b} <++>
        \titem{c} <++>
        \titem{d} po vienerių metų gautinoms sumoms.
      \end{enumerate}
    \end{condition}
    \begin{solution}
      Teisingas atsakymas yra \tref{d}.
    \end{solution}
  \end{task}
  
\end{tasks}

\begin{tasks}

  \emph{Atsakymą žymėkite taip: apibraukite teisingo atsakymo numerį.}

  \begin{task}
    \begin{condition}
      Balansas – finansinė ataskaita, kurioje:
      \begin{enumerate}
        \titem{a} <++>
        \titem{b} Parodomas visas įmonės turtas, nuosavas kapitalas ir
          įsipareigojimai paskutinę ataskaitinio laikotarpio dieną;
        \titem{c} <++>
        \titem{d} <++>
      \end{enumerate}
    \end{condition}
    \begin{solution}
      Teisingas atsakymas yra \tref{b}.
    \end{solution}
  \end{task}

  \begin{task}
    \begin{condition}
      Pelno (nuostolių) ataskaita – finansinė ataskaita, kurioje:
      \begin{enumerate}
        \titem{a} nurodomos visos per ataskaitinį laikotarpį įmonės
          uždirbtos pajamos, patirtos sąnaudos ir gauti veiklos
          rezultatai;
        \titem{b} <++>
        \titem{c} <++>
        \titem{d} <++>
      \end{enumerate}
    \end{condition}
    \begin{solution}
      Teisingas atsakymas yra \tref{a}.
    \end{solution}
  \end{task}

  \begin{task}
    \begin{condition}
      Sudaroma finansinė atskaitomybė, norint suteikti informaciją:
      \begin{enumerate}
        \titem{a} <++>
        \titem{b} <++>
        \titem{c} <++>
        \titem{d} apie įmonės finansinę būklę, veiklos rezultatus, tai yra
          apie turtą, nuosavą kapitalą, įsipareigjojimus, pajamas ir
          sąnaudas, pinigų srautus.
      \end{enumerate}
    \end{condition}
    \begin{solution}
      Teisingas atsakymas yra \tref{d}.
    \end{solution}
  \end{task}

  \begin{task}
    \begin{condition}
      Pilną finansinę atskaitomybę sudaro:
      \begin{enumerate}
        \titem{a} balansas, pelno (nuostolių) ataskaita, nuosavo
          kapitalo pokyčių ataskaita, pinigų srautų ataskaita;
        \titem{b} <++>
        \titem{c} <++>
        \titem{d} <++>
      \end{enumerate}
    \end{condition}
    \begin{solution}
      Teisingas atsakymas yra \tref{a}.
    \end{solution}
  \end{task}

  \begin{task}
    \begin{condition}
      Trumpą finansinę atskaitomybę sudaro:
      \begin{enumerate}
        \titem{a} <++>
        \titem{b} <++>
        \titem{c} balansas, pelno (nuostolių) ataskaita, nuosavo
          kapitalo pokyčių ataskaita, aiškinamasis raštas;
        \titem{d} <++>
      \end{enumerate}
    \end{condition}
    \begin{solution}
      Teisingas atsakymas yra \tref{c}.
    \end{solution}
  \end{task}

  \begin{task}
    \begin{condition}
      Balanse pateikiama informacija apie:
      \begin{enumerate}
        \titem{a} <++>
        \titem{b} turtą, įsipareigojimus ir nuosavą kapitalą –
          šie elementai apibūdina įmonės finansinę būklę;
        \titem{c} <++>
        \titem{d} <++>
      \end{enumerate}
    \end{condition}
    \begin{solution}
      Teisingas atsakymas yra \tref{b}.
    \end{solution}
  \end{task}

  \begin{task}
    \begin{condition}
      Pelno (nuostolių) ataskaitos elementai yra:
      \begin{enumerate}
        \titem{a} <++>
        \titem{b} <++>
        \titem{c} pajamos ir sąnaudos – šie elementai apibūdina įmonės
          veiklos rezultatus;
        \titem{d} <++>
      \end{enumerate}
    \end{condition}
    \begin{solution}
      Teisingas atsakymas yra \tref{c}.
    \end{solution}
  \end{task}

  \begin{task}
    \begin{condition}
      Apskaitos dokumentus surašę ir pasirašę asmenys atsako:
      \begin{enumerate}
        \titem{a} <++>
        \titem{b} <++>
        \titem{c} <++>
        \titem{d} už apskaitos dokumentų surašymo laiku ir teisingai,
          už juose esančių duomenų tikrumą ir ūkini̇ų operacijų
          teisėtumą.
      \end{enumerate}
    \end{condition}
    \begin{solution}
      Teisingas atsakymas yra \tref{d}.
    \end{solution}
  \end{task}

  \begin{task}
    \begin{condition}
      Klaida – tai:
      \begin{enumerate}
        \titem{a} <++>
        \titem{b} <++>
        \titem{c} <++>
        \titem{d} netikslumas, kuris atsiranda dėl neteisingo skaičiavimo,
          netikslaus apskaitos metodo taikymo, neteisingo ūkinės
          operacijos ar įvykio registravimo, dėl apgaulės ar
          apsirikimo.
      \end{enumerate}
    \end{condition}
    \begin{solution}
      Teisingas atsakymas yra \tref{d}.
    \end{solution}
  \end{task}

  \begin{task}
    \begin{condition}
      Esminė klaida – tai:
      \begin{enumerate}
        \titem{a} <++>
        \titem{b} klaida, kuri išaiškėja einamuoju ataskaitiniu
          laikotarpiu ir dėl kurios praėjusio ar kelių praėjusių
          laikotarpių finansinė atskaitomybė teisingai neatspindi
          įmonės finansinės būklės ir veiklos rezultatų;
        \titem{c} <++>
        \titem{d} <++>
      \end{enumerate}
    \end{condition}
    \begin{solution}
      Teisingas atsakymas yra \tref{b}.
    \end{solution}
  \end{task}

\end{tasks}
