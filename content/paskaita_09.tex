\chapter{Ūkinės operacijos fiksavimas}

Debetas Sąskaitos pavadinimas Kreditas

\section{Pokyčiai pinigų sąskaitoje}

\section{Sąskaita „Pinigai“}

\section{Debeto sąvoka}

\section{Kredito sąvoka}

\section{„T“ formos sąskaitos (lėktuvėliai)}

1, 2 klasės. (1 klasė ilgalaikis, o 2– trumpalaikis).

D+

K-

3, 4 klasės:
D- K+

5 klasė (pajamų klasė)
K+

6 klasė (sąnaudų klasė)
D+

\section{Sąskaitų planas}

\section{Sąskaitų plano klasės}

\section{Buhalterinės apskaitos organizavimas}

\section{Pirminiai apskaitos dokumentai}

\chapter{Uždaviniai}

1. Apskaitos požiūriu individualios (personalinės įmonės nuosavybė):
d) turi būti atskirta nuo savininko nuosavybės.

2. Pavyzdinio sąskaitų plano ketvirtoji klasė skirta:
a) įmonės …

3. Chronologine tvarka …
pagal datą

4. Sistemine tvarka registruose ūkinės operacijos fiksuojamos:
c Pagal jų ekonominį turinį

5. Ūkinių operacijų registravimas registruose, žurnaluose, t.y. jų
registravimas suvestinėse lentelėse:
a) Priskiriama suminei apskaitai;

6. Dvejybinis įrašas kaip apskaitos tvarkymo metodas – tai:
a) Susijusių ūkinių operacijų fiksavimas

7. Atskaitomybė – tai:
a

8. Ūkinis įvykis – tai:
ūkinis subjektas ~ įmonė

b) pavyzdžiui, uraganas, potvynis ir pan.

9. Ūkinė operacija:
b) Tikslinga veikla siekiant ūkinio rezultato;

10. Kiekvieną faktą prieš fiksuojant apskaitoje reikia:
b) Identifikuoti, t.y. nustatyti, kada faktas įvyko, ar …


1. Turto ir sąnaudų sąskaitų apskaitos objektų pakitimai registruojami:
a) padidėjimas – debete, sumažėjimas – kredite

2. Nuosavybės, įmonės įsipareigojimų ir pajamų sąskaitų apskaitos objektų
pakitimai registruojami:
b) sumažėjimas – debete, padidėjimas – kredite

3. Buhalterinės apskaitos informacijos tiekėjai yra:
a) Dokumentai. Juose fiksuojamos ūkinės operacijos ir ūkinei įvykiai

4. Pagal LR Buhalterinės apskaitos įstatymo nuostatas visos ūkinės
operacijos ir ūkiniai įvykiai turi būti:
c) Pagrįsti apskaitos dokumentais. Atskirais atvejais pagrindžiami su jais
susijusių…

5. Pinigų apskaitos dokumentus:
a) Taisyti draudžiama. Jei padaryta klaida, pinigų apskaitos dokumentas
anuliuojamas ir surašomas naujas;

6. Įplaukos, atsižvelgus į kaupimo principą, – tai:
d) gautas arba gaunamas turtas (dažniausiai pinigai)

7. Pajamos, atsižvelgus į kaupimo principą, – tai:
c) Uždirbtas per ataskaitinį laikotarpį turtas (dažniausiai pinigai)
nepriklausomai nuo to, ar jis jau gautas, ar dar ne;

8. Išlaidos, atsižvelgus į kaupimo principą, – tai:
c) Išleisti arba išleidžiami pinigai (ar kitas turtas), kurie iškeičiami
į kokį nors kitą turtą;

9. Sąnaudos, atsižvelgus į kaupimo principą, – tai:
a) Sunaudotas turtas, darbo jėga ir kt., siekiant gauti tam tikrą naudą,
t.y., kuri patirta uždirbant pajamas.

10. Galima lyginti:
d)  Tik įplaukas su išlaidomis arba tik pajamas su sąnaudomis.

1. Per
a)

2. 
d

3. Perdirbimo įmonėje vedamos šios atsargų apskaitos sąskaitos:
d

4. Įrašai kasos knygoje:
e  teisingi b ir d

5. Banko išrašuose pinigų pasikeitimai įmonės sąskaitose atspindi:
a) pinigų atsiskaitomoje sąskaitoje padidėjimai kredito skiltyje, o
sumažėjimai – debeto skiltyje

6. Visas ilgalaikis materialus turtas balanse turi būti fiksuojamas:
a) likvidumo didėjimo tvarka;

7. Apskaitoje visiškai nudėvėto, tačiau dar naudojamo įmonėje turto
likutinė vertė:
a) turi būti lygi nuliui

8. Ilgalaikiam turtui nepriklauso:
d) nėra teisingo atsakymo

9. Ilgalaikis nematerialusis turtas:
b) amortizuojamas;

10. Pirkėjų skolos, kurios bus grąžintos vėliau nei po vienerių metų,
priklauso:
d) po vienerių metų gautinoms sumoms


1. Balansas – finansinė ataskaita, kurioje:
b) Parodomas visas įmonės turtas, nuosavas kapitalas ir
įsipareigojimai paskutinę…

2. Pelno (nuostolių) ataskaita – finansinė ataskaita, kurioje:
a) Nurodomos

3. Sudaroma finansinė atskaitomybė, norint suteikti informaciją:
d

4. Pilną finansinę atskaitomybę sudaro:
a

5. Trumpą finansinę atskaitomybę sudaro:
c

6. Balanse pateikiama informacija apie:
b

7. Pelno (nuostolių) ataskaitos elementai yra:
?

8. Apskaitos dokumentus surašę
d

9. Klaida – tai:
d

10. Esminė klaida – tai:
