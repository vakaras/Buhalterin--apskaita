\chapter{7 paskaita}

\setslideprefix{12 - ukines operacijos ir klases/skaidres.pdf:}

\section{Ūkinės operacijos fiksavimas}

\slide{2}

Kiekviena ūkinė operacija, susijusi su turto ar nuosavybės pasikeitimu,
turi būti užregistruota nustatyta tvarka ir atsispindėti metinėje
finansinėje atskaitomybėje, kuri pateikiama įmonės valdytojams bei
išoriniams naudotojams, norintiems žinoti, kokiu turtu disponuoja
įmonė ir kam šis turtas priklauso.

\section{Duomenys sąskaitose}

\slide{3}

Duomenys į sąskaitas atkeliauja iš pirminių dokumentų, todėl
būtų pravartu į jas įrašyti ne tik besikeisiančias turto ar nuosavybės
sumas, bet ir pažymėti, iš kokių pirminių dokumentų jos atliekamos.
Tai padėtų greičiau patikrinti iš kur yra paimti duomenys.

\section{„Lėktuvėlis“}

\slide{4}

\begin{PlaneTable}{Sąskaitos pavadinimas}
  xxx & yyy \\
\end{PlaneTable}

\section{Pokyčiai pinigų sąskaitoje}

\slide{5}

\begin{tabularx}{\tablewidth}[]{X | l | X}
  Data & Pinigų įplaukos arba išlaidos & Suma \\
  \hline
  2011-01-01 & Gauta & 800 \\
  2011-01-03 & Išmokėta & 500 \\
  2011-01-04 & Gauta & 1200 \\
  2011-01-05 & Išmokėta & 300 \\
  2011-01-10 & Išmokėta & 450 \\
  2011-01-11 & Gauta & 100 \\
  2011-01-15 & Gauta & 250 \\
  2011-01-20 & Išmokėta & 750 \\
\end{tabularx}

\section{Sąskaita „Pinigai“}

\slide{6}

\begin{tabularx}{\tablewidth}[]{X | X | X | X}
  Data & Debetas & Kreditas & Likutis \\
  \hline
  2011-01-01 & 800  &       & 800  \\
  2011-01-03 &      & 500   & 300  \\
  2011-01-04 & 1200 &       & 1500 \\
  2011-01-05 &      & 300   & 1200 \\
  2011-01-10 &      & 450   & 750  \\
  2011-01-11 & 100  &       & 850  \\
  2011-01-15 & 250  &       & 1100 \\
  2011-01-20 &      & 750   & 350  \\
\end{tabularx}

\slide{7}

\begin{tabularx}{\tablewidth}[]{X | X | X}
  Data & Gauta & Išleista  \\
  \hline
  2011-01-01 & 800  &     \\
  2011-01-03 &      & 500 \\
  2011-01-04 & 1200 &     \\
  2011-01-05 &      & 300 \\
  2011-01-10 &      & 450 \\
  2011-01-11 & 100  &     \\
  2011-01-15 & 250  &     \\
  2011-01-20 &      & 750 \\
  \hline
  Iš viso:   & 2350 & 2000 \\
\end{tabularx}

\section{Debeto sąvoka}

\slide{8}

Tai kairioji buhalterinių sąskaitų pusė, kurioje yra apskaitomi skolinimų
įsipareigojimai. Turto sąskaitose debeto įrašai reiškia apskaitomos
sumos padidėjimą, o įsipareigojimų – sumažėjimą.

DLKŽ + TŽŽ:
\begin{description}
  \item[dèbetas] – fin. sąskaitos įplaukos ir skolos bei išlaidos,
    įrašomos buhalterijos knygų kairiojoje pusėje;
  \item[dèbetas] (lot. debet – jis skolingas) – fin. buhalterinių sąskaitų
    kairioji pusė, rodanti skolininkų įsiskolinimo apskaitą; aktyvinėse
    sąskaitose debeto įrašai reiškia apskaitomų sumų padidėjimą,
    pasyvinėse – sumažėjimą.
\end{description}

\section{Kredito sąvoka}

\slide{9}

Tai dešinioji buhalterinių sąskaitų pusė, kurioje yra apskaitomi įmonės
įsipareigojimai.

DLKŽ + TŽŽ:
\begin{description}
  \item[krèditas] – sąskaitos įplaukos ir skolos bei išlaidos, įrašomos
    buhalterijos knygų dešiniojoje pusėje;
  \item[kredìtas] – fin. 1. prekių ar pinigų skolinimas, ppr. už palūkanas,
    paskola. 2. asignuojamos lėšos.
\end{description}

\section{„T“ formos sąskaitos (lėktuvėliai)}

\slide{10}

Apskaitininkai patys pasirenka sąskaitos formą ir joje atspindimos
informacijos kiekį, tačiau visose buhalterinėse sąskaitose būtinai
skiriamos debeto ir kredito skiltys.

\slide{11}

\begin{PlaneTable}{Turto sąskaitos (1,2 klasė)}
  Objekto padidėjimai & Objekto sumažėjimai \\
\end{PlaneTable}

1 klasė yra ilgalaikis turtas, 2 – trumpalaikis.

\begin{PlaneTable}{Nuosavybės sąskaitos (3,4 klasė)}
  Objekto sumažėjimai & Objekto padidėjimai \\
\end{PlaneTable}

\slide{12}

\begin{PlaneTable}{Pajamų sąskaitos (5 klasė)}
  Objekto sumažėjimai & Objekto padidėjimai \\
\end{PlaneTable}

\begin{PlaneTable}{Sąnaudų sąskaitos (6 klasė)}
  Objekto padidėjimai & Objekto sumažėjimai \\
\end{PlaneTable}

\section{Debeto ir kredito taisyklė}

\slide{13}

\begin{PlaneTable}{Taisyklė}
  Turto 1, 2 kl. (+) & Turto 1, 2 kl. (-) \\
  Sąnaudų 6 kl. (+) & Sąnaudų 6 kl. (-) \\
  Pajamų 5 kl. (-) & Pajamų 5 kl. (+) \\
  Nuosavybės 3, 4 kl. (-) & Nuosavybės 3, 4 kl. (+) \\
\end{PlaneTable}

\section{Sąskaitos}

\slide{14}

\begin{PlaneTable}{Turto sąskaita}
  Likutis pradžiai & \\
  Padidėjimas (+) & Sumažėjimas (-) \\
  Likutis pabaigai & \\
\end{PlaneTable}

\begin{PlaneTable}{Nuosavybės sąskaita}
  & Likutis pradžiai \\
  Sumažėjimas (-) & Padidėjimas (+) \\
  & Likutis pabaigai \\
\end{PlaneTable}

\begin{PlaneTable}{Sąnaudų sąskaita}
  Likutis pradžiai & \\
  Padidėjimas (+) & Sumažėjimas (-) \\
  Likutis pabaigai & \\
\end{PlaneTable}

\begin{PlaneTable}{Pajamų sąskaita}
  & Likutis pradžiai \\
  Sumažėjimas (-) & Padidėjimas (+) \\
  & Likutis pabaigai \\
\end{PlaneTable}

\section{Sąskaitų planas}

\slide{15}

Tai visų įmonėje naudojamų sąskaitų sąrašas, kuriame pateikiami visų
sąskaitų pavadinimai bei numeriai.

Pavyzdinis sąskaitų planas yra parodomasis, kuriuo nebūtina vadovautis,
bet jis gali būti taikomas kaip paaiškinamoji ūkinių faktų atspindėjimo
apskaitoje priemonė ir kuriuo remiantis įmonės pagal savo veiklos pobūdį
parengia savo individualų sąskaitų planą.

\section{Sąskaitų plano klasės}

\slide{16}

\slide{17}

\begin{description}
  \item[Pirmoji klasė.] Joje fiksuojamas ilgalaikis turtas.
  \item[Antroji klasė.] Joje fiksuojamas trumpalaikis turtas.
  \item[Trečioji klasė.] Joje fiksuojama savininkų nuosavybė arba įmonės
    įsipareigojimai savininkams bei gautos subsidijos ir dotacijos.
    Savininkų nuosavybę sudaro ne tik kapitalas, rezervai, mokesčių
    atidėjimai, bet ir nepaskirstytasis pelnas.
  \item[…] Ups…
  \item[Septintoji, aštuntoji, devintoji klasės] Šios klasės yra paliktos
    laisvos, nes jos yra skirtos įmonės vidinei apskaitai, o šios
    apskaitos įstatymai nereglamentuoja, jos vedimo tvarką kiekviena
    įmonė pasirenka savo nuožiūra.
  \item[Nulinė klasė.] Ši klasė palikta atspindėti įmonėje esantį turtą
    ar įsipareigojimus, kurie neturi tiesioginio ryšio su įmonės turtu
    bei nuosavybe.
\end{description}

\section{Buhalterinės apskaitos organizavimas}

\slide{18}

\slide{19}

\section{Pirminiai apskaitos dokumentai}

\slide{20}

\slide{21}

\chapter{Uždaviniai}

\slide{22}

1. Apskaitos požiūriu individualios (personalinės įmonės nuosavybė):
d) turi būti atskirta nuo savininko nuosavybės.

2. Pavyzdinio sąskaitų plano ketvirtoji klasė skirta:
a) įmonės …

3. Chronologine tvarka …
pagal datą

4. Sistemine tvarka registruose ūkinės operacijos fiksuojamos:
c Pagal jų ekonominį turinį

5. Ūkinių operacijų registravimas registruose, žurnaluose, t.y. jų
registravimas suvestinėse lentelėse:
a) Priskiriama suminei apskaitai;

6. Dvejybinis įrašas kaip apskaitos tvarkymo metodas – tai:
a) Susijusių ūkinių operacijų fiksavimas

7. Atskaitomybė – tai:
a

8. Ūkinis įvykis – tai:
ūkinis subjektas ~ įmonė

b) pavyzdžiui, uraganas, potvynis ir pan.

9. Ūkinė operacija:
b) Tikslinga veikla siekiant ūkinio rezultato;

10. Kiekvieną faktą prieš fiksuojant apskaitoje reikia:
b) Identifikuoti, t.y. nustatyti, kada faktas įvyko, ar …


1. Turto ir sąnaudų sąskaitų apskaitos objektų pakitimai registruojami:
a) padidėjimas – debete, sumažėjimas – kredite

2. Nuosavybės, įmonės įsipareigojimų ir pajamų sąskaitų apskaitos objektų
pakitimai registruojami:
b) sumažėjimas – debete, padidėjimas – kredite

3. Buhalterinės apskaitos informacijos tiekėjai yra:
a) Dokumentai. Juose fiksuojamos ūkinės operacijos ir ūkinei įvykiai

4. Pagal LR Buhalterinės apskaitos įstatymo nuostatas visos ūkinės
operacijos ir ūkiniai įvykiai turi būti:
c) Pagrįsti apskaitos dokumentais. Atskirais atvejais pagrindžiami su jais
susijusių…

5. Pinigų apskaitos dokumentus:
a) Taisyti draudžiama. Jei padaryta klaida, pinigų apskaitos dokumentas
anuliuojamas ir surašomas naujas;

6. Įplaukos, atsižvelgus į kaupimo principą, – tai:
d) gautas arba gaunamas turtas (dažniausiai pinigai)

7. Pajamos, atsižvelgus į kaupimo principą, – tai:
c) Uždirbtas per ataskaitinį laikotarpį turtas (dažniausiai pinigai)
nepriklausomai nuo to, ar jis jau gautas, ar dar ne;

8. Išlaidos, atsižvelgus į kaupimo principą, – tai:
c) Išleisti arba išleidžiami pinigai (ar kitas turtas), kurie iškeičiami
į kokį nors kitą turtą;

9. Sąnaudos, atsižvelgus į kaupimo principą, – tai:
a) Sunaudotas turtas, darbo jėga ir kt., siekiant gauti tam tikrą naudą,
t.y., kuri patirta uždirbant pajamas.

10. Galima lyginti:
d)  Tik įplaukas su išlaidomis arba tik pajamas su sąnaudomis.

1. Per
a)

2. 
d

3. Perdirbimo įmonėje vedamos šios atsargų apskaitos sąskaitos:
d

4. Įrašai kasos knygoje:
e  teisingi b ir d

5. Banko išrašuose pinigų pasikeitimai įmonės sąskaitose atspindi:
a) pinigų atsiskaitomoje sąskaitoje padidėjimai kredito skiltyje, o
sumažėjimai – debeto skiltyje

6. Visas ilgalaikis materialus turtas balanse turi būti fiksuojamas:
a) likvidumo didėjimo tvarka;

7. Apskaitoje visiškai nudėvėto, tačiau dar naudojamo įmonėje turto
likutinė vertė:
a) turi būti lygi nuliui

8. Ilgalaikiam turtui nepriklauso:
d) nėra teisingo atsakymo

9. Ilgalaikis nematerialusis turtas:
b) amortizuojamas;

10. Pirkėjų skolos, kurios bus grąžintos vėliau nei po vienerių metų,
priklauso:
d) po vienerių metų gautinoms sumoms


1. Balansas – finansinė ataskaita, kurioje:
b) Parodomas visas įmonės turtas, nuosavas kapitalas ir
įsipareigojimai paskutinę…

2. Pelno (nuostolių) ataskaita – finansinė ataskaita, kurioje:
a) Nurodomos

3. Sudaroma finansinė atskaitomybė, norint suteikti informaciją:
d

4. Pilną finansinę atskaitomybę sudaro:
a

5. Trumpą finansinę atskaitomybę sudaro:
c

6. Balanse pateikiama informacija apie:
b

7. Pelno (nuostolių) ataskaitos elementai yra:
?

8. Apskaitos dokumentus surašę
d

9. Klaida – tai:
d

10. Esminė klaida – tai:
