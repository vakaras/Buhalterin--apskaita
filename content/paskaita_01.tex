\chapter{Autorinės sutartys}

\section{Autorinė sutartis}

Pagal autorinę sutartį viena šalis (autorius ar jo teisių turėtojas)
perduoda arba suteikia autorių turtines teises į literatūros, mokslo
…

Ar verta rizikuoti nesudaryti rašytinių autorinių sutarčių su autoriais,
kurių autoriniai darbai spausdinami periodikoje?

Autorių teisių objektai:
\begin{itemize}
  \item knygos, brošiūros, straipsniai, dienoraščiai ir kiti literatūros
    kūriniai;
  \item kalbos, paskaitos, pamokslai ir kiti žodiniai kūriniai;
  \item rašytini ir žodiniai mokslo kūriniai;
  \item dramos, muzikiniai dramos, pantonimos;
  \item …
\end{itemize}
Taip pat ir:
\begin{itemize}
  \item išvestiniai kūriniai;
  \item kūrinių rinkiniai ar duomenų rinkiniai, duomenų bazės,
    autoriaus intelektinės kūrybos rezultatai;
  \item teisės aktų ar administracinių, teisinių, norminio pobūdžio
    dokumentų neoficialūs vertinimai.
\end{itemize}

Autorius pateikė autorinį darbą publikavimui interneto puslapyje.
Kam priklauso to kūrinio autorinės teisės, jei autoriui buvo už
tai sumokėti pinigai? O jei dar buvo nespėta sumokėti? Ir ar gali
autorius pareikalauti nutraukti publikavimą tame tinklalapyje?

Lietuvoje autorių teisių objektais nelaikomi:
\begin{itemize}
  \item idėjos, procedūros, procesai, sistemos, veiklos metodai,
    koncepcijos, principai, atradimai ar atskiri duomenys;
  \item teisės aktai, oficialūs administracinio, teisinio ar norminio
    pobūdžio dokumentai, taip pat jų oficialūs vertimai;
  \item oficialūs valstybės simboliai ir ženklai;
  \item oficialiai įregistruoti teisės aktų projektai;
  \item įprastinio pobūdžio informaciniai pranešimai apie įvykius;
  \item liaudies meno kūriniai.
\end{itemize}

Autorių teisės į kompiuterių programas:
\begin{itemize}
  \item kompiuterių programos autorius yra fizinis asmuo ar fizinių
    asmenų grupė sukūrę programą;
  \item kompiuterių programa saugoma pagal šį įstatymą, jeigu ji yra
    originali;
  \item turtinės autorių teisės į kompiuterių programą, kurią sukūrė
    darbuotojas atlikdamas savo tarnybines pareigas ar darbo funkcijas,
    …
\end{itemize}

Duomenų bazių gamintojų teisės:
\begin{itemize}
  \item turi teisę uždrausti: perkelti visą duomenų bazės turinį ar esminę
    jos dalį į kitą laikmeną; bet kokiu būdu padaryti viešai
    prieinamą visą duomenų bazės turinį ar esminę jo dalį;
  \item teisės gali būti perduodamos kitiems asmenims pagal sutartį;
\end{itemize}

Duomenų bazių gamintojų teisių apribojimai:
\begin{itemize}
  \item kai duomenų bazė…
  \item neleidžiama daryti pakart…
\end{itemize}

Autorinę sutartį gali sudaryti: autorius (arba jo teisių perėmėjas) ir
kūrinio naudotojas. Jeigu kūrinį sukūrė du ir daugiau fizinių
asmenų bendru…

Ką daryti, jei autoriaus darbai spausdinami leidinyje nenurodant jo tikrosios pavardės (su pseudonimu) arba apskritai nenurodant autorystės, o tiesiog laikant, kad tai leidinio medžiaga? Autorius su tokia sąlyga sutinka…

Autorinėse sutartyse turi būti nurodyta:
\begin{itemize}
  \item kūrinio pavadinimas;
  \item perduodamos ar suteikiamos autorių turtinės teisės;
  \item sutarties galiojimo teritorija bei terminas;
  \item autorinio atlyginimo dydis, jo mokėjimo tvarka ir terminai;
  \item šalių ginčų sprendimo tvarka ir atsakomybė.
\end{itemize}
Negalima minėti sąlygų dėl darbo vietos ar laiko. (Nes tada būtų jau
darbo sutartis.)

Autorinis atlyginimas:
\begin{itemize}
  \item už kūrybą, tai yra kūrinių kūrimą ir turtinių teisių į savo
    sukurtus kūrinius perleidimą;
  \item už honorarą – atlyginimą už autorine licencine sutartimi
    suteikta teisę panaudoti kūrinį.
\end{itemize}

Autorinio atlyginimo apmokestinimas priklauso nuo:
\begin{itemize}
  \item ar autorius su autorinį atlyginimą išmokančiu asmeniu ar kitu
    asmeniu yra susijęs darbo teisiniais santykiais ar ne, ar 
    autorius turi meno kūrėjo statusą;
  \item ar autorius yra įregistravęs ir vykdo atitinkamą individualią
    veiklą;
  \item ar autoriui mokamas autorinis atlyginimas už kūrinių panaudojimą
    pagal…
\end{itemize}<++>

Dirbančio autoriaus sąvoka:
\begin{itemize}
  \item asmenys, dirbantys pagal darbo sutartys;
  \item kandidatai į notarus;
  \item valstybės politikai, teisėjai, valstybės pareigūnai, valstybės
    tarnautojai;
  \item asmenys, atlyginti…
\end{itemize}

Dirbantiems autoriams (tarifus galima rasti SODROS svetainėje):
\begin{itemize}
  \item 15 \% GPM;
  \item 9 \% PSDF;
  \item Darbdavys dar moka: 30,98\% VSDF.
\end{itemize}

Nedirbantiems autoriams:
\begin{itemize}
  \item 15 \% GPM;
  \item 9 \% PSDF (nuo 50 \% atlygio)<++>;
  \item Darbdavys dar moka: 29,7\% VSDF (nuo 50 \% atlygio).
\end{itemize}

Maksimali įmokomis į fondus apmokestinamoji suma 48 LRV dydžiai.

Darbdavys išmokėdamas pajamas pagal autorinė sutartį nuo priskaičiuotų
pajamų tą pačią dieną privalo apskaičiuoti ir sumokėti įmokas į
VSDF ir PSDF.

\begin{exmp}
  Tarkime ant popieriaus gauna 7000Lt.

  Jei darbuotojas turi darbo sutartį, jis sumoka:
  \begin{itemize}
    \item GPM = 7000 $\cdot$ 0,15 Lt,
    \item PSD = 7000 $\cdot$ 0,09 Lt.
  \end{itemize}
  Darbdavys sumoka: VSD = 7000 $\cdot$ 0,3098 Lt.

  Jei darbuotojas neturi darbo sutarties, jis sumoka:
  \begin{itemize}
    \item GPM = 7000 $\cdot$ 0,15 Lt
    \item PSD = 7000 $\cdot$ 0,5 $\cdot$ 0,09 Lt
  \end{itemize}
  Darbdavys tada sumoka: VSD = 7000 $\cdot$ 0,5 $\cdot$ 0,3098 Lt.
\end{exmp}

PSDF ir VSDF  neskaičiuojamos ir nemokamos šiais atvejais:
\begin{itemize}
  \item nuo autoriams mokamo atlyginimo už kurinių panaudojimą
    pagal suteiktas licencijas panaudoti kūrinius;
  \item taip pat nuo autoriams mokamo kompensacinio atlygini…
\end{itemize}

Apmokestinimo problemos:
\begin{itemize}
  \item pagal autorines sutartis gautoms pajamoms taikomas 15\% GPM
    tarifas, o įregistravus individualią veiklą – 5\%;
  \item išanalizavus GPM nuo su darbo santykiais susijusių pajamų vengimo
    atvejus nustatyta, kad ūkio subjektai faktinius darbo santykius
    ar jų…
  \item atvejai, kai vietoj darbo sutarčių sudaromos autorinės sutartys,
    dažniau pasitaiko paslaugų teikimo srityje;
  \item gana sunku įrodyti, kad gyventojams išmokėtas autorinis atlyginimas
    faktiškai…
\end{itemize}

Atlygio už autorinius kūrinius paskirstymas. Autorinis atlyginimas
paskirstomas remiantis kūrinių …

Atlygis Lietuvos autoriams. Autorinis atlyginimas, surinktas už kūrinių
kabelinę retransliaciją, paskirstomas pagal retransliuojamas programas,
73 \% surinktos sumos skiriant muzikos kūrinių autoriamas, 23\% procentus
…

Atlygis užsienio autoriams. Autorinis atlyginimas užsienio autoriams,
surinktas už jų…

Žr.: Autorinių ir gretutinių teisių įstatymą.
