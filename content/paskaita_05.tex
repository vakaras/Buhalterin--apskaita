\chapter{Individuali veikla}

\section{Veiklos pradžia}

\section{Įmonės ir individualios veiklos skirtumai}

\begin{itemize}
  \item civilinė atsakomybė:
    \begin{itemize}
      \item individualios veiklos atveju – atsakom visu savo turtu;
      \item UAB – atsako įmonės turtu;
      \item individualios įmonės atveju – 
    \end{itemize}
  \item įstatinis kapitalas tik AB ir UAB;
  \item įmonės valdymas (įskaitant buhalterinės apskaitos valdymą) –
    žymiai sudėtingesnė įmonėse.
  \item finansiniai įsipareigojimai – individualioje veikloje esi,
    kaip fizinis asmuo, o įmonė, kaip juridinis asmuo.
\end{itemize}

\section{Įmonės registravimas ir veikla}

\begin{enumerate}
  \item Sprendimas pradėti įmonės veiklą.
  \item Įmonės registravimas Juridinių asmenų registre.
  \item Pranešimo iš Valstybinės mokesčių inspekcijos gavimas.
  \item Papildomų duomenų Valstybinei mokesčių inspekcijai teikimas.
  \item Mokestinės prievolės: …
\end{enumerate}

\section{… apmokestinimas}

Individualios veiklos rūšys:
\begin{itemize}
  \item pagal pažymą – turėtų trukti 1-2 mėnesius, šeimos nariai turėtų
    būti darbinami pagal darbo sutartį;
  \item pagal verslo liudijimą – gali būti įdarbinami šeimos nariai,
    gali būti vienai dienai; negalima dirbti veiklos, kuri yra 
    licencijuojama (statyba, kasyba ir t.t.);
  \item advokatai, jų padėjėjai, notarai antstoliai – su jais nėra darbo
    sutarčių;
  \item sportininkų ir kitų atlikėjų veikla – darbo sutarčių su
    sportininkais sudaryti negalima.
\end{itemize}

\section{Pagal pažymą}

Prieš išsiregistruojant būtina pranešti 5 dienos prieš.

Yra sąrašas, kokiomis veiklomis galima užsiimti.

Kaupimo principas sudėtingėja, didėjant darbo apimtims.

\begin{enumerate}
  \item Sprendimas vykdyti veiklą pagal pažymą.
  \item Prašymo dėl įregistravimo į Mokesčių mokėtųjų registrą teikimas.
  \item Įregistruojamas į MMR.
  \item Pažymos apie įregistravimą išdavimas.
  \item Mokestinės prievolės: mokesčių mokėjimas ir deklaravimas.
  \item Individualios veiklos nutraukimas.
\end{enumerate}

\subsection{Mokesčių mokėtojai}

Mokesčius (gyventojų pajamų mokestį, įmokas į valstybinį socialinį ir
privalomą sveikatos draudimo fondus) moka…

\subsection{Nuolatinis Lietuvos gyventojas}

Kad, galėtų užsiimti veikla turi būti bent 18 metų.

\subsection{Individualios veiklos samprata}

Tęstinumas – vienas iš aštuonių apskaitos principų: veikla turėtų būti
tęsiama. 

\subsection{Individualios veiklos rūšys}

\begin{itemize}
  \item Sporto veikla – negali būti treneriai.
  \item Atlikėjo veikla – vestuvių muzikantas, jau negali būti.
    Atlikėjas privalo rengti viešus pasirodymus.
\end{itemize}

\subsection{Individualios veiklos registravimas}

\subsection{Gyventojų pajamų mokestis}

\subsubsection{Individualios veiklos pajamos}

\subsubsection{B klasės pajamos}

\subsubsection{A klasės pajamos}

Viskas, kas neįeina į B klasės pajamas.

\subsubsection{Pajamų pripažinimo momentas}

Kaupimo principas – pajamos pripažįstamos tik tada, kai gauname
konkrečius pinigus, o ne tada kai išrašoma sąskaita.

\subsubsection{Pajamų pripažinimas}

\subsubsection{GPM tarifas}

\subsubsection{Turto pardavimo pajamos}

\subsubsection{Individualios veiklos turto įsigijimo išlaidos}

\subsubsection{Remonto ir rekonstrukcijos išlaidos}

\subsubsection{Ribojamų dydžių leidžiami atskaitymai}

\begin{itemize}
  \item …
  \item Narių mokesčiai (ne daugiau kaip 0,2\% pajamų).
\end{itemize}

\subsubsection{Leidžiamų atskaitymų vertinimas}

\subsubsection{Nuolatinio LT gyventojo neleidžiami atskaitymai}

Galima taikyti metinį NPD, mėnesinio – negalima.

\subsubsection{Nenuolatinio LT gyventojo neleidžiami atskaitymai}

\subsubsection{Nuostolių perkėlimas}

\subsubsection{Mokesčio apskaičiavimas}

\subsubsection{Reikalavimai apskaitai}

\subsubsection{Mokesčio deklaravimas}

\subsection{Pridėtinės vertės mokestis}

\subsubsection{Registravimasis PVM mokėtojų registre}

\subsubsection{Nenuolatinių Lietuvos gyventojų PVM}

\subsection{Privalomas socia… draudimas}

\subsubsection{Individuali veikla pagal pažymą}

\section{Pagal verslo liudijimą}

\subsection{Privalumai ir trūkumai}

\begin{enumerate}
  \item Sprendimas vykdyti veiklą pagal verslo liudijimą.
  \item Prašymo dėl verslo liudijimo teikimas.
  \item Apskaičiuoto pajamų mokesčio sumokėjimas (taip pat ir visų skolų).
  \item …
\end{enumerate}

\subsection{Veiklos samprata turint verslo liudijimą}

\subsection{Verslo liudijimo galiojimo ribos}

\subsection{Verslo liudijimo turėtojas}

\subsection{Verslo liudijimo išdavimas}

\subsection{Verslo liudijime esantys skiriamieji ženklai}

\subsection{Verslo liudijimo galiojimo pratęsimas}

\subsection{Duomenų pasikeitimas}

\subsection{Pajamų apmokestinimas GPM}

\subsection{GPM dydis}

MMA – minimalus mėnesinis atlyginimas (800 Lt).

\subsection{GPM lengvatos}

\subsection{GPM apskaičiavimas}

\subsection{GPM grąžinimas}

\subsection{VSDF ir PSDF įmokos}

\subsection{VSDF ir PSDF įmokų dydžiai}
Bazinė pensija yra 360 Lt.

\subsection{Reikalavimai apskaitai}

Įdomus pastebėjimas: kvitą turi išrašyti pirkėjas.

\subsection{Pajamų deklaravimas}

\subsection{Atsakomybė už registracijos pažeidimus}

\subsection{Registravimasis PVM mokėtojų registre}

\subsection{Nenuolatinių Lietuvos gyventojų PVM}

\section{Sportininkų ir atlikėjų veikla}

\subsection{Veiklos samprata}

Viešas pasirodymas gali būti ir internetu, per televiziją ir panašiai.
(Vestuvių dainininkai nebus priskiriami.)

\subsection{Mokesčių mokėtojai}

\subsection{Atlikėjų ir sportininkų pajamos iš darbo santykių}

\subsection{Susijusių (darbo santykiais) atlikėjų ir sportininkų pajamos}

\begin{enumerate}
  \item Sprendimas vykdyti veiklą pagal pažymą.
  \item Prašymo dėl įregistravimo į MMR teikimas.
  \item Įregistruojamas į MMR.
  \item Pažymos…
\end{enumerate}

\subsection{Atlikėjų ir sportininkų ne iš darbo santykių pajamos}

\subsection{Nesusijusių (darbo santykiais) atlikėjų ir sportininkų
pajamos}

LRV = Lietuvos Respublikos Vyriausybės dydis (~1400 Lt).

\subsection{Registravimasis PVM mokėtojų registre}

\section{Advokatai, notarai ir antstoliai}

\subsection{Advokatų, notarų ir antstolių pajamos}


