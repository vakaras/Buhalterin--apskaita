\chapter{Su darbo santykiais susijusių pajamų apmokestinimas.}

\begin{enumerate}
  \item Neamokestinamaisiais dydžiais skaičiuojant gyventojų pajamų
    mokestį nenuolatiniam Lietuvos gyventojui yra laikoma:
    \begin{itemize}
      \item kintamas neapmokestinamas pajamų dydis.
    \end{itemize}
  \item Išmokos, nuo kurių yra skaičiuojamas gyventojų pajamų mokestis:
    \begin{itemize}
      \item darbo užmokesčio avansas;
      \item atostoginiai;
      \item darbo užmokesčio priedas (premija);
      \item kompensacija už nepanaudotas kasmetines atostogas;
      \item komandiruotės dienpinigiai, sudarantys 30 \% pagrindinio
        daro užmokesčio, tačiau sudarantys 60\% mokėtino darbo
        užmokesčio su premija;
      \item ligos pašalpa, apskaičiuota darbuotojui už pirmąsias dvi
        nedarbingumo dienas.
    \end{itemize}
  \item Išmokos, nuo kurių yra skaičiuojamos įmokos į PSDF ir VSDF:
    \begin{itemize}
      \item darbo užmokesčio avansas;
      \item atostoginiai;
      \item darbo užmokesčio priedas (premija);
      \item kompensacija už nepanaudotas kasmetines atostogas;
      \item komandiruotės dienpinigiai…
    \end{itemize}
  \item Apskaičiuojami mokesčiai, susiję su darbo santykiais iš visos
    apskaičiuotos darbo užmokesčio sumos atėmus NPD ir PNPD…
    \emph{Taip.}
  \item Darbuotojas yra atleidžiamas iš darbo ir jam išmokama kompensacija
    už nepanaudotas atostogas 600 Lt ir už pavėluotą darbo užmokesčio
    mokėjimą buvo apskaičiuoti ir išmokėti delspinigiai 90 Lt,…
    \emph{Ne.}
  \item Atsakomybė už metinio neapmokestinamo pajamų dydžio apskaitą
    tenka darbdaviui.
    \emph{Ne.}
    P. S. Buhalteris nėra darbdavys.
  \item Skaičiuojant neapmokestinamą pajamų dydį mokestiniu laikotarpiu
    yra atsižvelgiama tik į visas kas mėnesį mokamas išmokas, susijusias
    su darbo santykiais arba jų esmę atitinkančiais santykiais.
    \emph{Ne.}
  \item Kaip yra vertinamos gyventojo gautos pajamos natūra…

    Kai darbdavys savo darbuotojui suteikia naudą natūra, tai priskiriama
    su darbo santykiais susijusioms pajamoms. Pajamų natūra davėjas
    ir pajamų natūra gavėjas abipusiu susitarimu gali pasirinkti
    vieną iš gautos naudos vertinimo būdų:
    \begin{itemize}
      \item gautą naudą vertinti automobilio nuomos tikrąją rinkos kaina,
        apskaičiuojant pajamas natūra šiuo būdu, turi būti įvertinama
        turto naudojimo asmeniniais tikslais apimtis 
      \item gautą naudą vertinti pagal vieną iš normų (0,7\% arba 0,75\%)
        asmeniniais tikslais naudojamo automobilio tikrosios rinkos
        kainos, vertinant šiuo būdu nereikia nustatyti kiek faktiškai
        gyventojas naudojasi automobiliu asmeniniais tikslais; taikant
        0,75\% normą, ji apima visą gyventojo gaunamą naudą, susijusią
        su automobilio naudojimu asmeniniais tikslais, įskaitant
        pajamų natūra davėjo išlaidas gyventojo asmeniais tikslais
        sunaudotiems degalams.
    \end{itemize}

  \item Apskaičiuokite mokėtiną gyventojų pajamų mokestį, jeigu įmonės
    vadovas

    bruto = ant popieriaus, neto = į rankas

    Jei taikom 0,75\%, tai:
    \begin{align*}
      \t{pajamos natūra:} & 60000 \cdot 0,75\% = 450 Lt \\
      \t{ant popieriaus iš viso:} & 1450 Lt \\
      \t{NPD:} & 340 Lt \\
      \t{GPM:} & 166,50 Lt \\
    \end{align*}

  \item Apskaičiuokite mokėtinus įmonės mokesčius jeigu 2012-01-23 yra
    įdarbinamas asmuo, neturintis jokių teisių į pastovų ar papildomą
    neapmokestinamąjį dydį, kuriam numatomas 800 Lt bruto darbo
    užmokestis ir 2012-02-27 atleidžiamas iš darbo.

    Sausis turi 22 darbo dienas. Jis dirbo 7 dienas. Už sausį jis gauna
    $\frac{800}{22} \cdot 7 = 254,55 Lt$. Vasaris turi 20 darbo dienų.
    Jis dirbo iš jų 18. Taigi už vasarį jis gauna
    $\frac{800}{20} \cdot 18 = 720 Lt$.

    Už darbo laiką, jam priklauso $\frac{28}{365} \cdot 25 = 1,918$
    atostogų. Kadangi jis atleistas vasarį, tai skaičiuojame įdirbį
    pagal praeitą mėnesį, tai yra sausį:
    $1,918 \cdot 0,7 = 1,34 \cdot \frac{254,55}{7} = 48,72$.

    Darbuotojo darbo užmokestis: $720 + 254,55 + 48,72 = 1023,27$.
    \begin{align*}
      \t{VSDF:} & 1023,27 \cdot 0,3098 + 1023,27 \cdot 0,02\% = 319,06 \\
    \end{align*}

  \item Apskaičiuokite darbdavio mokėtinus mokesčius, jeigu neturinčiam
    teisės į PNPD ar fiksuotą NPDF darbuotojo, kurio pagrindinėje
    darbovietėje bruto darbo užmokestis yra 2150 Lt ir jam išmokama
    100 Lt priemoka už dirbtus viršvalandžius bei įteikiamas 900 Lt
    vertės abonementas į sporto klubą.

    \begin{align*}
      \t{Iš viso ant popieriaus: } & 2150 + 100 + (900 - 700) = 2450 Lt \\
      \t{VSDF:} & 2450 \cdot 0,3098 = 759,01 Lt \\
      \t{GF:} & 2450 \cdot 0,2 \% = 4,9 Lt \\
      \t{Viso:} & 763,91 Lt \\
    \end{align*}

  \item Tingiu…

    \begin{align*}
      \t{DU:} & \frac{800}{23} \cdot 11 = 382,61 Lt \\
      \t{2 nedarbo dienos:} &
        \frac{800}{23} \cdot 2 \cdot 0,85 = 59,13 Lt \\
      \t{DU iš viso:} & 382,61 + 59,13 = 441,74 Lt \\
      \t{GPM:} & 441,74 \cdot 15\% = 66,26 Lt \\
      \t{PSD:} & 382,61 \cdot 9\% = 34,43 Lt \\
      \t{Viso mokesčių:} & 100,69 Lt \\
      \t{VSD:} & 382,61 \cdot 0,3098 = 118,53 Lt \\
      \t{GF:} & 382,61 \cdot 0,02 \% = 0,77 Lt \\
      \t{Darbdavys iš viso mokesčių:} & 119,30 Lt \\
    \end{align*}

  \item Velniškai tingiu…

    \begin{align*}
      \t{DU už pirmas dvi:} & 286,96 Lt \\
      \t{DU už likusias 11:} & 44,35 Lt \\
      \t{DU iš viso:} & 286,96 + 44,35 = 331,31 Lt \\
      \t{GPM:} & 0 Lt \\
      \t{PSD:} & 286,96 * 0,09 = 25,83 Lt \\
      \t{Į rankas:} & 331,31 - 25,83 = 305,48 Lt \\
      \t{VSD:} & 88,90 Lt \\
      \t{GF:} & 0,57 Lt \\
      \t{Viso mokesčių:} & 89,47 Lt \\
    \end{align*}
    
  \item Vis dar tingiu…

    \begin{align*}
      \t{DU:} & 1000 Lt \\
      \t{GPM:} & 150 Lt \\
      \t{PSD:} & 90 Lt \\
      \t{Į rankas:} & 760 Lt \\
      \t{Delspinigiai:} & 760 \cdot 0,6 \% \cdot (13 - 7) = 2,74 Lt \\
      \t{DU už atostogas:} 500 Lt \\
      \t{DU iš viso:} 1500 Lt \\
      \t{…}
      \t{VSD:} & 464,7 Lt \\
      \t{GF:} & 3 Lt \\
    \end{align*}

\end{enumerate}
