\chapter{Uždaviniai: Su darbo santykiais susijusių pajamų apmokestinimas.}

\begin{tasks}

  \emph{Atsakymą žymėkite taip: apibraukite teisingo atsakymo numerį,
  galimi ir nei vienas ar keli teisingi atsakymai.}
  
  \begin{task}
    \begin{condition}
      Neapmokestinamaisiais dydžiais skaičiuojant gyventojų pajamų
      mokestį nenuolatiniam Lietuvos gyventojui yra laikoma:
      \begin{enumerate}
        \item neapmokestinamasis pajamų dydis;
        \item pastovus pajamų dydis;
        \item papildomas neapmokestinamasis pajamų dydis;
        \item fiksuotas neapmokestinamasis pajamų dydis;
        \item \label{p04:t1:a5} kintamas neapmokestinamasis pajamų dydis.
      \end{enumerate}
    \end{condition}
    \begin{solution}
      Teisingas atsakymas yra \ref{p04:t1:a5}. TODO: Kodėl?
    \end{solution}
  \end{task}

  \begin{task}
    \begin{condition}
      Išmokos, nuo kurių yra skaičiuojamas gyventojų pajamų mokestis:
      \begin{enumerate}
        \item \label{p04:t2:a1} darbo užmokesčio avansas;
        \item \label{p04:t2:a2} atostoginiai;
        \item \label{p04:t2:a3} darbo užmokesčio priedas (premija);
        \item \label{p04:t2:a4} kompensacija už nepanaudotas kasmetines
          atostogas;
        \item \label{p04:t2:a5} komandiruotės dienpinigiai, sudarantys
          30 \% pagrindinio daro užmokesčio, tačiau sudarantys 60\%
          mokėtino darbo užmokesčio su premija;
        \item \label{p04:t2:a6} įteikta darbuotojui dovana – dovanų
          čekis, kurio vertė 400 Lt;
        \item \label{p04:t2:a7} pašalpa, išmokėta mirus darbuotojo
          šeimos nariui;
        \item \label{p04:t2:a8} ligos pašalpa, apskaičiuota
          darbuotojui už pirmąsias dvi nedarbingumo dienas.
      \end{enumerate}
      \begin{solution}
        Viskas, išskyrus \ref{p04:t2:a6} (apmokestinama tik nuo 700 Lt)
        ir \ref{p04:t2:a7}.
      \end{solution}
    \end{condition}
  \end{task}

  \begin{task}
    \begin{condition}
      Išmokos, nuo kurių yra skaičiuojamos įmokos į PSDF ir VSDF:
      \begin{enumerate}
        \item \label{p04:t3:a1} darbo užmokesčio avansas;
        \item \label{p04:t3:a2} atostoginiai;
        \item \label{p04:t3:a3} darbo užmokesčio priedas (premija);
        \item \label{p04:t3:a4} kompensacija už nepanaudotas kasmetines
          atostogas;
        \item \label{p04:t3:a5} komandiruotės dienpinigiai, sudarantys
          30\% pagrindinio darbo užmokesčio sudarantys 60\% mokėtino
          darbo užmokesčio su premija;
        \item \label{p04:t3:a6} įteikta darbuotojui dovana – dovanų
          čekis, kurio vertė 400 Lt;
        \item \label{p04:t3:a7} pašalpa, išmokėta mirus darbuotojo
          šeimos nariui;
        \item \label{p04:t3:a8} ligos pašalpa, apskaičiuota
          darbuotojui už pirmąsias dvi nedarbingumo dienas.
      \end{enumerate}
    \end{condition}
    \begin{solution}
      Teisingi: \ref{p04:t3:a1}, \ref{p04:t3:a2}, \ref{p04:t3:a3},
      \ref{p04:t3:a4}, \ref{p04:t3:a5}.
    \end{solution}
  \end{task}

  \emph{Atsakymą žymėkite taip: jei sutinkate su teiginiu, po
  juo parašykite „taip“, jei nesutinkate – „ne“.}

  \begin{task}
    \begin{condition}
      Apskaičiuojami mokesčiai, susiję su darbo santykiais iš visos
      apskaičiuotos darbo užmokesčio sumos atėmus NPD ir PNPD,
      gautas skirtumas apmokestinamas galiojančiu gyventojų pajamų
      mokesčio tarifu.
    \end{condition}
    \begin{solution}
      \emph{Taip.}
    \end{solution}
  \end{task}

  \begin{task}
    \begin{condition}
      Darbuotojas yra atleidžiamas iš darbo ir jam išmokama
      kompensacija už nepanaudotas atostogas 600 Lt ir už pavėluotą
      darbo užmokesčio mokėjimą buvo apskaičiuoti ir išmokėti
      delspinigiai 90 Lt, kurie yra apskaičiuoti pagal LR
      delspinigių nustatymo už išmokų, susijusių su darbo santykiais
      pavėluotą mokėjimą yra apmokestinami gyventojų pajamų
      mokesčiu.
    \end{condition}
    \begin{solution}
      \emph{Ne.}
    \end{solution}
  \end{task}

  \begin{task}
    \begin{condition}
      Atsakomybė už metinio neapmokestinamo pajamų dydžio apskaitą
      tenka darbdaviui.
    \end{condition}
    \begin{solution}
      \emph{Ne.} (Buhalteris nėra darbdavys.)
    \end{solution}
  \end{task}

  \begin{task}
    \begin{condition}
      Skaičiuojant neapmokestinamą pajamų dydį mokestiniu
      laikotarpiu yra atsižvelgiama tik į visas kas mėnesį mokamas
      išmokas, susijusias su darbo santykiais arba jų esmę
      atitinkančiais santykiais.
    \end{condition}
    \begin{solution}
      \emph{Ne.}
    \end{solution}
  \end{task}

  \emph{Atsakymą žymėkite taip: po klausimu įrašykite atitinkamą
  atsakymą.}

  \begin{task}
    \begin{condition}
      Kaip yra vertinamos gyventojo gautos pajamos natūra, kai
      automobilis, kuris yra registruotas kaip darbovietės ilgalaikis
      turtas yra naudojamas ir asmeniniais darbuotojo tikslais?
    \end{condition}
    \begin{solution}
      Kai darbdavys savo darbuotojui suteikia naudą natūra, tai
      priskiriama su darbo santykiais susijusioms pajamoms. Pajamų
      natūra davėjas ir pajamų natūra gavėjas abipusiu susitarimu
      gali pasirinkti vieną iš gautos naudos vertinimo būdų:
      \begin{itemize}
        \item gautą naudą vertinti automobilio nuomos tikrąją rinkos kaina,
          apskaičiuojant pajamas natūra šiuo būdu, turi būti įvertinama
          turto naudojimo asmeniniais tikslais apimtis;
        \item gautą naudą vertinti pagal vieną iš normų (0,7\% arba 0,75\%)
          asmeniniais tikslais naudojamo automobilio tikrosios rinkos
          kainos, vertinant šiuo būdu nereikia nustatyti kiek faktiškai
          gyventojas naudojasi automobiliu asmeniniais tikslais; taikant
          0,75\% normą, ji apima visą gyventojo gaunamą naudą, susijusią
          su automobilio naudojimu asmeniniais tikslais, įskaitant
          pajamų natūra davėjo išlaidas gyventojo asmeniais tikslais
          sunaudotiems degalams.
      \end{itemize}
    \end{solution}
  \end{task}

  \begin{task}
    \begin{condition}
      Apskaičiuokite mokėtiną gyventojų pajamų mokestį, jeigu
      įmonės vadovas, kurio bruto (bruto yra ant popieriaus, o
      neto – į rankas) darbo užmokestis yra 1000 Lt įmonei nuosavybės
      teise priklausančiu lengvuoju automobiliu, kurio balansinė
      vertė yra 20 000 Lt, o vidutinė rinkos vertė yra 60 000 Lt
      30 \% laiko naudoja savoms reikmėms.
    \end{condition}
    \begin{solution}
      Jei taikom 0,75\% vertinimo būda, tai vadovas natūra ant
      popieriaus dar gauna $60000Lt \cdot 0,75\% = 450 Lt$,
      o iš viso ant popieriaus jis gauna: $1450 Lt$. Taigi:

      \pythonba{darbo}{1450 popierius}
      
    \end{solution}
  \end{task}

  \begin{task}
    \begin{condition}
      Apskaičiuokite mokėtinus įmonės mokesčius jeigu 2012-01-23
      yra įdarbinamas asmuo, neturintis jokių teisių į pastovų ar
      papildomą neapmokestinamąjį dydį, kuriam numatomas 800 Lt
      bruto darbo užmokestis ir 2012-02-27 atleidžiamas iš darbo.
    \end{condition}
    \begin{solution}
      Sausis turi 22 darbo dienas. Jis dirbo 7 dienas. Už sausį jis
      gauna $\frac{800}{22} \cdot 7 = 254,55 Lt$. Vasaris turi 20 darbo
      dienų. Jis dirbo iš jų 18. Taigi už vasarį jis gauna
      $\frac{800}{20} \cdot 18 = 720 Lt$.

      Už darbo laiką, jam priklauso\footnote{%
      Nepanaudotos atostogos skaičiuojamos kalendorinėmis, o darbo
      užmokestis/kompensacija – darbo dienomis.

      \emph{Kasmetinės atostogos – tai kalendorinėmis dienomis
      skaičiuojamas laikotarpis, [...] darbuotojui [...] mokant
      vidutinį darbo užmokestį} [Darbo kodeksas, 165.1]

      \emph{Kompensacijos dydis nustatomas pagal nepanaudotų
      kasmetinių atostogų, [...], darbo dienų skaičių
      } [Darbo kodeksas, 177.2]

      Darbo dienų metuose koeficientus skelbia vyriausybė
      (\url{http://goo.gl/hbB2z}).

      Penkių darbo dienų savaitei, koeficientas yra 0,7.}
      $\frac{28}{365} \cdot 25 = 1,918$
      dienų atostogų. Kadangi jis atleistas vasarį, tai skaičiuojame
      įdirbį pagal praeitą mėnesį, tai yra sausį:
      $1,918 \cdot 0,7 \cdot \frac{254,55}{7} %
= 1,34 \cdot \frac{254,55}{7} = 48,72$.

      Darbuotojo darbo užmokestis ant popieriaus:
      $720 + 254,55 + 48,72 = 1023,27$.

      \pythonba{darbo}{1023.27 popierius}

    \end{solution}
  \end{task}

  \begin{task}
    \begin{condition}
      Apskaičiuokite darbdavio mokėtinus mokesčius, jeigu
      neturinčiam teisės į PNPD ar fiksuotą NPD darbuotojo, kurio
      pagrindinėje darbovietėje bruto darbo užmokestis yra 2150 Lt ir
      jam išmokama 100 Lt priemoka už dirbtus viršvalandžius bei
      įteikiamas 900 Lt vertės abonementas į sporto klubą.
    \end{condition}
    \begin{solution}
      Kadangi prizai apmokestinami tik virš 700 Lt, tai darbuotojui
      iš viso ant popieriaus: $2150 + 100 + (900 - 700) = 2450 Lt$.

      \pythonba{darbo}{2450 popierius}
      
    \end{solution}
  \end{task}

  \begin{task}
    \begin{condition}
      Apskaičiuokite darbdavio mokėtinus mokesčius ir darbo
      užmokestį, jeigu I grupės invalidumą turinčio vaikų neturinčio
      darbuotojo, kurio nepagrindinėje darbovietėje bruto darbo
      užmokestis yra 800 Lt, darbuotojui sergant 12 darbo dienų iš 23
      priskaičiuojamos ligos pašalpa už pirmąsias dvi nedarbingumo
      dienas siekia 85 \%.
    \end{condition}
    \begin{solution}
      Darbuotojas iš 23 darbo dienų $23 - 12 = 11$ dirbo normaliai
      ir už tai ant popieriaus gavo
      $\frac{800 \cdot 11}{23} = 382,61 Lt$. Už pirmas dvi jo
      ligos dienas apmoka darbdavys 85 \%, tai yra
      $\frac{800 \cdot 2}{23} \cdot 85\% = 59,13 Lt$.
      Taigi iš viso darbuotojas ant popieriaus gavo
      $382,61 + 59,13 = 441,74 Lt$. Kadangi darbuotojas dirba
      ne pagrindinėje savo darbovietėje, tai NPD netaikome.
      GPM skaičiuojame nuo visos sumos, o visus kitus, tik
      nuo tada, kai jis buvo sveikas.

      \begin{tabularx}{15cm}{p{7.5cm}|p{7.5cm}}
        Ant popieriaus: & $441,74Lt$ \\
        GPM: & $441,74 \cdot 15\% = 66,26Lt$ \\
        PSDF: & $382,61 \cdot 9\% = 34,43Lt$ \\
        Į rankas: & $341,05Lt$ \\
        Darbuotojas sumoka: & $441,74 - 341,05 = 100,69Lt$ \\
        VSDF: & $382,61 \cdot 30,98\% = 118,53Lt$ \\
        GF: & $382,61 \cdot 0,2\% = 0,77Lt$ \\
        Darbo vietos kaina: & $561,04Lt$ \\
        Įmonė sumoka mokesčių: & $561,04 - 441,74 = 119,30Lt$ \\
      \end{tabularx}

    \end{solution}
  \end{task}

  \begin{task}
    \begin{condition}
      Apskaičiuokite darbdavio mokėtinus mokesčius ir darbo
      užmokestį, jeigu du mažamečius auginančiam vienišam
      darbuotojui, kurio pagrindinėje darbovietėje bruto darbo
      užmokestis yra 600 Lt, darbuotojui sergant 12 darbo dienų iš 23
      priskaičiuojamos ligos pašalpa už pirmąsias dvi nedarbingumo
      dienas siekia 85 \%.
    \end{condition}
    \begin{solution}
      Normaliai darbuotojas dirbo $23 - 12 = 11$ dienų ir už tai
      ant popieriaus gavo $\frac{600 \cdot 11}{23} = 286,96 Lt$.
      Dar už pirmas dvi jo sirgimo dienas sumokėjo darbdavys
      $\frac{600 \cdot 2}{23} \cdot 85\% = 44,35 Lt$.
      Taigi iš viso darbuotojas ant popieriaus gavo
      $286,96 + 44,35 = 331,30 Lt$.

      \begin{tabularx}{15cm}{p{7.5cm}|p{7.5cm}}
        Ant popieriaus: & $331,30Lt$ \\
        NPD: & $470,00Lt$ \\
        GPM: & $0Lt$ \\
        PSDF: & $286,96 \cdot 9\% = 25,83Lt$ \\
        Į rankas: & $305,47Lt$ \\
        Darbuotojas sumoka: & $25,83Lt$ \\
        VSDF: & $286,96 \cdot 30,98\% = 88,90Lt$ \\
        GF: & $286,96 \cdot 0,2\% = 0,57Lt$ \\
        Darbo vietos kaina: & $420,77Lt$ \\
        Įmonė sumoka mokesčių: & $420,77 - 331,30 = 89,47Lt$ \\
      \end{tabularx}

    \end{solution}
  \end{task}

  \begin{task}
    \begin{condition}
      Apskaičiuokite darbdavio mokėtinus mokesčius bei darbo
      užmokestį, jeigu darbuotojas, kurio nepagrindinėje
      darbovietėje nėra gavęs jokių priemokų prie 1000 Lt bruto
      darbo užmokesčio, pastarąjį gavo pavėluotai, t.y. po 13
      dienų nei nurodyta terminuotoje darbo sutartyje ir parašęs
      prašymą išėjo kasmetinių atostogų 14 dienų laikotarpiui
      (kai mėnesis turi 20 darbo dienų).
    \end{condition}
    \begin{solution}
      Darbo užmokestis (kadangi darbovietė nepagrindinė, tai
      NPD netaikome):

      \begin{tabularx}{15cm}{p{7.5cm}|p{7.5cm}}
        Ant popieriaus: & $1000,00Lt$ \\
        GPM: & $1000,00 \cdot 15\% = 150,00Lt$ \\
        PSDF: & $1000,00 \cdot 9\% = 90,00Lt$ \\
        Į rankas: & $760,00Lt$ \\
        Delspinigiai: & $760,00 \cdot (13 - 7) \cdot 0,06\% = 2,74Lt$ \\
      \end{tabularx}

      Už 14 kalendorinių dienų atostogų, tai yra 10 darbo dienų,
      darbuotojui priklauso $1000 \cdot \frac{10}{20} = 500 Lt$.

      Taigi darbuotojas iš viso ant popieriaus gaus $1500 Lt$,
      tai yra:

      \pythonba{darbo|be|npd}{1500 popierius}
      
    \end{solution}
  \end{task}
\end{tasks}
