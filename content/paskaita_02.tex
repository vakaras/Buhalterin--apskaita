\chapter{Su darbo santykiais susijusių pajamų apmokestinimas}

\section{Darbdavio mokami mokesčiai}

VSD (Valstybinis socialinis draudimas) įmokos (priklauso nuo rizikos
grupės):
\begin{itemize}
  \item 30,98 \%
  \item 31,10 \%
  \item 31,70 \%
\end{itemize}

Garantinio fondo įmokos 0,2 \%, kurias moka tik juridiniai asmenys,
pavyzdžiui ūkininkai, įdarbinę darbuotoją šių įmokų nemoka.


\section{Darbuotojų mokami mokesčiai}

\begin{itemize}
  \item GPM (Gyventojų pajamų mokestis) 15\%
  \item VSD (Valstybinis socialinis draudimas) įmokos 3\%
  \item PSD (Privalomasis socialinis draudimas) įmokos 6\%
\end{itemize}

\section{Apmokestinamųjų pajamų apskaičiavimas}

Iš pajamų, susijusių su darbo santykiais yra atimamos neapmokestinamosios
pajamos, atimamas NPD (Neapmokestinamas pajamų dydis) ir PNPD
(Papildomas neapmokestinamas pajamų dydis).

Nuo gautos sumos apskaičiuojamas GPM 15\% mokestis, kurį sumoka
mokestį išskaičiuojantis asmuo:
\begin{itemize}
  \item iki mėnesio 15 dienos, jeigu darbo užmokestis išmokėtas
    iki šios dienos;
  \item iki paskutinė mėnesio dienos, jei darbo užmokestis išmokėtas
    po 15 dienos.
\end{itemize}

Neapmokestinamosios pajamos:
\begin{itemize}
  \item pašalpos, kurios yra išmokamos mirus darbuotojui;
  \item pašalpos, kurios yra išmokamos mirus darbuotojo sutuoktiniui, 
    vaikams ar tėvams;
  \item kompensacijos, kurios arba kurių dydis yra reglamentuotas
    LR įstatymais ir kitais norminiais aktais, išskyrus darbo sutarties
    nutraukimą ir nepanaudotas atostogas;
  \item darbdavio mokamos gyvybės draudimo įmokos jeigu sutarties terminas
    yra ilgesnis nei 10 metų arba išmoką gaunantis asmuo bus
    sulaukęs pensijinio amžiaus ir pensijų įmokų suma, neviršijanti
    per mokestinį laikotarpį 25\% darbuotojui apskaičiuotų su
    darbo santykiais susijusių pajamų;
  \item kitų gyventojų mokamos gyvybės draudimo ir pensijų įmokos;
  \item per mokestinį laikotarpį gautų prizų vertė, neviršijanti
    700 Lt;
  \item delspinigiai už išmokų, susijusių su darbo santykiais,
    pavėluotą mokėjimą;
  \item prizai (jei yra paskelbtas laimėtojas konkurso ar varžybų
    pagrindu).
\end{itemize}

\section{Pajamų natūra sąvoka}

\begin{defn}[Pajamos natūra]
  Pajamų išmokėjimo forma (o ne pajamų rūšis).
\end{defn}

Ne pajamos natūra:
\begin{itemize}
  \item gyventojo nauda, gauta asmeniui, susijusiam su gyventoju darbo
    santykiais arba jų esmę atitinkančiais santykiais sumokėjus
    (visiškai ar iš dalies) už gyventojui suteiktas gydymo
    paslaugas, kai to reikalauja teisės aktai;
  \item gyventojo nauda, gauta kitam asmeniui tiesiogiai
    …
  \item asmens, susijusio su gyventojų darbo santykiais arba jų esmę
    atitinkančiais…
\end{itemize}

\section{Pajamų natūra apskaita}

Darbdavys savo darbuotojams 2010 metais ir vėlesniais metais suteikęs naudą
natūra, kuri priskiriama…

\section{Komandiruotės Lietuvoje apmokestinimas}

Jei komandiruotė trunka ilgiau nei 1 dieną:
\begin{itemize}
  \item apskaičiuojami dienpinigiai pagal jų galiojančia normą
    (15\% BSI\footnote{Bazinė socialinė išmoka.}).
\end{itemize}

Jei komandiruotė trunka vieną dieną:
\begin{itemize}
  \item apskaičiuojami dienpinigiai pagal jų galiojančią normą
    (15 \% $\frac{BSI}{2}$);
  \item jeigu dienpinigių suma neviršija 50 \% darbo užmokesčio,
    tai jie neapmokestinami GPM;
  \item jeigu dienpinigių suma viršija 50 \% darbo užmokesčio,
    tai jie apmokestinami GPM.
\end{itemize}

\section{Komandiruotės užsienyje apmokestinimas}

\begin{itemize}
  \item Apskaičiuojama pagal dienpinigių vykstantiems į užsienio
    komandiruotes normas.
  \item jeigu darbuotojo, vykstančio į komandiruotę darbo
    užmokestis yra mažesnis nei 1,3 MMA (minimalioji mėnesinė
    alga), tai dienpinigiai neapmokestinami GPM;
  \item jeigu darbuotojo, vykstančio į komandiruotę, darbo
    užmokestis yra didesnis nei 1,3 MMA, tai apmokestinama
    tik ta dienpinigių suma, kuri viršija, jei
    viršija, 50 \% darbo užmokesčio.
\end{itemize}

Padidėjusios išlaidos gali būti kompensuojamos…

Išlaidos kompensuojamos už faktiškai dirbtą laiką, kai darbas atliekamas
kelionėje, lauko sąlygomis, susijęs…

\section{Neapmokestinamas pajamų dydis (NPD)}

\begin{itemize}
  \item Fiksuotas NPD.
  \item Kintamas NPD.
\end{itemize}

NPD mokestiniu laikotarpiu taikomas tik nuolatiniam LR gyventojui
ir tik su darbo santykiais susijusioms pajamoms.

\subsection{Fiksuotas NPD}

\begin{itemize}
  \item 800 Lt/mėn. asmenims, kuriems yra nustatytas 0-25\% darbingumo
    lygis, arba nustatytas sunkus neįgalumo lygis arba pensininkams
    nustatytas didelių specialiųjų poreikių lygis.
  \item 600 Lt/mėn. asmenims, kuriems yra nustatytas 30-55\% darbingumo
    lygis, arba nustatytas vidutinis ar lengvas neįgalumo lygis
    arba pensininkams nustatytas vidutinių ar nedidelių specialiųjų
    poreikių lygis.
\end{itemize}

\subsection{Kintamas NPD}

\begin{itemize}
  \item $DU \leq 800$ Lt, tai NPD lygus 470 Lt.
  \item 800 Lt < DU < 3150 Lt, tai
    NPD $= 470 - 0,2 (DU - 800)$;
  \item $DU \geq 3150$ Lt, tai NPD lygus 0.
\end{itemize}

\section{Kintamo NPD skaičiavimas}

Apskaičiuojant NPD į darbuotojo mėnesio pajamas, susijusias su darbo
santykiais, įskaitoma:
\begin{itemize}
  \item …
  \item Išeitinė kompensacija.
  \item Išeitinė išmoka.
  \item Ligos pašalpa, mokama iš darbdavio lėšų už dvi pirmąsias ligos
    dienas.
  \item …
\end{itemize}

Apskaičiuojant NPD į darbuotojo mėnesio pajamas, susijusias su darbo
santykiais, neįskaitoma:
\begin{itemize}
  \item dovanos;
  \item prizai;
  \item darbuotojo nauda, gauta darbdaviui sumokėjus darbuotojo
    apgyvendinimo ir gyvenamojo ploto išlaikymo išlaidas;
  \item darbuotojo gauta nauda darbdaviui sumokėjus už darbuotojui
    suteiktas įvairias laisvalaikio praleidimo, pramogų,
    poilsio, turizmo ir kitas panašaus pobūdžio paslaugas;
  \item …
\end{itemize}

\section{NPD taikymas}

\begin{itemize}
  \item NPD taikomas apskaičiuojant nuolatinių LR gyventojų 
    apmokestinamąsias pajamas.
  \item Nenuolatiniams LR gyventojams NPD taikomas tik pasibaigus
    mokestiniam laikotarpiui ir pateikus metinę GPM deklaraciją.
  \item …
  \item Gyventojas turi teisę atsisakyti mokestiniu laikotarpiu
    taikyti NPD.
  \item Pasibaigus mokestiniam laikotarpiui bus išsiaiškinta, kiek
    per metus galima buvo pritaikyti MNPD (metinio neapmokestinamų
    pajamų dydžio). Susidaręs NPD skirtumas turės būti sumokamas
    (grąžinamas) iš (į) valstybės biudžetą.
\end{itemize}

Gyventojo metinės pajamos yra lygios gyventojo mokestinio laikotarpio
apmokestinamųjų pajamų, išskyrus išmokas, mokamas pasibaigus…

\section{Metinio NPD taikymas}

\begin{itemize}
  \item Kai yra taikomas fiksuotas 800 Lt NPD, tai MNPD yra
    lygus 9600 Lt (800 $\cdot$ 12).
  \item Kai yra taikomas fiksuotas 600 Lt NPD, tai MNPD yra
    lygus 7200 Lt (600 $\cdot$ 12).
  \item Kai yra taikomas kintamas NPD:
    \begin{itemize}
      \item MNPD = 5640 Lt, kai metinės pajamos neviršija 9600 Lt
        (800 $\cdot$ 12);
      \item MNPD = $5640 - 0,2(DU - 9600)$, kai metinės pajamos
        viršija 9600Lt, tačiau yra mažesnės už 37 800 Lt
        (3150 $\cdot$ 12);
      \item MNPD = 0, kai metinės pajamos viršija 37 800 Lt.
    \end{itemize}
\end{itemize}

\section{Pavėluotai išmokėtas darbo užmokestis}

Kai ne dėl darbuotojo kaltės pavėluotai išmokamas darbo užmokestis ar
kitos su darbo santykiais susijusios …

Delspinigių dydį sudaro 0,06 \% priklausančios išmokėti sumos už
kiekvieną praleistą kalendorinę dieną, pradedant skaičiuoti po
7 kalendorinių dienų, kai išmokos teisės aktuose ar kolektyvinėje
(jeigu jos nėra – darbo)…

Delspinigiai nėra įskaitomi, kai yra apskaičiuojamas darbuotojo vidutinis
darbo užmokestis…

\section{Papildomas neapmokestinamas pajamų dydis (PNPD)}

Taikomas tik nuolatiniams Lietuvos gyventojams – tėvams arba įtėviams,
kurie, augina vaikus iki 18 metų, taip pat vyresnius, jeigu
jie mokosi dieninėse bendrojo lavinimo mokyklose.

\begin{itemize}
  \item Už pirmą vaiką – 100 Lt.
  \item Už antrą ir kiekvieną kitą – 200 Lt.
\end{itemize}

\section{PNPD taikymas}

Atsiradus arba pasibaigus teisei į PNPD ar teisei į didesnį
PNPD…

\section{Metinis PNPD}

Metinį PNPD, pateikęs metinę pajamų mokesčio deklaraciją, galės
prisitaikyti ir su darbo santykiais susijusių pajamų negavęs, tačiau
gavęs kitokios…

\section{VSDF draudimo rūšys}

\begin{itemize}
  \item Pensijų socialinis draudimas.
  \item Ligos ir motinystės socialinis draudimas.
  \item Nedarbo socialinis draudimas.
  \item Nelaimingų atsitikimų darbe ar profesinių ligų socialinis draudimas.
  \item Sveikatos draudimas.
\end{itemize}

\section{VSDF įmokų mokėtojai}

\begin{itemize}
  \item Apdraustieji asmenys.
  \item Draudėjai.
  \item Savarankiškai dirbantys asmenys (IĮ savininkai, TŪB ir KŪB
    tikrieji nariai, individualios veiklos vykdytojai…)

    EDV – ekonominio dydžio vienetas. ($\approx 1000 EU$)
\end{itemize}

\section{Apmokestinamosios pajamos}

Socialinio draudimo įmokos skaičiuojamos nuo kiekvienam …

\begin{itemize}
  \item Darbo užmokestis, kuris yra numatytas įstatymų ar kitų norminių
    aktų tvarka.
  \item …
\end{itemize}

\section{VSDF įmokų dydis}

Jeigu pagal įstatymus darbuotojas dirbo ne visą darbo laiką,…

\section{Draudėjų VSDF įmokų tarifai}

31,70 \% arba 31,10 \% arba  30,98 \%, kurie susideda iš:
\begin{itemize}
  \item draudėjų bendrojo įmokų tarifo – 30,7 \% (…)
    …
\end{itemize}

\section{Įmokų mokėjimo tvarka}

Draudėjo ir …

\section{Apdraustųjų VSDF įmokų tarifai}

Apdraustieji moka VSD įmokas:
\begin{itemize}
  \item 3 \% pensijų socialiniam draudimui;
  \item 6 \% sveikatos draudimui.
\end{itemize}

\section{VSF įmokų skaičiavimas}

VSD įmokos neskaičiuojamos nuo:
\begin{itemize}
  \item Pašalpos, kurią išmoka darbdavys mirus darbuotojui arba darbuotojo
    sutuoktiniui, vaikams, tėvams.
  \item Komandiruočių išmokų, kurios yra neapmokestinamos GPM.
  \item Išmokų turtinei žalai dėl sveikatos sutrikdymo ar mirčiai
    atlyginti.
  \item Ligos pašalpų, mokamų iš draudėjo lėšų už pirmąsias dvi
    ligos dienas…
  \item Draudėjų lėšų, sumokėtų už darbuotojų sveikatos profilaktiką.
  \item Teismų priteistos sumos turtinei …
\end{itemize}

\section{Savarankiškai dirbantys asmenys}

\section{Ligos pašalpos skaičiavimas}

Darbdavys už pirmąsias 2 ligos dienas gali neskaičiuoti jeigu
apdraustasis asmuo neturi reikiamos ligos ir…

\section{Ligos pašalpos gavėjai}

Teisę gauti pašalpą turi asmenys, kurie turi ne trumpesnį kaip
3 mėnesių per paskutinius 12 mėnesių arba ne trumpesnį kaip…

\section{Ligos pašalpos skaičiavimo tvarka}

…

\begin{itemize}
  \item Asmenims, gaunantiems autorinį atlyginimą, ligos pašalpa mokama
    nuo 3 laikinojo nedarbingumo dienos iki darbingumo atgavimo dienos
    iš VSDF lėšų.
    …
\end{itemize}

\section{Ligos pašalpų apmokestinimas}

Ligos pašalpos už 2 pirmąsias ligos dienas yra apmokestinamos
15\% GPM ir 6\% PSD.

Nėra apmokestinama 31\% ir nemoka VSDF?

\section{VSD įmokų deklaravimas}

\begin{itemize}
  \item Darbdavys priima naują darbuotoją ir prieš tris dienas pateikia
    VSD ataskaitą.
  \item Darbdavys atleidžia darbuotoją ir per 3 dienas po atleidimo
    pateikia VSD ataskaitą.
  \item Darbdavys apie darbuotojo nedraudiminius laikotarpius per 3 dienas
    nuo pradžios arba pabaigos pateikia VSD ataskaitą.
  \item Darbdavys apie vaiko priežiūros atostogas per 3 dienas nuo
    suteikimo ar atšaukimo pateikia VSD ataskaitą…
\end{itemize}

\section{Įmokų į garantinį fondą mokėtojai}

\section{Įmokų į GF dydis}

\section{Įmokų į GF mokėjimo tvarka}


