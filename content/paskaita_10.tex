\chapter{Turtas}

\setslideprefix{13 - turtas + Pratybos/skaidres.pdf:}

\section{Turto sąvoka}

\slide{2}

Turtu apskaitoje yra laikomi ekonominiai ištekliai, kurie turi
savininką ir kuriais disponuodama įmonė tikisi ateityje gauti
tam tikrą naudą (pelną).

\subsection{Ekonominių išteklių sąvoka}

\slide{3}

Gali būti ne tik materialūs dalykai, bet ir tokie dalykai, kaip
darbuotojai, gražus kraštovaizdis ir panašiai.

\section{Įmonės turto klasifikavimas}

\slide{4}

\subsection{Pagal naudojimo laiką}

\slide{5}

Yra skiriami trumpalaikis ir ilgalaikis turtas.

\subsubsection{Ilgalaikio turto sąvoka}

\slide{11}

Ilgalaikis turtas – tai turtas, kurio naudingas eksploatavimo
laikotarpis ilgesnis negu vienas ataskaitinis laikotarpis,
tai yra kuris naudojamas pajamoms uždirbti ilgiau negu
vienerius metus. Taip pat jis turėtų viršyti minimalią sumą
(rekomenduojama 2000Lt).

\subsubsection{Ilgalaikis materialusis turtas}

\slide{15}

Tai materialusis turtas, kuris teikia įmonei ekonominės naudos,
naudojamas ilgiau nei vienerius metus ir kurio įsigijimo
(pasigaminimo) savikaina yra ne mažesnė už įmonės nustatytą
minimalią ilgalaikio materialiojo turto vertę.

\subsubsection{Ilgalaikio materialiojo turto apskaitos dokumentai}

\slide{16}

\subsubsection{Ilgalaikio materialiojo turto apskaita}

\slide{17}

\slide{18}

\subsubsection{Ilgalaikio materialiojo turto įsigijimo savikaina}

\slide{19}

Savikaina = pirkimo kaina (+ negrąžintini mokesčiai) + išlaidos
(tiesiogiai susijusios su turto parengimu veikti ir naudoti).

Pavyzdys: \slide{20} \slide{21}

\subsubsection{Ilgalaikio materialiojo turto apskaitos ypatumai}

\slide{22}

\slide{23}

\slide{24}

\subsubsection{Ilgalaikio materialiojo turto nusidėvėjimas}

\slide{25}

Nusidėvėjimas – tai riboto naudojimo laiko turto nudėvimosios vertės
priskyrimas sąnaudoms ir paskirstymas per visą planuojamą naudingo
tarnavimo laiką, atsižvelgiant į realų to turto ekonominės
vertės kitimą.

\subsubsection{Ilgalaikio materialiojo turto nusidėvėjimo ypatumai}

\slide{26}

\subsubsection{Ilgalaikio materialiojo turto nusidėvėjimo metodai}

\slide{27}

Tiesiogiai proporcingas metodas: \slide{28}

Dvigubas mažėjančios vertės metodas: \slide{29}, \slide{30}

\subsubsection{Nusidėvėjimo metodų įtaka veiklos pelnui}

\slide{31}

\subsubsection{Ilgalaikio materialiojo turto nusidėvėjimas neskaičiuojamas}

\slide{32}

\subsubsection{Remontas ir rekonstravimas}

\slide{33}

\subsubsection{Ilgalaikio materialiojo turto netekimas}

\slide{34}

\subsubsection{Nebetinkamo naudoti turto likvidavimas}

\slide{35}

\slide{36}

\subsubsection{Turto perleidimas tretiesiems asmenims}

\slide{37}

\subsubsection{Ilgalaikio materialiojo turto nuvertėjimas}

\slide{38}

\subsubsection{Ilgalaikio materialiojo turto perkainojimas}

\slide{39}

\subsection{Pagal užbaigtumo laipsnį}

\slide{6}

Turtas yra skirstomas į žaliavas, nebaigtą gamybą ir gatavus gaminius.

\subsection{Pagal naudojimo pobūdį}

\slide{7}

Yra skirstomas į žemę, pastatus ir statinius, mašinas ir įrengimus,
atsargas, skolas įmonei ir pinigus.

\subsection{Pagal vaidmenį įmonės veikloje}

\slide{8}

Turtas yra skirstomas į aktyvųjį ir pasyvųjį.

\subsubsection{Aktyvusis turtas}

\slide{9}

Aktyviuoju turtu yra laikomos mašinos, įrenginiai, transporto priemonės,
įrankiai ir kitas tiesiogiai įmonės veikloje naudojamas turtas.

\subsubsection{Pasyvusis turtas}

\slide{10}

Pasyviuoju turtu yra laikomi žemė, gamtos ištekliai, pastatai, statiniai,
kiti įrenginiai. Tai turtas, tiesiogiai nedalyvaujantis gaminant
produkciją, bet sudarantis sąlygas šiai veiklai.

\section{Materialinių vertybių apskaita}

\slide{12}

\section{Nematerialusis turtas}

\slide{40}

Turtas, kurio negalima apčiuopti ar paliesti, jis neturi materialiojo
turinio. Jis vis labiau įsigali įmonių veikloje. Nematerialiajam turtui
galima priskirti programinę įrangą, firmos turimas autorines teises,
patentus, licencijas, įmonės prestižą, įvairias kitas privilegijas.

\subsection{Ilgalaikio nematerialiojo turto apskaita}

\slide{41}

Ilgalaikis nematerialusis turtas apskaitoje registruojamas debetuojant
nematerialiojo turto įsigijimo savikainos sąskaitas ir kredituojant
atitinkamas turto ar įsipareigojimų sąskaitas.

\slide{42}

\slide{43}

\subsection{Ilgalaikio nematerialiojo turto apskaitos dokumentai}

\slide{44}

\subsection{Ilgalaikio nematerialiojo turto įsigijimo būdai}

\slide{45}

Nematerialusis turtas gali būti:
\begin{enumerate}
  \item įsigytas iš išorinio šaltinio (už tam tikrą kainą, nemokamai,
    mainais);
  \item pasigamintas / sukurtas;
  \item įgytas po verslo sujungimo – prestižas.
\end{enumerate}

\subsubsection{Nematerialiojo turto, įsigyto iš išorinio šaltinio apskaita}

\slide{46}

\subsubsection{Ilgalaikis nematerialusis turtas}

\slide{47}

Nėra nematerialusis turtas:
\begin{itemize}
  \item darbuotojo mokymo, kvalifikacijos išlaidos;
  \item įmonės veiklos steigimo, reorganizavimo išlaidos;
  \item reklamos, reprezentacijos išlaidos;
  \item tyrimo išlaidos.
\end{itemize}

Yra nematerialusis turtas:
\begin{itemize}
  \item prekių ženklai (įsigyti);
  \item patentai ir licencijos;
  \item autorių ir gretutinės teisės;
  \item kompiuterių programos;
  \item taršos integruotos prevencijos ir kontrolės leidimai;
  \item plėtros išlaidos.
\end{itemize}

\subsubsection{Mainais įsigytas nematerialusis turtas}

\slide{48}

\subsubsection{Ilgalaikio nematerialiojo turto pavyzdžiai}

\slide{49}

\slide{50}

\slide{51}

\subsubsection{Ilgalaikio nematerialiojo turto amortizacija}

\slide{52}

Tiesiogiai proporcingas metodas: \slide{55}.

\subsubsection{Ilgalaikio nematerialiojo turto amortizacijos ypatumai}

\slide{53}

\slide{54}

\section{Turtas ir nuosavybė}

\slide{56}

Turtas parodo kas konkrečiai priklauso įmonei, kitaip tariant kokiu
turtu ji disponuoja. Skolintojų ir skolininkų nuosavybė nusako kiek ir
kam priklauso įmonės turtas.

\section{Prieš uždavinius}

1 – ilgalaikis turtas
2 – trumpalaikis turtas
D+, K-

3, 4
D-, K+

5
K

6
D

\chapter{Uždaviniai}

\begin{tasks}

  \begin{task}
    \begin{condition}
      UAB „Restoranas“ gegužės mėnesį įvyko tokios operacijos: …

      Sudarykite bendrąjį žurnalą, perkelkite operacijas į
      atitinkamas sąskaitas, sudarykite balansą 31 d. ir pelno
      (nuostolio) ataskaitą 31 d.
    \end{condition}
    \begin{solution}
      \begin{tabularx}{\ltablewidth}[]{X | p{8cm} | X | X}
        \multicolumn{4}{c}{Bendrasis žurnalas} \\
        Data & Turinys & Debetas & Kreditas \\
        \hline
        2012-05-05 &
          Įsteigta bendrovė UAB "Restoranas". Steigėjai - 3
          akcininkai. Jie įnešė po 30 000 Lt kiekvienas į
          atsiskaitomąją sąskaitą banke.
          & 271 & 301 \\
        2012-05-06 &
          Įsigytos patalpos restoranui už 80 000 Lt
          & 121 & 271 \\
        2012-05-07 &
          Pasirašyta sutartis su teikėjais dėl maisto produktų
          įsigijimo už 100000Lt
          & & \\
        2012-05-07 &
          Išsinuomotas sandėlis maisto produktams laikyti ir
          sumokėtas1400 Lt gegužės mėn. nuomos mokestis
          & 61 & 271 \\
        2012-05-08 &
          Į sandėlį įvesta signalizacija. Sumokėta 1600 Lt
          & 61 & 271 \\
        2012-05-10 &
          Įsigyta atsargų už 6000 Lt Sumokėti pinigai
          & 201 & 271 \\
        2012-05-10 &
          Sumokėta 300 Lt už paskelbtą reklamą
          & 61 & 271 \\
        2012-05-14 &
          Parduota patiekalų už 4200 Lt. Gauti pinigai
          & 271 & 50 \\
        2012-05-15 &
          Išsimokėtinai isigyta atsargų už 8400 Lt
          & 201 & 443 \\
        2012-05-20 &
          Parduota patiekalų už 8000 Lt, pirkėjai įsipareigojo
          sąskaitą apmokėti per 20 dienų
          & 241 & 50 \\
        2012-05-27 &
          Sumokėta transporto organizacijai už patiekalų pristatymą
          į namus pirkėjams 200Lt
          & 61 & 271 \\
        2012-05-31 &
          Išmokėtas DU už 3600 Lt
          & 61 & 271 \\
        2012-05-31 &
          Gauta 80 Lt sąskaita už elektrą. Įmonė ją apmokės kitą mėn.
          & 61 & 443 \\
        2012-05-31 &
          Atlikus inventorizaciją, nustatyta, kad sunaudota atsargų
          už 8540 Lt
          & 61 & 201 \\
      \end{tabularx}

      Pastaba dėl 2012-05-07: sutarties pasirašymai ataskaitose
      neregistruojami.

      Numeriai iš pavyzdinio sąskaitų plano:
      \begin{description}
        \item[121] – pastatai ir statiniai;
        \item[201] – atsargos;
        \item[241] – pirkėjų įsiskolinimas;
        \item[271] – sąskaitos bankuose;
        \item[301] – įstatinis pasirašytasis kapitalas (pagrindinis
          kapitalas);
        \item[443] – skolos tiekėjams;
        \item[50] – pardavimo pajamos;
        \item[61] – veiklos sąnaudos;
      \end{description}

      \strong{Turtas (1, 2 klasė)}

      \begin{PlaneTable}{121}
        2012-05-06 \hfill $80 000 Lt$ & \\
      \end{PlaneTable}

      \begin{PlaneTable}{201}
        2012-05-10 \hfill $6000 Lt$ & \\
        & 2012-05-31 \hfill $8540 Lt$ \\
      \end{PlaneTable}

      \begin{PlaneTable}{241}
        2012-05-20 \hfill $8000 Lt$ & \\
      \end{PlaneTable}

      \begin{PlaneTable}{271}
        2012-05-05 \hfill $90000 Lt$ & \\
        & 2012-05-06 \hfill $80000 Lt$ \\
        & 2012-05-07 \hfill $1400 Lt$ \\
        & 2012-05-08 \hfill $1600 Lt$ \\
        & 2012-05-10 \hfill $6000 Lt$ \\
        & 2012-05-10 \hfill $300 Lt$ \\
        2012-05-14 \hfill $4200 Lt$ & \\
        & 2012-05-27 \hfill $200 Lt$ \\
        & 2012-05-31 \hfill $3600 Lt$ \\
      \end{PlaneTable}

      \strong{Nuosavybė (3, 4 klasė)}

      \begin{PlaneTable}{301}
        & 2012-05-05 \hfill $90000 Lt$ \\
      \end{PlaneTable}

      \begin{PlaneTable}{443}
        & 2012-05-15 \hfill $8400 Lt$ \\
        & 2012-05-31 \hfill $80 Lt$ \\
      \end{PlaneTable}

      \strong{Pajamos (5 klasė)}

      \begin{PlaneTable}{50}
        & 2012-05-14 \hfill $4200 Lt$ \\
        & 2012-05-20 \hfill $8000 Lt$ \\
      \end{PlaneTable}

      \strong{Sąnaudos (6 klasė)}

      \begin{PlaneTable}{61}
        2012-05-07 \hfill $1400 Lt$ & \\
        2012-05-08 \hfill $1600 Lt$ & \\
        2012-05-09 \hfill $300 Lt$ & \\
        2012-05-27 \hfill $200 Lt$ & \\
        2012-05-31 \hfill $3600 Lt$ & \\
        2012-05-31 \hfill $80 Lt$ & \\
        2012-05-31 \hfill $8540 Lt$ & \\
      \end{PlaneTable}

      \begin{tabularx}{\tablewidth}[]{X | p{5cm} | X}
        \multicolumn{3}{c}{Pelno (nuostolio) ataskaita} \\
        Eil. Nr. & & Suma, Lt \\
        \hline
        1. & Pardavimai (50) & 12200 \\
        2. & Sąnaudos (61) & 15720 \\
        3. & Pelnas (nuostolis) & (3520) \\
        4. & Pelno mokestis & 0 \\
        5. & Grynasis pelnas & (3520) \\
      \end{tabularx}

      \begin{tabularx}{\tablewidth}[]{l | X}
        \multicolumn{2}{c}{Balansas} \\
        \multicolumn{2}{c}{Turtas} \\
        \hline
        \multicolumn{2}{l}{Ilgalaikis turtas} \\
        Pastatai ir statiniai & 80000 Lt \\
        \multicolumn{2}{l}{Trumpalaikis turtas} \\
        Atsargos & FIXME Lt \\
        Pirkėjų įsiskolinimas & 8000 Lt \\
        Sąskaitos bankuose & 1100 Lt \\
        \hline
        Turtas iš viso & FIXME:94960 Lt \\
        \multicolumn{2}{c}{Nuosavybė} \\
        \hline
        \multicolumn{2}{l}{Nuosavas kapitalas} \\
        Nuostolis & (3520) Lt \\
        Akcinis kapitalas & 90000 Lt \\
        \multicolumn{2}{l}{Mokėtinos sumos ir įsipareigojimai} \\
        Skolos tiekėjams & 8480 Lt \\
        \hline
        Nuosavybė iš viso & 94960 Lt \\
      \end{tabularx}

    \end{solution}
  \end{task}

\end{tasks}

\chapter{Užduotys: Finansinė apskaita}

\slide{58}

\begin{tasks}
  
  \emph{Atsakymą žymėkite taip: apibraukite teisingo atsakymo numerį.}

  \begin{task}
    \begin{condition}
      Išleista akcijų:
      \begin{enumerate}
        \titem{a} <++>
        \titem{b} padidėjo turto ir nuosavybės sumos;
        \titem{c} <++>
        \titem{d} <++>
      \end{enumerate}
    \end{condition}
    \begin{solution}
      Teisingas atsakymas yra \tref{b}.
    \end{solution}
  \end{task}

  \begin{task}
    \begin{condition}
      Turtinis įnašas kapitalui didinti (pastatas):
      \begin{enumerate}
        \titem{a} padidėjo turto (pastatas) ir savininko nuosavybės
          sumos;
        \titem{b} <++>
        \titem{c} <++>
        \titem{d} <++>
      \end{enumerate}
    \end{condition}
    \begin{solution}
      Teisingas atsakymas yra \tref{a}.
    \end{solution}
  \end{task}

  \begin{task}
    \begin{condition}
      Gauta skolon žaliavų:
      \begin{enumerate}
        \titem{a} <++>
        \titem{b} <++>
        \titem{c} <++>
        \titem{d} padidėja turto (žaliavų) ir nuosavybės (įsipareigojimų)
          sumos.
      \end{enumerate}
    \end{condition}
    \begin{solution}
      Teisingas atsakymas yra \tref{d}.
    \end{solution}
  \end{task}

  \begin{task}
    \begin{condition}
      Pirkta žaliavų (už pinigus):
      \begin{enumerate}
        \titem{a} <++>
        \titem{b} <++>
        \titem{c} turto apimtis nesikeičia (padidėja žaliavų ir
          sumažėja pinigų sumos);
        \titem{d} <++>
      \end{enumerate}
    \end{condition}
    \begin{solution}
      Teisingas atsakymas yra \tref{c}.
    \end{solution}
  \end{task}

  \begin{task}
    \begin{condition}
      Sumokėta tiekėjams už žaliavas:
      \begin{enumerate}
        \titem{a} <++>
        \titem{b} sumažėja turto (pinigų) ir nuosavybės (įsipareigojimų)
          sumos;
        \titem{c} <++>
        \titem{d} <++>
      \end{enumerate}
    \end{condition}
    \begin{solution}
      Teisingas atsakymas yra \tref{b}.
    \end{solution}
  \end{task}

  \begin{task}
    \begin{condition}
      Sunaudota žaliavų:
      \begin{enumerate}
        \titem{a} sumažėja turto (žaliavų) ir padidėja sąnaudų sumos;
        \titem{b} <++>
        \titem{c} <++>
        \titem{d} <++>
      \end{enumerate}
    \end{condition}
    \begin{solution}
      Teisingas atsakymas yra \tref{a}.
    \end{solution}
  \end{task}

  \begin{task}
    \begin{condition}
      Gauta įrengimų skolon:
      \begin{enumerate}
        \titem{a} <++>
        \titem{b} padidėjo turto (įrengimai) ir nuosavybės
          (įsipareigojimų) sumos;
        \titem{c} <++>
        \titem{d} <++>
      \end{enumerate}
    \end{condition}
    \begin{solution}
      Teisingas atsakymas yra \tref{b}.
    \end{solution}
  \end{task}

  \begin{task}
    \begin{condition}
      Sumokėta dalis skolos už įrengimus:
      \begin{enumerate}
        \titem{a} <++>
        \titem{b} <++>
        \titem{c} sumažėja turto (pinigų) ir nuosavybės (įsipareigojimų)
          sumos;
        \titem{d} <++>
      \end{enumerate}
    \end{condition}
    \begin{solution}
      Teisingas atsakymas yra \tref{c}.
    \end{solution}
  \end{task}

  \begin{task}
    \begin{condition}
      Gautas skolon krovininis automobilis:
      \begin{enumerate}
        \titem{a} padidėjo turto (automobilio) ir nuosavybės
          (įsipareigojimų) sumos;
        \titem{b} <++>
        \titem{c} <++>
        \titem{d} <++>
      \end{enumerate}
    \end{condition}
    \begin{solution}
      Teisingas atsakymas yra \tref{a}.
    \end{solution}
  \end{task}

  \begin{task}
    \begin{condition}
      Gauta sąskaita už patalpų nuomą:
      \begin{enumerate}
        \titem{a} <++>
        \titem{b} <++>
        \titem{c} <++>
        \titem{d} padidėja sąnaudų ir nuosavybės (įsipareigojimai)
          sumos.
      \end{enumerate}
    \end{condition}
    \begin{solution}
      Teisingas atsakymas yra \tref{d}. (Pagal kaupimo principą.)
    \end{solution}
  \end{task}

\end{tasks}

Bus testo pavidalu 10 klausimu ir keli mini uždaviniai.
Atsiminti turtas=nuosavybė, atskirti trumpalaikį nuo ilgalaikio.
Banko sąskaita = 271.
Žinoti klases, ir kada, kuri yra debetuojama, kuri kada kredituojama.
Atskirti materialų nuo nematerialaus.

Negalima naudotis sąskaitų planais.
Turėti skaičiuotuvą.
Maks. 30 minučių.
