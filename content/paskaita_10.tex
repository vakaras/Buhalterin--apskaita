\chapter{Turtas}

\section{Turto sąvoka}

\section{Ekonominių išteklių sąvoka}

Gali būti ne tik materialūs dalykai, bet ir tokie dalykai, kaip
gražus kraštovaizdis ir panašiai.

\section{Įmonės turto klasifikavimas}

\subsection{Pagal …}
\subsection{Pagal užbaigtumo laipsnį}
\subsection{Pagal naudojimo pobūdį}
\subsection{Pagal vaidmenį įmonės viduje}
\subsubsection{Aktyvusis turtas}
\subsubsection{Pasyvusis turtas}
\section{Ilgalaikio turto sąvoka}

Ilgalaikis turtas – tai turtas, kurio naudingas eksploatavimo
laikotarpis ilgesnis negu vienas ataskaitinis laikotarpis,
tai yra kuris naudojamas pajamoms uždirbti ilgiau negu
vienerius metus. Taip pat jis turėtų viršyti minimalią sumą
(rekomenduojama 2000Lt).

\section{Materialinių vertybių apskaita}

\section{Ilgalaikis materialusis turtas}

\subsection{Ilgalaikio materialiojo turto apskaita}
\subsection{Ilgalaikio materialiojo turto įsigijimo savikaina}

Savikaina = pirkimo kaina (+ negrąžintini mokesčiai) + išlaidos
(tiesiogiai susijusios su turto parengimu veikti ir naudoti).

\section{Prieš uždavinius}

1 – ilgalaikis turtas
2 – trumpalaikis turtas
D+, K-

3, 4
D-, K+

5
K

6
D

\chapter{Uždaviniai}

\section{1 užduotis}

Data & Debetas & Kreditas \\
2012-05-05 & 271 & 301 \\
2012-05-06 & 121 & 271 \\
2012-05-07 & Pastaba: sutarties pasirašymas ataskaitos neatsiranda. \\
2012-05-07 & 61 & 271 \\
2012-05-08 & 61 & 271 \\
2012-05-10 & 201 & 271 \\
2012-05-10 & 61 & 271 \\
2012-05-14 & 271 & 50 \\
2012-05-15 & 201 & 443 \\
2012-05-20 & 241 & 50 \\
2012-05-27 & 61 & 271 \\
2012-05-31 & 61 & 271 \\
2012-05-31 & 61 & 443 \\
2012-05-31 & 61 & 201 \\

121 – pastatai ir statiniai;

201 – atsargos;
241 – pirkėjų įsiskolinimas;
271 – sąskaitos bankuose;

301 – įstatinis pasirašytasis kapitalas (pagrindinis kapitalas);

443 – skolos tiekėjams;

50 – pardavimo pajamos;

61 – veiklos sąnaudos;

Data & D 271 K
05 & $3 \cdot 30 000 = 90 000 Lt$ & \\
06 & & $80 000 Lt$ \\
07 & & $1400 Lt$ \\
08 & & $1600 Lt$ \\
10 & & $6000 Lt$ \\
10 & & $300 Lt$ \\
14 & $4200 Lt$ & \\
27 & & $200 Lt$ \\
31 & & $3600 Lt$ \\

Data & D 121 K

\section{Pelno (nuostolio) ataskaita}

Pardavimai (50): & 12 200 Lt
Sąnaudos (51): & 15720 Lt
Pelnas (nuostolis): & (3520) Lt
Pelno mokestis: & 0 Lt
Grynasis pelnas: & (3520) Lt

\section{Balansas}

Ilgalaikis turtas:
+ 80000 Lt Pastatai.

Trumpalaikis turtas:
+ 8000 Lt Pirkėjų įsiskolinimas 241
+ 1100 Lt Pinigai sąskaitoje
+ ? Lt Atsargos

Turtas iš viso:
+ 94960 Lt 

Nuosavas kapitalas:
+ (3520) Lt nuostolis;
+ 90000 Lt akcinis kapitalas;

Mokėtinos sumos ir įsipareigojimai
+ 8440 Lt

\chapter{Testukas}

1. Išleista akcijų:
b) Padidėjo turto ir nuosavybės sumos.

2. Turtinis įnašas kapitalui didinti (pastatas):
a

3. Gauta skolon žaliavų:
d) Padidėja turto (žaliavų) ir nuosavybės (įsipareigojimų) sumos.

4.  Pirkta žaliavų (už pinigus):
c)

5. Sumokėta tiekėjams už žaliavas:
b)

6.
a

7.
b

8. Sumokėta dalis skolos už įrengimus:
c Sumažėja …

9.
a

10. Gauta sąskaita už patalpų nuoma:
d 
// Pagal kaupimo principą.

Bus testo pavidalu 10 klausimu ir keli mini uždaviniai.
Atsiminti turtas=nuosavybė, atskirti trumpalaikį nuo ilgalaikio.
Banko sąskaita = 271.
Žinoti klases, ir kada, kuri yra debetuojama, kuri kada kredituojama.
Atskirti materialų nuo nematerialaus.

Negalima naudotis sąskaitų planais.
Turėti skaičiuotuvą.
Maks. 30 minučių.
