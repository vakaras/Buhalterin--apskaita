\chapter{Turtas}

\setslideprefix{13 - turtas + Pratybos/skaidres.pdf:}

\section{Turto sąvoka}

\slide{2}

Turtu apskaitoje yra laikomi ekonominiai ištekliai, kurie turi
savininką ir kuriais disponuodama įmonė tikisi ateityje gauti
tam tikrą naudą (pelną).

\subsection{Ekonominių išteklių sąvoka}

\slide{3}

Gali būti ne tik materialūs dalykai, bet ir tokie dalykai, kaip
darbuotojai, gražus kraštovaizdis ir panašiai.

\section{Įmonės turto klasifikavimas}

\slide{4}

\subsection{Pagal naudojimo laiką}

\slide{5}

Yra skiriami trumpalaikis ir ilgalaikis turtas.

\subsubsection{Ilgalaikio turto sąvoka}

\slide{11}

Ilgalaikis turtas – tai turtas, kurio naudingas eksploatavimo
laikotarpis ilgesnis negu vienas ataskaitinis laikotarpis,
tai yra kuris naudojamas pajamoms uždirbti ilgiau negu
vienerius metus. Taip pat jis turėtų viršyti minimalią sumą
(rekomenduojama 2000Lt).

\subsubsection{Ilgalaikis materialusis turtas}

\slide{15}

Tai materialusis turtas, kuris teikia įmonei ekonominės naudos,
naudojamas ilgiau nei vienerius metus ir kurio įsigijimo
(pasigaminimo) savikaina yra ne mažesnė už įmonės nustatytą
minimalią ilgalaikio materialiojo turto vertę.

\subsubsection{Ilgalaikio materialiojo turto apskaitos dokumentai}

\slide{16}

\subsubsection{Ilgalaikio materialiojo turto apskaita}

\slide{17}

\slide{18}

\subsubsection{Ilgalaikio materialiojo turto įsigijimo savikaina}

\slide{19}

Savikaina = pirkimo kaina (+ negrąžintini mokesčiai) + išlaidos
(tiesiogiai susijusios su turto parengimu veikti ir naudoti).

Pavyzdys: \slide{20} \slide{21}

\subsubsection{Ilgalaikio materialiojo turto apskaitos ypatumai}

\slide{22}

\slide{23}

\slide{24}

\subsubsection{Ilgalaikio materialiojo turto nusidėvėjimas}

\slide{25}

Nusidėvėjimas – tai riboto naudojimo laiko turto nudėvimosios vertės
priskyrimas sąnaudoms ir paskirstymas per visą planuojamą naudingo
tarnavimo laiką, atsižvelgiant į realų to turto ekonominės
vertės kitimą.

\subsubsection{Ilgalaikio materialiojo turto nusidėvėjimo ypatumai}

\slide{26}

\subsubsection{Ilgalaikio materialiojo turto nusidėvėjimo metodai}

\slide{27}

Tiesiogiai proporcingas metodas: \slide{28}

Dvigubas mažėjančios vertės metodas: \slide{29}, \slide{30}

\subsubsection{Nusidėvėjimo metodų įtaka veiklos pelnui}

\slide{31}

\subsubsection{Ilgalaikio materialiojo turto nusidėvėjimas neskaičiuojamas}

\slide{32}

\subsubsection{Remontas ir rekonstravimas}

\slide{33}

\subsubsection{Ilgalaikio materialiojo turto netekimas}

\slide{34}

\subsubsection{Nebetinkamo naudoti turto likvidavimas}

\slide{35}

\slide{36}

\subsubsection{Turto perleidimas tretiesiems asmenims}

\slide{37}

\subsubsection{Ilgalaikio materialiojo turto nuvertėjimas}

\slide{38}

\subsubsection{Ilgalaikio materialiojo turto perkainojimas}

\slide{39}

\subsection{Pagal užbaigtumo laipsnį}

\slide{6}

Turtas yra skirstomas į žaliavas, nebaigtą gamybą ir gatavus gaminius.

\subsection{Pagal naudojimo pobūdį}

\slide{7}

Yra skirstomas į žemę, pastatus ir statinius, mašinas ir įrengimus,
atsargas, skolas įmonei ir pinigus.

\subsection{Pagal vaidmenį įmonės veikloje}

\slide{8}

Turtas yra skirstomas į aktyvųjį ir pasyvųjį.

\subsubsection{Aktyvusis turtas}

\slide{9}

Aktyviuoju turtu yra laikomos mašinos, įrenginiai, transporto priemonės,
įrankiai ir kitas tiesiogiai įmonės veikloje naudojamas turtas.

\subsubsection{Pasyvusis turtas}

\slide{10}

Pasyviuoju turtu yra laikomi žemė, gamtos ištekliai, pastatai, statiniai,
kiti įrenginiai. Tai turtas, tiesiogiai nedalyvaujantis gaminant
produkciją, bet sudarantis sąlygas šiai veiklai.

\section{Materialinių vertybių apskaita}

\slide{12}

\section{Nematerialusis turtas}

\slide{40}

Turtas, kurio negalima apčiuopti ar paliesti, jis neturi materialiojo
turinio. Jis vis labiau įsigali įmonių veikloje. Nematerialiajam turtui
galima priskirti programinę įrangą, firmos turimas autorines teises,
patentus, licencijas, įmonės prestižą, įvairias kitas privilegijas.

\subsection{Ilgalaikio nematerialiojo turto apskaita}

\slide{41}

Ilgalaikis nematerialusis turtas apskaitoje registruojamas debetuojant
nematerialiojo turto įsigijimo savikainos sąskaitas ir kredituojant
atitinkamas turto ar įsipareigojimų sąskaitas.

\slide{42}

\slide{43}

\subsection{Ilgalaikio nematerialiojo turto apskaitos dokumentai}

\slide{44}

\subsection{Ilgalaikio nematerialiojo turto įsigijimo būdai}

\slide{45}

Nematerialusis turtas gali būti:
\begin{enumerate}
  \item įsigytas iš išorinio šaltinio (už tam tikrą kainą, nemokamai,
    mainais);
  \item pasigamintas / sukurtas;
  \item įgytas po verslo sujungimo – prestižas.
\end{enumerate}

\subsubsection{Nematerialiojo turto, įsigyto iš išorinio šaltinio apskaita}

\slide{46}

\subsubsection{Ilgalaikis nematerialusis turtas}

\slide{47}

Nėra nematerialusis turtas:
\begin{itemize}
  \item darbuotojo mokymo, kvalifikacijos išlaidos;
  \item įmonės veiklos steigimo, reorganizavimo išlaidos;
  \item reklamos, reprezentacijos išlaidos;
  \item tyrimo išlaidos.
\end{itemize}

Yra nematerialusis turtas:
\begin{itemize}
  \item prekių ženklai (įsigyti);
  \item patentai ir licencijos;
  \item autorių ir gretutinės teisės;
  \item kompiuterių programos;
  \item taršos integruotos prevencijos ir kontrolės leidimai;
  \item plėtros išlaidos.
\end{itemize}

\subsubsection{Mainais įsigytas nematerialusis turtas}

\slide{48}

\subsubsection{Ilgalaikio nematerialiojo turto pavyzdžiai}

\slide{49}

\slide{50}

\slide{51}

\subsubsection{Ilgalaikio nematerialiojo turto amortizacija}

\slide{52}

Tiesiogiai proporcingas metodas: \slide{55}.

\subsubsection{Ilgalaikio nematerialiojo turto amortizacijos ypatumai}

\slide{53}

\slide{54}

\section{Turtas ir nuosavybė}

\slide{56}

Turtas parodo kas konkrečiai priklauso įmonei, kitaip tariant kokiu
turtu ji disponuoja. Skolintojų ir skolininkų nuosavybė nusako kiek ir
kam priklauso įmonės turtas.

\section{Prieš uždavinius}

1 – ilgalaikis turtas
2 – trumpalaikis turtas
D+, K-

3, 4
D-, K+

5
K

6
D

\chapter{Uždaviniai}

\slide{58}

\section{1 užduotis}

Data & Debetas & Kreditas \\
2012-05-05 & 271 & 301 \\
2012-05-06 & 121 & 271 \\
2012-05-07 & Pastaba: sutarties pasirašymas ataskaitos neatsiranda. \\
2012-05-07 & 61 & 271 \\
2012-05-08 & 61 & 271 \\
2012-05-10 & 201 & 271 \\
2012-05-10 & 61 & 271 \\
2012-05-14 & 271 & 50 \\
2012-05-15 & 201 & 443 \\
2012-05-20 & 241 & 50 \\
2012-05-27 & 61 & 271 \\
2012-05-31 & 61 & 271 \\
2012-05-31 & 61 & 443 \\
2012-05-31 & 61 & 201 \\

121 – pastatai ir statiniai;

201 – atsargos;
241 – pirkėjų įsiskolinimas;
271 – sąskaitos bankuose;

301 – įstatinis pasirašytasis kapitalas (pagrindinis kapitalas);

443 – skolos tiekėjams;

50 – pardavimo pajamos;

61 – veiklos sąnaudos;

Data & D 271 K
05 & $3 \cdot 30 000 = 90 000 Lt$ & \\
06 & & $80 000 Lt$ \\
07 & & $1400 Lt$ \\
08 & & $1600 Lt$ \\
10 & & $6000 Lt$ \\
10 & & $300 Lt$ \\
14 & $4200 Lt$ & \\
27 & & $200 Lt$ \\
31 & & $3600 Lt$ \\

Data & D 121 K

\section{Pelno (nuostolio) ataskaita}

Pardavimai (50): & 12 200 Lt
Sąnaudos (51): & 15720 Lt
Pelnas (nuostolis): & (3520) Lt
Pelno mokestis: & 0 Lt
Grynasis pelnas: & (3520) Lt

\section{Balansas}

Ilgalaikis turtas:
+ 80000 Lt Pastatai.

Trumpalaikis turtas:
+ 8000 Lt Pirkėjų įsiskolinimas 241
+ 1100 Lt Pinigai sąskaitoje
+ ? Lt Atsargos

Turtas iš viso:
+ 94960 Lt 

Nuosavas kapitalas:
+ (3520) Lt nuostolis;
+ 90000 Lt akcinis kapitalas;

Mokėtinos sumos ir įsipareigojimai
+ 8440 Lt

\chapter{Testukas}

1. Išleista akcijų:
b) Padidėjo turto ir nuosavybės sumos.

2. Turtinis įnašas kapitalui didinti (pastatas):
a

3. Gauta skolon žaliavų:
d) Padidėja turto (žaliavų) ir nuosavybės (įsipareigojimų) sumos.

4.  Pirkta žaliavų (už pinigus):
c)

5. Sumokėta tiekėjams už žaliavas:
b)

6.
a

7.
b

8. Sumokėta dalis skolos už įrengimus:
c Sumažėja …

9.
a

10. Gauta sąskaita už patalpų nuoma:
d 
// Pagal kaupimo principą.

Bus testo pavidalu 10 klausimu ir keli mini uždaviniai.
Atsiminti turtas=nuosavybė, atskirti trumpalaikį nuo ilgalaikio.
Banko sąskaita = 271.
Žinoti klases, ir kada, kuri yra debetuojama, kuri kada kredituojama.
Atskirti materialų nuo nematerialaus.

Negalima naudotis sąskaitų planais.
Turėti skaičiuotuvą.
Maks. 30 minučių.
